% This is a template for lecture notes.
\documentclass{article}
%\usepackage[UTF8]{ctex}
\usepackage{amssymb}
\usepackage{amsmath}
\usepackage{amsthm}
\usepackage{geometry}
\usepackage{booktabs}
\usepackage{bm}
\usepackage{tcolorbox}
\usepackage{indentfirst}
%\CTEXoptions[today=old]

%Some commonly used notations
%\geometry{a4paper,bottom = 3cm,left = 3cm, right = 3cm}

%for reference
\usepackage{hyperref}
\usepackage[capitalise]{cleveref}
\crefname{enumi}{}{}

\newtheorem{theorem}{Theorem}
\newtheorem{lemma}[theorem]{Lemma}
\newtheorem{proposition}[theorem]{Proposition}
\newtheorem{corollary}[theorem]{Corollary}
\newtheorem{fact}[theorem]{Fact}
\newtheorem{definition}[theorem]{Definition}
\newtheorem{remark}[theorem]{Remark}
\newtheorem{question}[theorem]{Question}
\newtheorem{answer}[theorem]{Answer}
\newtheorem{exercise}[theorem]{Exercise}
\newtheorem{example}[theorem]{Example}
%\newenvironment{proof}{\noindent \textbf{Proof:}}{$\Box$}
\newtheorem{observation}[theorem]{Observation}

%to use newcommand for convenience
\newcommand\field{\mathbb{F}}
\newcommand\Real{\mathbb{R}}
\newcommand\Q{\mathbb{Q}}
\newcommand\Z{\mathbb{Z}}
\newcommand\complex{\mathbb{C}}

%this is how we define operators.
\DeclareMathOperator{\rank}{rank} % rank

\title{Indicator function and Infinite events}
\author{Ji Jiabao}
\date{\today}

\begin{document}
\maketitle

This is just a simple discussion on $Exer.2, Exer.3$ in the handout 0324
\begin{exercise} 
    \text{Show that } 
    $$\lim_{n \to \infty}sup\textbf{1}_{E_n} = \textbf{1}_{\lim \limits_{n\rightarrow \infty} sup E_n}$$
    $$\lim_{n \rightarrow \infty}inf\textbf{1}_{E_n} = \textbf{1}_{\lim \limits_{n\rightarrow \infty} inf E_n}$$
    
\begin{proof}
    
    To prove this, we need to show the equations keep for any $\omega \in \mathcal{F}$.

    For left side, derive the function as defined in 2.5(a) in textbook for real consequences, we have
    \begin{equation*}
        \begin{aligned}
            \lim \limits_{n \to \infty}sup \textbf{1}_{E_n}(\omega) & =   
                \downarrow \lim\limits_{m}\{ \mathop{sup}\limits_{n \geq m} {\textbf{1}_{E_n}(\omega)}\} \\
            & =
            \left\{
                \begin{array}{ll}
                1 & \forall m \in \mathbb{N}, \exists n \ge m, \omega \in E_n \\ 
                0 & \forall m \in \mathbb{N}, \forall n \ge m, \omega \notin E_n
                \end{array}
            \right.    
        \end{aligned}
    \end{equation*}
    Thus, we have 
    \begin{equation}
        left = \label{equ:p1}
        \left\{
            \begin{array}{ll}
            1 & \forall m \in \mathbb{N}, \omega \in \cup_{n \ge m} E_n \Rightarrow \omega \in \cap_{m \in \mathbb{N}} \cup_{n \geq m} E_{n}\\ 
            0 & \forall m \in \mathbb{N}, \omega \notin \cup_{n \ge m} E_n, \Rightarrow \omega \notin \cap_{m \in \mathbb{N}} \cup_{n \geq m} E_{n}
            \end{array}
        \right.    
    \end{equation}
    For right side, the function can be written as
    \begin{equation} 
        \textbf{1}_{\lim \limits_{n\rightarrow \infty}} sup E_n( \omega) = \label{equ:p2}
        \left\{
            \begin{array}{ll}
            1 & \omega \in \cap_{m \in \mathbb{N}} \cup_{n \geq m} E_{n} \\
            0 & \omega \notin \cap_{m \in \mathbb{N}} \cup_{n \geq m} E_{n}
            \end{array}
        \right.    
    \end{equation}
    Based on (\ref{equ:p1}) and (\ref{equ:p2}), we have proved $\lim_{n \to \infty}sup\textbf{1}_{E_n} = \textbf{1}_{\lim \limits_{n\rightarrow \infty} sup E_n}$
    
    For $lim inf$, it is a similar proof.
\end{proof}    
\end{exercise}

\begin{exercise}

    Let $\left(y_{n}\right)_{n \in \mathbb{N}}$ be a sequence of reals from [0,1] such that $\sum_{n \in \mathbb{N}} y_{n}=\infty .$ 
    Show that $\prod_{n \in \mathbb{N}}\left(1-y_{n}\right)=0$

\begin{proof}
    \hspace*{1em}

    $\forall y_n \ge 0, 1 - y_n \le e^{-y_n}$, since $f = 1-x - e^{-x}$, $f(0) = 0, f^{'}(x) = -1 + e^{-x} > 0(x \ge 0)$
    So we have 
    \begin{equation*}
        0 \le \prod_{n \in \mathbb{N}}(1 - y_n) \le \prod_{n \in \mathbb{N}}e^{-y_n} \le e^{-\sum_{n\in \mathbb{N}} y_n} \le e^{-\infty} = 0
    \end{equation*}
    Then we have $\prod_{n \in \mathbb{N}}\left(1-y_{n}\right)=0$.

    And this exercise is used in the proof of \text{Borel-Cantelli lemma \uppercase\expandafter{\romannumeral2}}
\end{proof}
\end{exercise}

\end{document}