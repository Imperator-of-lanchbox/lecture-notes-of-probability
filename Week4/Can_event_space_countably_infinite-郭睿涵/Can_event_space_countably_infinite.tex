\documentclass[UTF8]{ctexart}

\usepackage{graphicx,amsmath,amssymb,amsthm, boxedminipage}

\usepackage{algorithm}
\usepackage{algpseudocode}


\newtheorem{theorem}{Theorem}%[section]
\newtheorem{proposition}[theorem]{Proposition}
\newtheorem{lemma}[theorem]{Lemma}
\newtheorem{corollary}[theorem]{Corollary}
\newtheorem{definition}[theorem]{Definition}

\newcommand{\scalar}[2]{\ensuremath{\langle #1, #2\rangle}}
\newcommand{\floor}[1]{\left\lfloor #1 \right\rfloor}
\newcommand{\ceil}[1]{\left\lceil #1 \right\rceil}
\newcommand{\norm}[1]{\|#1\|}
\newcommand{\pfrac}[2]{\left(\frac{#1}{#2}\right)}
\newcommand{\nth}[1]{#1\textsuperscript{th}}

% \newcommand{\nth}[1]{#1\textsuperscript{th}}
\newcommand{\E}{\mathop{\mathbb{E\/}}}
\newcommand{\N}{\mathbb{N}}

\newcommand{\R}{\mathbb{R}}

\newtheorem{exercise}[theorem]{Exercise}
\newtheorem{exerciseD}[theorem]{*Exercise}
\newtheorem{exerciseDD}[theorem]{**Exercise}

\let\oldexercise\exercise
\renewcommand{\exercise}{\oldexercise\normalfont}

\let\oldexerciseD\exerciseD
\renewcommand{\exerciseD}{\oldexerciseD\normalfont}

\let\oldexerciseDD\exerciseDD
\renewcommand{\exerciseDD}{\oldexerciseDD\normalfont}


 
\newtheorem{problem}{Problem}
\newtheorem*{problem*}{Problem}
\begin{document}

\author{郭睿涵}
\date{\today}

\title{Can an event space be a countably infinite set?\\
  \vspace{3mm}
{\large	Shanghai Jiaotong University, Spring 2020\\
}
}
\maketitle

\paragraph{Exercise 2.}Can an event space be a countably infinite set?
\paragraph{Solution}First of all, event space in measure theory based probability, it is required to be a sigma-algebra. Then we need to proof a sigma-algebra that contains infinitely many sets must be uncountable.\\
Assume we have a set $\mathbb{X}$, and an infinite sigma-algebra $\mathbb{S}$ on it. I want to proof that $\mathbb{S}$ is uncountable by contradiction.\\
\paragraph{Assumption} $\mathbb{S}=\{A_{i}\}_{i=1}^{\infty}$,$B_{x} =\cap_{x\in A_{i}}A_{i}$. Due to $\mathbb{S}$ is countable, $B_{x}$ is made of countable intersection, which means it belongs to $\mathbb{S}$.\\
\paragraph{Lemma} $c \in B_x \cap B_y \rightarrow B_x = B_y$
\paragraph{Proof} If $x \notin B_c$, $B_x \backslash B_c \subset S$ with $x \in B_x \backslash B_c$. But $B_x$ is the intersection of all the intervals containing x. Therefore $B_x \backslash B_c = B_x$, which means,$B_x = B_c$.\\
Analogouly, we have $B_y = B_c$. So we've got $B_x = B_y$ when $c \in B_x \cap B_y$.
\\\\
If there are finite sets of the form $B_x$, then: $\mathbb{S}$ is a union of a finite number of disjoint sets, which leads to $\mathbb{S}$ is finite.
\\\\
If there are countable-infinite sets of the form $B_x$, then suppose $G=\{B_x\}_{x\in X}$.By taking all the possible disjoint unions from G you can form $\|P(G)\|$ new different sets(P means power set), hense an uncountable number of different sets.(G is countably infinite, then P(G) is uncountable)
\\\\
Notice that every possible union of sets in G is a set that belongs to S, since $B_x \in S$ and S is a sigma-algebra. This means that S should be uncountable in order to contain this uncountable number of all possible different unions of the sets in the family G.
\\\\
In conclusion, an event space can't be a countably infinite set.
\end{document}
