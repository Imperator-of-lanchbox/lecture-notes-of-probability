\documentclass{article}
% \usepackage[UTF8]{ctex}
\usepackage{amssymb}
\usepackage{amsmath}
\usepackage{amsthm}
\usepackage{geometry}
\usepackage{booktabs}
\usepackage{bm}
\usepackage{tcolorbox}
\usepackage{setspace}
\renewcommand{\baselinestretch}{1.3}

%for reference
\usepackage{hyperref}
\usepackage[capitalise]{cleveref}
\crefname{enumi}{}{}

\newtheorem{theorem}{Theorem}
\newtheorem{lemma}[theorem]{Lemma}
\newtheorem{proposition}[theorem]{Proposition}
\newtheorem{corollary}[theorem]{Corollary}
\newtheorem{fact}[theorem]{Fact}
\newtheorem{definition}[theorem]{Definition}
\newtheorem{remark}[theorem]{Remark}
\newtheorem{question}[theorem]{Question}
\newtheorem{answer}[theorem]{Answer}
\newtheorem{exercise}[theorem]{Exercise}
\newtheorem{example}[theorem]{Example}
\newtheorem{observation}[theorem]{Observation}


%this is how we define operators.
\DeclareMathOperator{\rank}{rank} % rank

\title{The continuity of distribution function}
\author{Guo Linsong~~518030910419}
\date{\today}

\begin{document}
    \maketitle

\begin{tcolorbox}
    \begin{question}\label{question}
    Construct an example to show that the distribution function of a random variable may not be left-continuous.
   \end{question}
\end{tcolorbox}



\begin{theorem}
Let $X$ be a random variable.The distribution function $F_X(x)=\mathcal{L}_X(-\infty,x]$ is right-continuous.
\end{theorem}

\begin{proof}
    Let $\{a_n=\frac{1}{n},n\in\mathbb{N}\}$ be a countable sequence of numbers.Let $A_n=(-\infty,x+a_n]$ and $A=(-\infty,x]$.Then $A_n,A\in \mathcal{B}$ and $A_n\downarrow A$.$\mathcal{L}_x$ is a probability measure on $(R,\mathcal{B})$.According to lemma $1.10(b)$ in textbook, we have $$\mathcal{L}_X(A_n)\downarrow\mathcal{L}_X(A)$$
    Thus $$\mathcal{L}_X(-\infty, x]=\lim\limits_{n\rightarrow\infty}\mathcal{L}_X(-\infty, x+\frac{1}{n}]$$
    Together with the monotonicity of $\mathcal{L}_X$, we have $\mathcal{L}_X(-\infty, x]=\mathcal{L}_X(-\infty, x^{+}]$.Thus $F_X(x)=F_X(x^{+})$,the distribution function $F_X(x)=\mathcal{L}_X(-\infty,x]$ is right-continuous.
\end{proof}
 
However,The distribution function may not left-continuous.Considering two consecutive coin tosses,Let $X$ be the number of heads.

$$F_{X}(x)=  \mathcal{L}_X(x)=\left\{\begin{array}{ll}
0, & \text { if } x\in(-\infty,0) \\
1 / 4, & \text { if } x \in [0,1) \\
3 / 4, & \text { if } x \in [1,2) \\
1, & \text { if } x \in [2,+\infty)
\end{array}\right.$$

We have $F_X(0^-)\neq F_X(0)=F_X(0^+),F_X(1^-)\neq F_X(1)=F_X(1^+)$ and $F_X(2^-)\neq F_X(2)=F_X(2^+)$.Therefore, $F_X(x)$ is not left-continuous.

\end{document}
