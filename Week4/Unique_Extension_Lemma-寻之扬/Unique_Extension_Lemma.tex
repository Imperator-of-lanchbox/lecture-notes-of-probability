% !TEX program = xelatex
\input{D:/template.tex}

\newcommand{\D}{\mathcal{D}}
\newcommand{\I}{\mathcal{I}}
\newcommand{\Leb}{\operatorname{Leb}}
% \operatorname
\newcommand{\ds}{$d$-system}

\title{The $\sigma$-Finite Constraint on Uniqueness of Extension Lemma}
\author{Zhiyang Xun}

\begin{document}

\maketitle      
\begin{tcolorbox}
    \begin{lemma}
        \label{main}
        Let $I$ be a $\pi$-system on a set $S$ and let $\Sigma = \sigma(I)$.
        Suppose that $\mu_1$ and $\mu_2$ are two measures on $(S, \Sigma)$ such that $\mu_1 =\mu_2$ on $I$. If there is a sequence $A_n \in I$ with $ \bigcup_{n = 1}^{\infty} A_n = S$ and
        $\mu_(A_n) < \infty$,
        then $\mu_1 = \mu_2$.
    \end{lemma}    
\end{tcolorbox}

We will give a proof for this lemma and show the reason that we need to add the $\sigma$-finite constraint.

% This article was intended to show that the ``$\mu_1(S) = \mu_2(S) < \infty$'' constraint can be relaxed to ``$\mu_1$ and $\mu_2$ are $\sigma$-finite and $\mu_1(S) = \mu_2(S)$''. However, while I was trying to prove that, I found that the conclusion does not hold when $\mu_1$ or $\mu_2$ is not finite.

First we need to introduce the definition of a $d$-system \footnote{The word ``$d$-system'' is an abbreviation of Dynkin System to honor Eugene Dynkin. It is sometimes referred to as $\lambda$-system (Dynkin himself used this term).}.
\begin{definition}
    Let $S$ be a set, and let $\D$ be a collection of subsets of $S$. Then $\D$ is called a $d$-system (on $S$) if
    \begin{enumerate}[label=(\alph*), topsep=0pt]
        \setlength{\itemsep}{0pt}
        \item $S \in \D$,
        \item if $A, B \in \D$ and $A \subseteq B$ then $B \setminus A \in \D$,
        \item if $A_1, A_2, A_3, ...$ is a sequence of subsets in $\D$ and $A_n \subseteq A_{n+1}$ for all $n \geq 1$, then $ \bigcup _{n=1}^{\infty }A_{n}\in \D$.
    \end{enumerate}
\end{definition}

   It's easy to verify the intersection of {\ds}s is still a \ds. Thus similar to the definition of $\sigma(C)$, for $C \subseteq 2^S$ we denote the smallest \ds\ containing $C$ by $d(C)$.  We present Dynkin's lemma here without proof.
\begin{lemma}
    (Dynkin's Lemma) If $\I$ is a $\pi$-system, then $d(\I) = \sigma(\I)$.
\end{lemma}

Now we will give a proof for the finite case of \cref{main}: ``If $\mu_1(S) = \mu_2(S) < \infty$, then $\mu_1 = \mu_2$''. Define
\[ \D = \{F \in \Sigma : \mu_1(F) = \mu_2(F)\} \]
Evidently $I \subseteq \D$. Thus we simply need to verify $\D$ is a \ds\ on $S$. Then according to Dynkin's Lemma, we can see 
\[ \Sigma = \sigma(\I) = d(\I) \subseteq \mathcal{D}\]
which implies $\mu_1 = \mu_2$ on $\Sigma$.
 
First, the fact that $S \in \D$ is given. Then, if $A, B \in \D$ and $A \subseteq B$,  
\begin{equation}
    % \tag{*}
    \label{t}
    \mu_1(B \setminus A) = \mu_1(B) - \mu_1(A) = \mu_2(B) - \mu_2(A) = \mu_2(B \setminus A)
\end{equation}
so that $B \setminus A \in \D$. Finally, if $A_n \in \mathcal{D}$ and $A_n \subseteq A_{n+1}$. Due to the monotone-convergence properties of measures, 
\[\mu_1(\bigcup _{n=1}^{\infty }A_{n}) = \lim_{n \to \infty}\mu_1(A_n) = \lim_{n \to \infty}\mu_2(A_n) = \mu_2(\bigcup _{n=1}^{\infty }A_{n})\] 
Hence, $\D$ is a \ds.

This proof is only valid for the finite case because correctness of \cref{t} may cause problem. We need to notice whether $\infty - \infty = \infty - \infty$ is uncertain. Thus we can see why \cref{ex4} has two extensions.
% \begin{tcolorbox}
\begin{example}
    \label{ex4}
    Let $S = (0, 1]$, and let $\Sigma_0$ be the subsets of $S$ which are finite unions of
    disjoint left-open right-closed intervals. Obviously, $\Sigma_0$ is a $\pi$-system. Define 
    \[
        \mu_0(F) = \left\{ 
            \begin{array}{lcl}
                0 & & \text{if } F = \emptyset \\
                \infty & & \text{if } F \neq \emptyset
            \end{array}
        \right.
    \]
    We find two ways to extend $\mu_0$ to $\mathcal{B} (0, 1]$: $\mu_1$ and $\mu_2$:
    \begin{align*}
        \mu_1(F) &= \left\{ 
            \begin{array}{lcl}
                0 & & \text{if } F = \emptyset \\
                \infty & & \text{if } F \neq \emptyset
            \end{array}
        \right. \\
        \mu_2(F) &= \text{number of elements in } F.
    \end{align*}
   It's easy to check they are both correct. However, we notice that 
   \[\mu_1(\{1\}) = \infty \neq 1 = \mu_2(\{1\})\]
   That's because 
   $\{1\} = (0, 1] \setminus \bigcup_{n = 1}^{\infty }(0, 1 - \frac{1}{2^n}]$, but $\mu(\{1\})$ cannot be determined by $\infty - \infty$.
\end{example}
% \end{tcolorbox}

Now let us consider the $\sigma$-finite case. For any arbitrary set $A \in \Sigma$ satisfying $\mu_1(A) = \mu_2(A) < \infty$, it's easy to see $ \D = \{F \in \Sigma : \mu_1(F \cap A) = \mu_2(F \cap A)\} = \Sigma$ by repeating the argument above, because for any $F \in \Sigma$, $\mu_i(F \cap A) \leq \mu_i(A) < \infty$. 

By assumption, there exists a sequence $A_n \in I$ with $ \bigcup_{n = 1}^{\infty} A_n = S$. Define $B_n = A_n \setminus \bigcup_{k = 1}^{n - 1} A_k$. Since $B_n$ is pairwise distinct and $\bigcup_{n = 1}^{\infty} B_n = S$, for $F \in \Sigma$, 
\[\mu_1(F) = \mu_1(F \cap \bigcup_{n = 1}^{\infty} B_n ) = \mu_1(\bigcup_{n = 1}^{\infty} (F \cap B_n)) = \sum_{n=1}^\infty \mu_1(F \cap B_n)\]
\[\mu_2(F) = \mu_2(F \cap \bigcup_{n = 1}^{\infty} B_n ) = \mu_2(\bigcup_{n = 1}^{\infty} (F \cap B_n)) = \sum_{n=1}^\infty \mu_2(F \cap B_n)\]

Because $B_n \in \Sigma$, we have 
\[\mu_1(B_n) = \mu_1(B_n \cap A_n) = \mu_2(B_n \cap A_n) = \mu_2(B_n) < \infty\]
Now we can see that $\mu_1(F \cap B_n) = \mu_2(F \cap B_n)$. Hence, $\mu_1(F) = \mu_2(F)$ for any $F \in \Sigma$. This finishes the proof.

Notice that the so-called ``$\sigma$-finite'' constraint in this lemma is a stronger constraint than ``$\mu_1$ and $\mu_2$ are $\sigma$-finite''. Here we provide a counterexample for the latter condition.

\begin{example}

Let $S = \R$, and let $I = \{[0, u] | u \in [0, +\infty)\} \cup \R$. Obviously $I$ is a $\pi$-system on $\R$. Furthermore, $\sigma(I) \subset \mathcal{B}(\R)$. We use $\mu_1$ to denote Lebesgue measure on $\sigma(I)$. Thus
 $(\R, \sigma(I), \mu_1)$ is a $\sigma$-finite measure space.

We can construct another $\sigma$-finite measure $(\R, \sigma(I), \mu_2)$ where $\mu_2(F) = \Leb(F \cap [0, +\infty))$. Clearly $\mu_1 = \mu_2$ on $I$, but $\mu_1((-\infty, 1]) = \infty \neq 1 = \Leb([0, 1]) = \mu_2((-\infty, 1])$.
 
\end{example}

\end{document}