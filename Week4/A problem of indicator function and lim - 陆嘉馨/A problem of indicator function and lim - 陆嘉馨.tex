\documentclass{article}
\usepackage{cite}
% \usepackage[UTF8]{ctex}
\usepackage{amssymb}
\usepackage{amsmath}
\usepackage{amsthm}
\usepackage{geometry}
\usepackage{booktabs}
\usepackage{bm}
\usepackage{enumerate}
\usepackage{tcolorbox}
% \CTEXoptions[today=old]
%Some commonly used notations
%\geometry{a4paper,bottom = 3cm,left = 3cm, right = 3cm}

%for reference
\usepackage{hyperref}
\usepackage[capitalise]{cleveref}
\crefname{enumi}{}{}

\newtheorem{theorem}{Theorem}
\newtheorem{lemma}[theorem]{Lemma}
\newtheorem{proposition}[theorem]{Proposition}
\newtheorem{corollary}[theorem]{Corollary}
\newtheorem{fact}[theorem]{Fact}
\newtheorem{definition}[theorem]{Definition}
\newtheorem{remark}[theorem]{Remark}
\newtheorem{question}[theorem]{Question}
\newtheorem{answer}[theorem]{Answer}
\newtheorem{exercise}[theorem]{Exercise}
\newtheorem{example}[theorem]{Example}
%\newenvironment{proof}{\noindent \textbf{Proof:}}{$\Box$}
\newtheorem{observation}[theorem]{Observation}

%to use newcommand for convenience
\newcommand{\mbb}{\mathbb}
\newcommand{\mbf}{\mathbf}
\newcommand{\mbz}{\mathbb{Z}}
\newcommand{\mbn}{\mathbb{N}}
\newcommand{\mbp}{\mathbb{P}}
\newcommand{\mbh}{\mathbb{H}}
\newcommand{\mbq}{\mathbb{Q}}
\newcommand{\vep}{\varepsilon}
\newcommand{\rd}{\mathrm{d}}
\newcommand{\inv}{^{-1}}
\newcommand{\hp}{^\prime}
\newcommand{\mca}{\mathcal{A}}
\newcommand{\mcb}{\mathcal{B}}
\newcommand{\mcc}{\mathcal{C}}
\newcommand{\mcm}{\mathcal{M}}
\newcommand{\mcr}{\mathcal{R}}
\newcommand{\mcf}{\mathcal{F}}
\newcommand{\mfa}{\mathfrak{A}}
\newcommand{\mfb}{\mathfrak{B}}
\newcommand{\mfc}{\mathfrak{C}}
\newcommand{\mfi}{\mathfrak{I}}
\newcommand{\Iff}{\mbox{iff }}
\newcommand{\AND}{\mbox{ and }}

\title{A Problem of Indicator Function and $\lim$}
\author{Lu Jiaxin\\
Student ID: 518030910412}
\date{\today}

\begin{document}
    \maketitle

\begin{tcolorbox}

\begin{definition}[\bf Indicator Function]
    The indicator function of a subset $A$ of a set $X$ is a function
    $$
    1_A : X \rightarrow \{0, 1\}
    $$
    defined as
    $$
    1_A (x) := \left\{
        \begin{aligned}
            1 & \mbox{ if } x \in A, \\
            0 & \mbox{ if } x \notin A.\\
        \end{aligned}
        \right.
    $$
\end{definition}
\begin{exercise}
    Show that 
    \begin{align}
        \limsup_{n\to \infty} 1_{E_n} &= 1_{\limsup_{n\to\infty} E_n}\\
        \liminf_{n\to \infty} 1_{E_n} &= 1_{\liminf_{n\to\infty} E_n}
    \end{align}
    % $$\limsup_{n\to \infty} 1_{E_n} = 1_{\limsup_{n\to\infty} E_n}$$
    % and 
    % $$\liminf{n\to \infty} 1_{E_n} = 1_{\liminf_{n\to\infty} E_n}$$.
\end{exercise}

\end{tcolorbox}

\begin{proof}
    (1) By definition $\limsup E_n = \bigcap{m\in\mbn} \bigcup{l\geq m} E_l$. In other words, $x\in \limsup E_n$ if and only if for all $m\in\mbn$ such that $x\in E_l$ for some $l\geq m$.

    Thus $1_{\limsup_{n\to\infty} E_n} (x) = 1$ if and only if for all $m\in\mbn$ such that for some $l \geq m$ we have $x\in E_l$.

    Now consider a sequence of $(x_n \mid n\in\mbn)$ of real numbers $x_n$. By definition, we have $\limsup_{n\to\infty} x_n = \inf_{m\in\mbn}\sup_{l\geq m} x_l$. So $\limsup_{n\to\infty}1_{E_n}(x) = 1$ if and only if for all $m\in\mbn$ such that there is some $l\geq m$ we have that $\sup_{l\geq m}1_{E_n}(x) = 1$. This also holds if and only if for all $m\in\mbn$ such that there is some $l \geq m$ we have that $x\in E_l$.

    By these, we have $1_{\limsup_{n\to\infty}E_n}(x) = 1$ if and only if $\limsup_{n\to\infty} 1_{E_n}(x) = 1$. And these also yields $1_{\limsup_{n\to\infty}E_n}(x) = 0$ if and only if $\limsup_{n\to\infty} 1_{E_n}(x) = 0$ and consequently $1_{\limsup_{n\to\infty}E_n}(x) = \limsup_{n\to\infty} 1_{E_n}(x)$.

    Thus, $\limsup_{n\to \infty} 1_{E_n} = 1_{\limsup_{n\to\infty} E_n}$ holds.\\

    (2) By definition $\liminf E_n = \bigcup_{m\in\mbn} \bigcap_{l\geq m} E_l$. In other words, $x\in \liminf E_n$ if and only if there is an $m\in\mbn$ such that $x\in E_l$ for all $l\geq m$.

    Thus, $1_{\liminf_{n\to\infty} E_n} (x) = 1$ if and only if there is some $m\in\mbn$ such that for all $l\geq m$ we have $x\in E_l$.

    Also, from (1) we can have $\liminf_{n\to\infty} x_n = \sup_{m\in\mbn}\inf_{l\geq m} x_l$. So $\liminf_{n\to\infty}1_{E_n}(x) = 1$ if and only if there is some $m\in\mbn$ such that for all $l\geq m$ we have that $\inf_{l\geq m}1_{E_n}(x) = 1$. This also holds if and only if there is some $m \in \mbn$ such that for all $l\geq m$ we have that $x\in E_l$.

    By these, we have $1_{\liminf_{n\to\infty}E_n}(x) = 1$ if and only if $\liminf_{n\to\infty} 1_{E_n}(x) = 1$. And these also yields $1_{\liminf_{n\to\infty}E_n}(x) = 0$ if and only if $\liminf_{n\to\infty} 1_{E_n}(x) = 0$ and consequently $1_{\liminf_{n\to\infty}E_n}(x) = \liminf_{n\to\infty} 1_{E_n}(x)$.

    Thus, $\liminf_{n\to \infty} 1_{E_n} = 1_{\liminf_{n\to\infty} E_n}$ holds.

\end{proof}


\end{document}