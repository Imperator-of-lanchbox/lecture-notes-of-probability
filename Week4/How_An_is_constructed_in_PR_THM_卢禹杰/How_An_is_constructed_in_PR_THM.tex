\documentclass[UTF8]{article}

\usepackage[T1]{fontenc}
\usepackage{textcomp}
\usepackage{theorem}
% \usepackage[dutch]{babel}
\usepackage{amsmath, amssymb}
\usepackage{import}
\usepackage{pdfpages}
\usepackage{transparent}
\usepackage{xcolor}
\usepackage{enumerate}
\usepackage{setspace} 
%\usepackage{ebgaramond}
%\fontfamily{ebgaramond}
% ------------- coding style setting --------------%
%\usepackage{libertine}
\usepackage{listings}
\usepackage{enumitem}
\setlist{nosep}
\definecolor{codegreen}{rgb}{0,0.6,0}
\definecolor{codegray}{rgb}{0.5,0.5,0.5}
\definecolor{codepurple}{rgb}{0.58,0,0.82}
\definecolor{backcolour}{rgb}{0.95,0.95,0.92}

\lstdefinestyle{mystyle}{
    backgroundcolor=\color{backcolour},
    commentstyle=\color{codegreen},
    keywordstyle=\color{magenta},
    numberstyle=\tiny\color{codegray},
    stringstyle=\color{codepurple},
    basicstyle=\ttfamily\footnotesize,
    breakatwhitespace=false,
    breaklines=true,
    captionpos=b,
    keepspaces=true,
    numbers=left,
    numbersep=5pt,
    showspaces=false,
    showstringspaces=false,
    showtabs=false,
    tabsize=2
}

\lstset{style=mystyle}

% ----------------- geometry and fancy head -----------

\usepackage{geometry}
\geometry{left=2.5cm,right=2.5cm,top=3cm,bottom=3cm}
\usepackage[many]{tcolorbox}
\tcbuselibrary{skins, breakable, theorems}

\usepackage{fancyhdr}
\usepackage{syntonly} % dubugging
% \syntaxonly
\fancypagestyle{mainFancy}{
    \fancyhf{}
    %\renewcommand\headrulewidth{0pt}       % 页眉横线
    %\renewcommand\footrulewidth{0pt}
    
    \fancyhead[L]{Probability Theory}       % 页眉章标题
    \fancyhead[R]{Assignment}         % 页眉文章题目
    \fancyfoot[C]{\thepage}                 % 页眉编号
}
\pagestyle{mainFancy}


% --------------- environment setting ------------------

\newtheorem{thm}{Theorem}
\newtheorem{pro}{Problem}
\newtheorem{lemma}{Lemma}
\newtheorem{defi}{Definition}
\newtheorem{li}{Example}
\newenvironment{proof}{\paragraph{Proof:}}{\hfill$\square$}
\newenvironment{jie}{\paragraph{Show:}}{\hfill$\square$}

\numberwithin{pro}{section}
\numberwithin{thm}{section}
\numberwithin{defi}{section}
\numberwithin{lemma}{section}


\tcolorboxenvironment{pro}{
  enhanced,
  borderline={0.4pt}{0.4pt}{black},
  boxrule=0.4pt,
  colback=white,
  coltitle=black,
  sharp corners,
}
\tcolorboxenvironment{thm}{
  enhanced,
  borderline={0.4pt}{0.4pt}{black},
  boxrule=0.4pt,
  colback=white,
  coltitle=black,
  sharp corners,
}
\tcolorboxenvironment{lemma}{
  enhanced,
  borderline={0.4pt}{0.4pt}{black},
  boxrule=0.4pt,
  colback=white,
  coltitle=black,
  sharp corners,
}
\tcolorboxenvironment{defi}{
  enhanced,
  borderline={0.4pt}{0.4pt}{black},
  boxrule=0.4pt,
  colback=white,
  coltitle=black,
  sharp corners,
}

% ----------------- macros and command -----------------
\usepackage{stmaryrd} 
\newcommand\contra{\scalebox{1.5}{$\lightning$}}
\definecolor{correct}{HTML}{009900}
\newcommand\correct[2]{\ensuremath{\:}{\color{red}{#1}}\ensuremath{\to }{\color{correct}{#2}}\ensuremath{\:}}
\newcommand\green[1]{{\color{correct}{#1}}}

% horizontal rule
\newcommand\hr{
		    \noindent\rule[0.5ex]{\linewidth}{0.5pt}
	}
\def\mf(#1){\mathfrak{#1}} 
\def\setn(#1,#2){\left\{#1_1,#1_2,\cdots, #1_#2 \right\}  }


\let\implies\Rightarrow
\let\impliedby\Leftarrow
\let\iff\Leftrightarrow
\let\ldots\cdots


\newcommand\dif{\,\mathrm{d}}
\newcommand\e{\,\mathrm{e}}
\newcommand\R{\,\mathbb{R}}
\newcommand\Q{\,\mathbb{Q}}
\newcommand\C{\,\mathbb{C}}
\newcommand\N{\,\mathbb{N}}
\newcommand\A{\,\mathbb{A}}
\newcommand\Z{\,\mathbb{Z}}
\newcommand\ep{\,\varepsilon}
\newcommand\F{\,\varphi}
\newcommand\T{\,\mathbb{T}}
\newcommand\HH{\,\mathbb{H}}
\author{Yujie Lu \quad Haichen Dong \\ \textsc{ACM Class 18} }

\title{How  $A_{n}$ is constructed in the proof of Poincare' Recurrence Theorem}
\begin{document}
\maketitle
The construction of $A_{n}$ in the proof given by Mr. Wu is really tricky when 
I take the first look. 
\[
		A_{n} = \{x\in E  \mid  x \not \in T^{-kn}\left( E \right) , \forall k\} 
.\] 
And now I try to give some intuition behind the construction. 
First let us take a look at another version of Poincare' Recurrence Theorem:
\begin{pro}
		Let $\left( \Omega,\mathcal{F},P \right) $ be a probability space, and 
		let $T$ be a map from $\Omega$ to itself with the property 
		that $T\left( F \right) \in \mathcal{F}, \forall F\in \mathcal{ F}$. Plus
		$P\left( T\left( F \right)  \right) = P\left( F \right)$. 

		For $\forall E \in \mathcal{F}$, prove that $P\left( E \setminus
		\limsup_{n \to \infty}T^{n}\left( E \right)  \right) =0$
\end{pro}
\begin{proof}
		First we use a simple trick in set operation 
		\[
				E \setminus \bigcap_{n\ge 1} R_{n} = \bigcup_{n\ge 1} \left( E\setminus 
				R_{n} \right) 
		.\] 
		to simpify the problem. Applying the equation above, our goal as
		\[
				P\left( \bigcup_{k\ge 1} \left( 
				E \setminus \bigcup_{n\ge k} T^{n}(E) \right)  \right) = 0
		.\] 
		Otherwise we assume 
		\[
				\exists k\ge 1, s.t. \  P\left( E \setminus \bigcup_{n\ge k} 
				T^{n}\left( E \right) \right)  > 0
		.\] 
		Therefore this problem is equivalent to prove 
		\[
				P\left( A_k \right) =0,\text{ in which }  A_k = \{ x \in E| x \not\in \bigcup_{n\ge k} 
				T^{n}\left( E \right) \} 
		.\] 
		Notice that \textbf{this is exactly what the original form has proved}. 
		Here we give a direct proof for it. Since $P\left( X \right) \le \infty$, we
		have 
		\[
				P\left( E \cup \bigcup_{n\ge k} T^{n}\left( E \right)  \right)  > 
				P\left( \bigcup_{n\ge k} T^{n}\left( E \right)  \right) 
		.\] 
		Applying $T$ to the set above $k$ times, there is
		\begin{align*}
				P\left( T^{k}\left( E \right) \cup 
				\bigcup_{n\ge 2k} T^{n}\left( E \right) \right) &= 
				P \left( T^{k}\left( E \cup \bigcup_{n\ge k}T^{n}\left( E \right)   \right)  \right) \\ 
			&=  P\left(
		E \cup \bigcup_{n\ge k} T^{n}\left( E \right) \right)  \\
& > P\left(
			\bigcup_{n\ge k} T^{n}\left( E \right)  \right) 
		.\end{align*}
		This leads to a contradiction since 
		\[
				T^{k}\left( E \right) \cup \bigcup_{n\ge 2k} T^{n}(E) 
				\subset  \bigcup_{n\ge k} T^{n}(E)
		.\] 
		Which completes our proof.
\end{proof}

\noindent \textbf{Remark} : what this proof tells us is that the construction 
of $A_{n}$ in the original proof \textbf{does not come from nowhere}.
It is exactly something derived from the objective equation, which reveals 
that the original proof is somewhat natural.

\end{document}
