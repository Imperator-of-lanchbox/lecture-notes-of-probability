\documentclass[UTF8, 12pt]{ctexart}
\usepackage{enumitem}
\usepackage{amsmath}
\usepackage{amssymb}
\usepackage{amsfonts}
\usepackage{mathrsfs}
\usepackage{XCharter}
\usepackage{fancyhdr}
\setCJKmainfont{DENGL.TTF}
\usepackage{eulervm}
\usepackage{graphicx}
\usepackage{mdframed}
\usepackage{ntheorem}

\topmargin -.5in
\textheight 9in
\oddsidemargin -.25in
\evensidemargin -.25in
\textwidth 7in
\pagestyle{fancy}

\newenvironment{proof}{\noindent\ignorespaces\textbf{Proof:}}{\hfill $\square$\par\noindent}
\newenvironment{solution}{\noindent\ignorespaces\textbf{Solution:}}{\hfill $\square$\par\noindent}

\newtheorem{claim}{Claim}
\newtheorem*{lemma*}{Lemma}
\newtheorem*{theorem*}{Theorem}
\newtheorem*{problem*}{Problem}

\newenvironment{rcases}{\left.\begin{aligned}}{\end{aligned}\right\rbrace}

\title{Product of A Divergent Series \\ \large An application of Borel-Cantelli Theorem}
\author{张志成 518030910439}
\date{\today}

\begin{document}
    \maketitle
    \begin{problem*}
        Let $(y_n)_{n\in\mathbb{N}}$ be a sequence of reals from $[0,1]$ such that $\sum_{n\in\mathbb{N}}y_n = \infty$.
        Show that $\prod_{n\in\mathbb{N}}(1-y_n) = 0$.
    \end{problem*}

    \begin{proof}
        Recall \textbf{the second Borel-Cantelli Lemma(BC2)} we learned in class: 
        \begin{lemma*}
            If the events $E_n$ are pairwisely independent, then $$ \sum_n P(E_n) = \infty \Longrightarrow P(lim sup E_n) = 1 $$
        \end{lemma*} \par
        We observe that this problem is very similar to this Lemma, in particular $y_n$ is analogous to $P(E_n)$. Suppose we can find
        for each $y_n$ an event $E_n$ such that $y_n = P(E_n)$, then $\sum_n P(E_n) = \infty$.
        
        Then by the BC2 lemma, 
        \begin{align*}
            &\quad\quad\quad P(\text{lim}\ \text{sup} E_n) = P(\bigcap_{m\in\mathbb{N}}\bigcup_{n\geq m} E_n) = 1 \\
            &\Longrightarrow P(\bigcup_{n\geq 1} E_n) = 1 \\
            &\Longrightarrow P((\bigcup_{n\geq 1} E_n)^c) = P(\bigcap_{n\geq 1} E_n^c) = 0
        \end{align*}

        Now we can prove the statement in the problem,
        \begin{align*}
            \prod_{n\in\mathbb{N}}(1-y_n) &= \prod_{n\in\mathbb{N}}P(E_n^c)\\
            &= P(\bigcap_{n\geq 1} E_n^c) \\
            &= 0
        \end{align*}

        The only thing left is an explicit expression for $E_n$ such that $P(E_n) = y_n$.
        Consider a probaility space $(\Omega, \mathcal{F}, P)$.
        Define $X_n: \Omega \rightarrow [0,1]$ to be a \textbf{equally distributed} random variable, namely
        $$ \forall \omega \in \Omega\ \forall i\in[0,1]\ \forall j\in[0,1],\ P(X_n(\omega) = i) = P(X_n(\omega) = j) $$

        Then we can simply let $$ E_n = \{\omega\ |\ X_n(\omega) \leq y_n \} $$

        Thus, we have shown that $\prod_{n\in\mathbb{N}}(1-y_n) = 0$.
    \end{proof}
\end{document}