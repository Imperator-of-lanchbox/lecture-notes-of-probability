\documentclass[12pt]{article}

%\usepackage[UTF8]{ctex}
\usepackage{geometry}
\usepackage{amsthm}
\usepackage{amsmath}
\usepackage{amssymb}
\usepackage{mathtools}
\usepackage{enumerate}
\usepackage{hyperref} 
\usepackage{tcolorbox}

\geometry{a4paper, left = 2cm, right = 2cm, top = 2cm}

\newcommand\problem[1]{\section*{Problem #1}}

\newcommand\bE{\mathbb{E}}
\newcommand\bF{\mathbb{F}}
\newcommand\bN{\mathbb{N}}
\newcommand\bZ{\mathbb{Z}}
\newcommand\bQ{\mathbb{Q}}
\newcommand\bR{\mathbb{R}}
\newcommand\fC{\mathbf{C}}
\newcommand\fF{\mathbf{F}}
\newcommand\fN{\mathbf{N}}
\newcommand\fQ{\mathbf{Q}}
\newcommand\fR{\mathbf{R}}
\newcommand\fZ{\mathbf{Z}}
\newcommand\cF{\mathcal{F}}
\newcommand\cU{\mathcal{U}}

\newcommand\pro{\mathbf{P}}
\newcommand\ce{\coloneqq}
\newcommand\lproof{\item ``$\Leftarrow$'' :}
\newcommand\rproof{\item ``$\Rightarrow$'' :}

\newcommand{\leb}{\text{Leb}}
\newcommand*{\dif}{\mathop{}\!\mathrm{d}}
\newcommand{\ord}{\text{ord}}
\newcommand{\floor}[1]{\lfloor {#1}\rfloor}
\newcommand{\ind}[1]{\mathbf{1}_{#1}}
\newcommand{\ms}{\rm{m}\Sigma}

\newtheorem{claim}{Claim}
\newtheorem{definition}{Definition}
\newtheorem{lemma}{Lemma}
\newtheorem{theorem}{Theorem}
\newtheorem{corollary}{Corollary}
\newtheorem{remark}{Remark}

\title{A proof of a $\mu$-integrable lemma and $h(f\mu)=(hf)\mu$ using the standard machine}
\author{WU Runzhe\\
	Student ID : 518030910432}
\date{}

\setlength{\parskip}{1em}

\begin{document}
	\maketitle \large
	
	\begin{tcolorbox}
		\begin{lemma}\label{ll}
			If $f\in (\ms)^+$ and $h\in (\ms)$, then $h\in \mathcal{L}^1(S,\Sigma,f\mu)$ iff $fh\in \mathcal{L}^1(S,\Sigma,\mu)$ and then $(f\mu)(h)=\mu(fh)$.
		\end{lemma}
	\end{tcolorbox}

	To that end, let's consider the following lemma.
	
	\begin{lemma}\label{l1}
		If $f\in (\ms)^+$ and $h\in (\ms)^+$, then $(f\mu)(h)=\mu(fh)$.
	\end{lemma}

	\begin{proof}[Proof of lemma \ref{l1}]
		
		Let's prove it using the standard machine.
		
		First, we should show it holds when $h$ is an indicator function. Suppose $h=\ind{A}$ with $A\in\Sigma$, and then, by definition, we have $(f\mu)(\ind{A})=(f\mu)(A)=\mu(f\cdot \ind{A})$.
		
		Then, consider the situation where $h\in \rm{SF^+}$. Assume $h=\sum_{k=1}^na_k\ind{A_k}$. By the linearity, the left hand side equals $(f\mu)(\sum_{k=1}^na_k\ind{A_k})=\sum_{k=1}^n a_k(f\mu)(\ind{A_k})=\sum_{k=1}^n a_k\cdot\mu(f\cdot \ind{A_k})=\mu(f\cdot \sum_{k=1}^n a_k\ind{A_k})=\mu(fh)$.
		
		When $h\in \ms^+$, we can easily construct a sequence of non-negative simple functions $(h_n)$ such that $h_n\uparrow h$, and obviously $fh_n\uparrow fh$ as well. As we already have $(f\mu)(h_n)=\mu(fh_n)$ for each $n\in\bN$, by MON and letting $n\rightarrow \infty$ on both sides, we obtain $(f\mu)(h)=\mu(fh)$.
		
	\end{proof}

	Now we can prove the first part of the lemma \ref{ll}, that is,                                      
	
	$$h\in \mathcal{L}^1(S,\Sigma,f\mu) \text{ iff } fh\in \mathcal{L}^1(S,\Sigma,\mu).$$
	
	By definition, $h\in \mathcal{L}^1(S,\Sigma,f\mu)$ if $(f\mu)(|h|)<\infty$. Since $|h|\in \ms^+$, we have $(f\mu)(|h|)=\mu(f|h|)=\mu(|fh|)<\infty$, which means $fh\in \mathcal{L}^1(S,\Sigma,\mu)$, and it holds vice versa.
	
	When $h\in \mathcal{L}^1(S,\Sigma,f\mu)$, by linearity,  $(f\mu)(h)=(f\mu)(h^+-h^-)=(f\mu)(h^+)-(f\mu)(h^-)=\mu(fh^+)-\mu(fh^-)=\mu(fh)$.
	
	\begin{remark}
		This lemma implies something about ``division'' and integration by substitution because it shows that 
		$$\int_S h \dif{(f\mu)}=(f\mu)(h)=\mu(fh)=\int_Shf\dif\mu$$
		which to some degree means $\dif{(f\mu)}=f\dif\mu$, and furthermore, $f=\frac{\dif{(f\mu)}}{\dif\mu}$. That looks like a division and the method of substitution on integration.
		Let $\lambda$ denote $f\mu$, then we have $f=\frac{\dif{\lambda}}{\mu}$ here, and we say that $\lambda$ has \textbf{density} $f$ relative to $\mu$.
	\end{remark}

	\begin{remark}
		A trivial but common corollary goes here. Let $F\in\Sigma$, then $\mu(F)=0$ implies $\lambda(F)=0$ because $\lambda(F)=(f\mu)(F)=\mu(f\cdot \ind{F})=0$. That means $\mu$-null sets have nothing to do with the result of integration.
	\end{remark}

	
\end{document}
