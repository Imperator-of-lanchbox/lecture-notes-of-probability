%! Author = jinho
%! Date = 4/23/2020
\documentclass[a4paper, linespread=1.5]{article}
%\usepackage[UTF8]{ctex}
\usepackage{xeCJK}
\usepackage{geometry}
\usepackage{amsmath}
\usepackage{amssymb}
\usepackage{amsthm}
\usepackage{graphicx}
\usepackage{keyval}
\usepackage[dvipsnames,svgnames,x11names]{xcolor}
\usepackage{float}
\usepackage{ifthen}
\usepackage{calc}
\usepackage{ifplatform}
\usepackage{fancyvrb}
\usepackage{minted}
\usepackage{hyperref}
\usepackage{enumerate}
\usepackage{multicol}
\usepackage[all]{xy}
\usepackage{ulem}
\usepackage{epstopdf}
\usepackage{mathrsfs}
\usepackage{cancel}
\usepackage{algorithm}
\usepackage{algorithmic}
\usepackage{amsfonts}
\setlength{\parskip}{0.2\baselineskip}
\setlength{\parindent}{2em}
%\geometry{left=2.7cm,right=2.7cm,top=2.7cm,bottom=2.7cm}


\newtheorem{theorem}{Theorem}
\newtheorem{proposition}[theorem]{Proposition}
\newtheorem{lemma}[theorem]{Lemma}
\newtheorem{corollary}[theorem]{Corollary}
\newtheorem{definition}[theorem]{Definition}
\newtheorem{exercise}[theorem]{Exercise}

\newtheorem{innercustom}{\customname}
\providecommand{\customname}{}
\newcommand{\newcustomtheorem}[2]{
    \newenvironment{#1}[1]
    {
        \renewcommand\customname{#2}
        \renewcommand\theinnercustom{##1}
        \innercustom
    }
    {\endinnercustom}
}
\newcustomtheorem{customthm}{Theorem}
\newcustomtheorem{customprop}{Proposition}
\newcustomtheorem{customlemma}{Lemma}
\newcustomtheorem{customcorollary}{Corollary}
\newcustomtheorem{customdef}{Definition}
\newcustomtheorem{customex}{Exercise}
\newcustomtheorem{customremark}{Remark}

\newcommand{\Natural}{\mathbb{N}}
\newcommand{\addbigcup}{\bigcup{\kern-1.12em{+}}\kern0.3em}
\newcommand{\nth}[1]{#1\textsuperscript{th}}

% Document
\begin{document}
    \title{ Proof of Radon-Nikodym theorem}
    \author{金弘义\ 518030910333}
    \date{\today}
    \maketitle
    \begin{customthm}{}
        If $\mu$ and $\lambda$ are $\sigma$-finite measures on $(S,\Sigma)$ such that for all $F\in\Sigma$ with $\mu(F)=0$, $\lambda(F)=0$, then $\lambda=f\mu$ for some $f\in(m\Sigma)^{+}$.
    \end{customthm}
    \begin{proof}
        Let's first consider a simplified version of Radon–Nikodym theorem:
        Let $\mu$ and $\lambda$ be finite measures on $(S,\Sigma)$ such that for all $F\in\Sigma$ with $\mu(F)=0$, $\lambda(F)=0$, then there exists a $\mu$-nullset $D$ and $f\in(m\Sigma)^{+}$ such that $\lambda(A)=\lambda(A\cap D)+f\mu(A)$ for all $A\in\Sigma$.

        Let $H=\{h\in(m\Sigma)^{+}:h\mu(A)\le\lambda(A)\text{for all }A\in\Sigma \}$.
        Then $H\ne \emptyset$, because it contains at least the zero function.
        Now suppose $h_1,h_2\in H$.
        Let $A_1=\{x\in A:h_1(x)\le h_2(x)\}$, $A_2=\{x\in A:h_1(x)> h_2(x)\}$.
        Then
        \begin{align*}
            \max \{h_1,h_2\}\mu(A)&=\int_{S}\max \{h_1,h_2\}1_{A}d\mu\\
                                    &=\int_{S}\max \{h_1,h_2\}1_{A_1}d\mu+\int_{S}\max \{h_1,h_2\}1_{A_2}d\mu \\
                                    &=\int_{S}h_{2}1_{A_1}d\mu+\int_{S}h_{1}1_{A_2}d\mu \\
                                    &=h_{2}\mu(A_{1})+h_{1}\mu(A_{2}) \\
                                    &<\lambda(A_{1})+\lambda(A_{2}) \\
                                    &=\lambda(A)
        \end{align*}
        So $\max{h_1,h_2}\in H$.

        Now let $h_n$ be a sequence such that $\lim\limits_{n\to\infty}h_n\mu(A)=\sup\limits_{h\in H} h\mu(A)$.
        Since $H$ is closed by taking the maximum, we can simply replace $h_n$ by $\max\limits_{i=1}^{n}h_i$, and this makes $h_n$ an increasing sequnece.
        Assume $h_n\uparrow f$, then we have $h_n\mu(A)\uparrow f\mu(A)$ by MON.
        So $f\mu(A)=\sup\limits_{h\in H} h\mu(A)\le \lambda(A)$.

        Let $\tau(A)=\lambda(A)-f\mu(A)$.
        It's obviously a non-negative function.
        Define $\epsilon_n=\{E\in\Sigma:\mu(E)>n\tau(E)\}$.

        \textbf{Claim 1.} There exists a countable disjoiont subfamily $F_n$ of $\epsilon_n$ such that $S\setminus \bigcup\limits_{F\in F_n}F$ does not contain any member of $\epsilon_n$.

        \textbf{Proof.} By Zorn's lemma, we know there's a maximal element in disjoint subfamilies of $\epsilon_n$.
        Let $F_n$ be such a subfamily.
        If $S\setminus\bigcup\limits_{F\in F_n}F_n$ contains $\epsilon\in\epsilon_n$, then we can enlarge $F_n$ to be $F_n \cup \epsilon$, because $\epsilon$ is disjoint with all members in $F_n$.
        This violates that $F_n$ is the maximal element.
        So $S\setminus \bigcup\limits_{F\in F_n}F$ does not contain any member of $\epsilon_n$.

        Since $\mu(S)<\infty$ and $\bigcup\limits_{F\in F_n}F \subseteq S$, we can know that $\forall k\in\mathbb{N} ,B_k:=\{F\in F_n:\mu(F)>\frac{1}{k}\}$ has finite elements.
        $F_n=\{F\in F_n:\mu(F)>0\}=\bigcup_{k=1}^{\infty}B_k$, which means that $F_n$ is countable.

        \textbf{Claim 2.} Let $D=\bigcup\limits_{n=1}^{\infty}(S\setminus E_n)$, where $E_n=\bigcup\limits_{F\in F_n}F$.
        Then  $\tau(S\setminus D)=0$.

        \textbf{Proof.}Using the $\sigma$-additivity, we know that $E_n\in\epsilon_n$.
        \begin{align*}
            \tau(S\setminus D)&\le \tau(E_n)   &(S\setminus D\subseteq E_n)\\
                              &<\frac{\mu(E_n)}{n} &(E_n\in\epsilon_n)\\
                              &\le \frac{\mu(S)}{n}\to 0&(\mu(S)<\infty)
        \end{align*}

        \textbf{Claim 3.} $\mu(D)=0$

        \textbf{Proof.} It suffices to show that $\mu(S\setminus E_n)=0$.
        Let $B=S\setminus E_n$.
        If $\mu(B)>0$, then $(f+\frac{1_B}{n})\mu(S)=f\mu(S)+\frac{\mu(B)}{n}>f\mu(S)=\sup\limits_{h\in H} h\mu(S)$.
        So $f+\frac{1_B}{n}\notin H$.
        This means that $\exists A\in\Sigma,(f+\frac{1_B}{n})\mu(A)=f\mu(A)+\frac{\mu(A\cap B)}{n}>\lambda\mu(S) $.
        Now we have $\mu(A\cap B) > n(\lambda(A)-f\mu(A))=n\tau(A)\ge n\tau(A\cap B)$.
        So $A\cap B\in\epsilon_n$, which violates claim 1.
        This finishes the proof of claim 3.

        Finally, we gather the 3 claims together,
        \begin{align*}
            \tau(A)&=\tau(A\cap D)+\tau(A\setminus D) \\
                   &=\tau(A\cap D)&(\text{by claim 2})\\
                   &=\lambda(A\cap D)-f\mu(A\cap D)\\
                   &=\lambda(A\cap D) &(f 1_{A\cap D}=0 \text{ a.e. by claim 3})
        \end{align*}
        Now we apply this simplified version: Since $\mu(D)=0$, then $\lambda(D)=0$, $\lambda(A\cap D)=0$.
        So $\lambda(A)=\lambda(A\cap D)+f\mu(A)=f\mu(A)$.

        Since now we have proved that if $\mu$ and $\lambda$ are finite measures on $(S,\Sigma)$ such that for all $F\in\Sigma$ with $\mu(F)=0$, $\lambda(F)=0$, then $\lambda=f\mu$ for some $f\in(m\Sigma)^{+}$.
        Then we extend it to $\sigma$-finite measures.

        Suppose $\mu,\lambda$ is $\sigma$-finite.
        Then there exists a sequence $(A_n)\in\Sigma$ such that $\mu(A_n)<\infty$ and $\bigcup A_n=S$ and a sequence $(B_n)\in\Sigma$ such that $\lambda(B_n)<\infty$ and $\bigcup B_n=S$.
        \begin{align*}
            A_i=A_i \cap S &=A_i \cap \bigcup\limits_{j} B_j =\bigcup\limits_{j}A_i \cap B_j\\
            S&=\bigcup\limits_{i} A_i=\bigcup\limits_{i,j}A_i\cap B_j
        \end{align*}
        Let $S_{i,j}=A_i \cap B_j$.
        Then $S_{i,j}$ is a collection of disjoint sets whose union is $S$ and have finite measures under both $\mu$ and $\lambda$.
        For each $S_{i,j}$, there is a $f_{i,j}\in(m\Sigma)^{+}$ such that $\lambda(A)=f_{i,j}\mu(A)$ for all $A\subseteq S_{i,j}$.
        Let $f(x)=f_{i,j}(x)$ if $x\in S_{i,j}$ .
        \begin{align*}
            \lambda(A)&=\sum\limits_{i,j}\lambda(A\cap S_{i,j}) \\
                    &=\sum\limits_{i,j}f_{i,j}\mu(A\cap S_{i,j}) \\
                    &=\sum\limits_{i,j}f\mu(A\cap S_{i,j}) \\
                    &=f\mu(A)
        \end{align*}
        This finishes the proof.

    \end{proof}
\end{document}