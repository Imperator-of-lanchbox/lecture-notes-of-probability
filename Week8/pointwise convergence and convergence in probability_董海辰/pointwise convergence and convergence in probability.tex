\input{/Users/oscar/Documents/LaTeX_Templates/HW.tex}

\title{Pointwise convergence and convergence in probability}
\date{\today}
\author{董海辰 518030910417}

\begin{document}
\maketitle

\begin{thm}{}{}
    In measure space $(S, \Sigma, \mu)$, consider function sequence $(f_n) \subseteq m\Sigma$ and $f \in m\Sigma$ :
    \begin{itemize}
        \item $(f_n)$ almost surely(pointwise) converges to $f_n$ if:
            $$\mu(f_n \not\to f) = 0.$$
            In probability space $(\Omega, \mathcal F, P)$, we can also say that $P(f_n \to f) = 1$.
        \item $(f_n)$ converges to $f_n$ in probability if:
            $$\forall \varepsilon >0, \mu(|f_n -f| > \varepsilon ) \to  0.$$
    \end{itemize}
\end{thm}

To make the equations more readable, in the following discussion, let
$$A_{n, \varepsilon } = (|f_n - f| > \varepsilon).$$

\begin{thm}{}{}
    If $f_n \to f$ almost surely, then $f_n \to f$ in probability.
\end{thm}

\begin{proof}[Proof]
    $f_n \to f$ almost surely, thus 
    \begin{math}
        \mu (f_n \not \to f) 
        &= \mu (\{x: \exists \varepsilon > 0, \forall m \in \mathbb{N} _+, \exists n \ge m, |f_n(x) - f(x)| > \varepsilon \} ) \\
        &= \mu (\{x : \exists k \in \mathbb{N} _+, \forall m \in \mathbb{N} _+, \exists n \ge m, |f_n(x ) - f(x )| > \frac{1}{k} \} ) \\
        &= \mu (\bigcup_{k = 1}^\infty \bigcap_{m} \bigcup_{n=m} ^\infty A_{n,\frac{1}{k}} ) = 0
    .\end{math}

    Then $\forall k > 0, \mu (\bigcap_{m}\bigcup_{n=m} A_{n,\frac{1}{k}}) = 0$ and therefore
    $$\forall \varepsilon >0, \mu(\bigcap_{m} \bigcup_{n=m} A_{n,\varepsilon }) = 0.$$

    Then by Reversed Fatou Lemma,
    \begin{math}
        \forall \varepsilon >0, \lim_{n \to \infty} \mu (A_{n,\varepsilon }) \le \bigcap_{m} \bigcup_{n=m} \mu (A_{n,\varepsilon }) \le \mu (\bigcap_{m}\bigcup_{n=m} A_{n,\varepsilon }) = 0  
    .\end{math}

    Therefore, $f_n \to f$ in probability.
\end{proof}

\begin{thm}{}{}
    If $f_n \to f$ in probability, it is not necessary that $f_n \to f$ almost surely.
\end{thm}
\begin{proof}[Counter-example]
    Let $(S, \mathcal F, \mu ) = ([0, 1), \mathcal B([0, 1), \text{Leb})$, $n = \sum _{j=1}^i j + d(1 \le d \le i)$,
    \begin{math}
        f_n(x) = \begin{cases}
            1, x \in [\frac{d-1}{i}, \frac{d}{i})\\
            0, \text{ otherwise }
        \end{cases}
    .\end{math}

    Then, $\forall \varepsilon >0, \mu (A_{n, \varepsilon }) \le \frac{1}{i} \to 0$. 

    But $\forall x \in [0, 1)$, $f_n(x)$ does not converge, thus $\mu (f_n \not\to f) = 1 \neq 0$.
\end{proof}

\begin{thm}{}{}
    If $f_n \to f$ in probability, then there is a subsequence $(f_{n_i})$ such that $f_{n_i} \to f$ almost surely.
\end{thm}

\begin{proof}[Proof 1]
    $f_n \to f$ in probability, thus
    $$\forall k, \exists n_k, \mu(A_{n_k, \frac{1}{k}}) < \frac{1}{k^2} .$$

    In this case, $\sum _k \mu(A_{n_k, \frac{1}{k}})$ converges. By BC1, we know that
    $$\mu(\bigcap_{l}\bigcup_{i=l} A_{n_i, \frac{1}{i}}) = 0.$$

    Assume that $x \in (f_n \not\to f)$, i.e., $\forall \varepsilon >0, x \in A_{n_i, \varepsilon }$ infinitely often. 

    Let $n_{i_0} = 0$, and at every step we let $\varepsilon _j = \frac{1}{j}$, and choose an $n_{i_j}$ such that $n_{i_j} > \max(j, n_{i_{j-1}})$ and $x \in A_{n_{i_j}, \varepsilon _j} \subseteq A_{n_{i_j}, \frac{1}{i_j}}$, i.e., $x \in A_{n_i, \frac{1}{i}}$ infinitely often.

    Thus $(f_{n_i} \not\to f) \subseteq \bigcap_{l}\bigcup_{i=l} A_{n_i, \frac{1}{i}} $.And we have $\mu(\bigcap_{l} \bigcup_{i=l} A_{n_i, \frac{1}{i}}) = 0$, so $\mu(f_{n_i} \not \to f) = 0$, $f_{n_i} \to f$ almost surely.

\end{proof}

\begin{proof}[Proof 2]

    To show that there is a subsequence $(f_{n_k})$ such that $f_{n_k} \to f$ almost surely, i.e.,
    \begin{math}
        \forall \varepsilon >0, \mu (\bigcap_{l} \bigcup_{i=l} A_{n_i, \varepsilon }) = 0
    .\end{math}

    An idea is to get this via BC1, which needs us to prove that 
    $$\forall \varepsilon >0, \sum _{i=1}^\infty \mu (A_{n_i, \frac{1}{k}}) < \infty.$$

    And we know that $f_n \to f$ in probability. Let $n_0 = 0$, $\forall k$, there exists $n_k > n_{k-1}$, such that $\mu (A_{n_i,\frac{1}{i}}) < \frac{1}{i^2}$.  Therefore:
    \begin{math}
        \forall \varepsilon >0, \sum _{i=1}^\infty \mu (A_{n_i, \frac{1}{k}}) 
        &= (\sum _{i=1}^{\left\lceil \frac{1}{\varepsilon } \right\rceil } + \sum _{i=\left\lceil \frac{1}{\varepsilon } \right\rceil  +1 }^\infty) \mu (A_{n_i, \varepsilon }) \\
        &< \sum _{i=1}^{\left\lceil \frac{1}{\varepsilon } \right\rceil } \mu(A_{n_i, \varepsilon }) + \sum _{i=\left\lceil \frac{1}{\varepsilon } \right\rceil  +1 }^\infty \mu (A_{n_i, \frac{1}{i}}) \\
        &< \sum _{i=1}^{\left\lceil \frac{1}{\varepsilon } \right\rceil } \mu(A_{n_i, \varepsilon }) + \sum _{i=\left\lceil \frac{1}{\varepsilon } \right\rceil  +1 }^\infty \frac{1}{i^2} < \infty
    .\end{math}

    Therefore, $f_{n_k} \to f$ almost surely.
\end{proof}
\end{document}
