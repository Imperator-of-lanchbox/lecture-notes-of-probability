\input{/Users/oscar/Documents/LaTeX_Templates/HW.tex}

\title{Integrals over subsets}
\date{\today}
\author{董海辰 518030910417}

\begin{document}
\maketitle

\begin{thm}{}{}
    In measure space $(S, \Sigma, \mu)$, $A \in \Sigma$. Show that:
    \begin{enumerate}
        \item $(A, \Sigma\cap 2^A)$ is a $\sigma $-algebra.
        \item $\forall f \in (m\Sigma)^+$, 
            $$\mu_{\Sigma\cap 2^A}(f) = \mu(f\cdot 1_A).$$
    \end{enumerate}
\end{thm}

\begin{proof}[Proof]
    ~
    \begin{enumerate}
        \item We have:
            \begin{itemize}
                \item $A \in \Sigma \implies A \in \Sigma\cap 2^A$.
                \item $\forall F \in \Sigma \cap 2^A$, let $H = F \cap 2^A \in \Sigma$, then $\Sigma \setminus F = \Sigma\setminus(F \cap 2^A) = (\Sigma \cap 2^A)\setminus F \in \Sigma \implies \Sigma\cap 2^A \in \Sigma\cap 2^A$.
                \item $\forall (F_i)_{i\in \mathbb{N} }$ where $F_i \in \Sigma \cap 2^A \subseteq \Sigma$. Then $\bigcup_{i} F_i \in \Sigma$, and $\bigcup_{i} F_i \in 2^A$, thus, $\bigcup_{i} F_i \in \Sigma \cap 2^A$.
            \end{itemize}

            Therefore, $(A, \Sigma\cup 2^A)$ is a $\sigma $-algebra.
        \item We will show this with the standard machine.
            \begin{itemize}
                \item Let $f = 1_B, B \in \Sigma \cap 2^A$. We have $\mu(f \cdot 1_A) = \mu(1_B \cdot 1_A) = \mu(A\cap B)$. Meanwhile, 
                    $$\mu|_{\Sigma \cap 2^A}(1_B) = \mu|_{\Sigma \cap 2^A}((B\setminus A) \cup (B\cap A)) = \mu|_{\Sigma \cap 2^A}(B\cap A) + 0 = \mu(A \cap B).$$

                    Thus $\mu(f\cdot 1_A) = \mu|_{\Sigma \cap 2^A}(f)$ for all $f = 1_B$.

                \item Then we can let $f \in SF^+$. Let $f = \sum ^n b_i 1_{B_i}$. Then 
                    \begin{math}
                        \mu(f\cdot 1_A) = \mu((\sum ^n b_i 1_{B_i})1_A) &= \mu(\sum ^n b_i 1_{B_i \cap A}) \\
                                                                        &= \sum ^n b_i\mu(1_{B_i \cap A}) \\
                                                                        &= \sum ^n b_i \mu|_{\Sigma \cap 2^A}(1_{B_i}) =  \mu|_{\Sigma \cap 2^A}(\sum ^n b_i 1_{B_i})
                    .\end{math}
                \item Finally we let $f \in (m\Sigma)^+$, $\mu(f) = \sup \{\mu(h): h \in SF^+, h\le f\} $.

                    \begin{math}
                        \mu(f \cdot 1_A) &= \sup \{\mu(h): h \in SF^+, h \le f\cdot 1_A\} \\
                                         &= \sup \{\mu(g\cdot 1_A): g \in SF^+, g \cdot 1_A \le f \cdot 1_A\}  \\
                                         &= \sup \{\mu|_{\Sigma \cap 2^A}(g): g \in SF^+, g \cdot 1_A \le f \cdot 1_A\}  \\
                                         &= \sup \{\mu|_{\Sigma \cap 2^A}(g): g \in SF^+, g\le f\}  = \mu|_{\Sigma \cap 2^A}(f)
                    .\end{math}
            \end{itemize}

            Therefore, for all $f \in (m\Sigma)^+$, the result of integral over a subset is equal to the result of the integral in restricted measure space.
    \end{enumerate}
\end{proof}


\end{document}
