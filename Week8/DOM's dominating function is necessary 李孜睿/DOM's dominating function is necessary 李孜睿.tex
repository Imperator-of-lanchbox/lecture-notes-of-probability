% This is a template for lecture notes.
\documentclass{article}
\usepackage[UTF8]{ctex}
\usepackage{amssymb}
\usepackage{amsmath}
\usepackage{amsthm}
\usepackage{geometry}
\usepackage{booktabs}
\usepackage{bm}
\usepackage{tcolorbox}
\CTEXoptions[today=old]
%Some commonly used notations
%\geometry{a4paper,bottom = 3cm,left = 3cm, right = 3cm}

%for reference
\usepackage{hyperref}
\usepackage[capitalise]{cleveref}
\crefname{enumi}{}{}

\newtheorem{theorem}{Theorem}
\newtheorem{lemma}[theorem]{Lemma}
\newtheorem{proposition}[theorem]{Proposition}
\newtheorem{corollary}[theorem]{Corollary}
\newtheorem{fact}[theorem]{Fact}
\newtheorem{definition}[theorem]{Definition}
\newtheorem{remark}[theorem]{Remark}
\newtheorem{question}[theorem]{Question}
\newtheorem{answer}[theorem]{Answer}
\newtheorem{exercise}[theorem]{Exercise}
\newtheorem{example}[theorem]{Example}
%\newenvironment{proof}{\noindent \textbf{Proof:}}{$\Box$}
\newtheorem{observation}[theorem]{Observation}

%to use newcommand for convenience
\newcommand\field{\mathbb{F}}
\newcommand\Real{\mathbb{R}}
\newcommand\Q{\mathbb{Q}}
\newcommand\Z{\mathbb{Z}}
\newcommand\complex{\mathbb{C}}
\newenvironment{myproof}{\ignorespaces\paragraph{Proof:}}{\hfill $\square$\par\noindent}
%this is how we define operators.
\DeclareMathOperator{\rank}{rank} % rank

\title{DOM's dominating function is necessary}
\author{李孜睿 518030910424}
\date{\today}


\begin{document}
	\maketitle
	\begin{tcolorbox}
	\begin{theorem}
	\text{DOM(Lebesgue's Dominated Convergence Theorem)}
	\end{theorem}
	
	$f_n \rightarrow f(\text{a.e.})$  and $|f_n(s)|\leq g(s),\forall s\in S,\forall n\in\mathbb{N}$, $\mu(g)<\infty$ $\Rightarrow$ $\mu(|f_n-f|)\rightarrow 0$ which also implies $\mu(f_n)\rightarrow\mu(f)$.
	\end{tcolorbox}
	
	
	
	I will prove the necessity of the dominating function $g\in\mathcal{L}^1(S,\Sigma,\mu)$ by giving a counterexample.
	
	
	
	Assume $S=(0, 1]$,  $\Sigma=\mathcal{B}(S)$, and $\mu=\text{Leb}(S)$. Define $f=0$ and
	
	$$
	f_n(s)=\left\{
	\begin{array}{lr}
	n, s\in(0,\frac1n]\\
	0, s\in(\frac1n,1]
	\end{array}
	\right.
	$$
	
	Clearly, $f_n\rightarrow f(\text{a.e.})$. But $\mu(|f_n-f|)=n\times\mu((0,\frac1n])=1\Rightarrow\mu(|f_n-f|)\rightarrow 1$, contradicting to DOM's conclusion $\mu(|f_n-f|)\rightarrow 0$.	
\end{document}