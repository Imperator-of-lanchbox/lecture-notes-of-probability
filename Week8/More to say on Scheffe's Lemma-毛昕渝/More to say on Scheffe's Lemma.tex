\documentclass[12pt]{article}
\usepackage{amssymb}
%\usepackage[UTF8]{ctex}
\usepackage{amsmath}
\usepackage{amsthm}
\usepackage{geometry}
\usepackage{booktabs}
\usepackage{enumerate}
\usepackage{bm}
\usepackage{cite}
%\usepackage{CJK}
\usepackage[many]{tcolorbox}
%\tcbuselibrary{listingsutf8}
%\tcbuselibrary{skins, breakable, theorems, most}
%\geometry{a4paper,bottom = 3cm,left = 3cm, right = 3cm}

%for reference
\usepackage{hyperref}
\usepackage[capitalise]{cleveref}
\crefname{enumi}{}{}

\theoremstyle{plain}
\newtheorem{theorem}{Theorem}
\newtheorem{lemma}[theorem]{Lemma}
\newtheorem{proposition}[theorem]{Proposition}
\newtheorem{question}[theorem]{Question}

\theoremstyle{definition}
\newtheorem{definition}[theorem]{Definition}
\newtheorem*{remark}{Remark}
%\newenvironment{proof}{\noindent \textbf{Proof:}}{$\Box$}

%to use newcommand for convenience
\newcommand\field{\mathbb{F}}
\newcommand\R{\mathbb{R}}
\newcommand\Q{\mathbb{Q}}
\newcommand\Z{\mathbb{Z}}
\newcommand\N{\mathbb{N}}
\newcommand\cc{\mathcal{C}}
\newcommand\bb{\mathcal{B}}
\newcommand\pp{\mathcal{P}}
\newcommand\PP{\mathbb{P}}
\newcommand\nn{\mathcal{N}}
\newcommand\LL{\mathcal{L}^1(S,\Sigma,\mu)}
\newcommand\one{\bm{1}}
\newcommand\eps{\varepsilon}
\newcommand\pr[1]{\mathcal{P} \left( #1\right)}
\DeclareMathOperator{\leb}{Leb}   
\DeclareMathOperator{\diff}{d}   

\title{More to say on Scheffe's Lemma}
\author{Xinyu Mao}
\date{\today}
\begin{document}
\maketitle
Scheffe's Lemma is proved in our textbook(see section 5.10):
\begin{lemma}[Scheffe] \label{scheffe}
    Suppose that $f_n, f \in \LL$ and $f_n \to f$ (a.e.), then
    $$
        \mu(|f_n|) \to \mu(|f|) \iff \mu(|f_n - f|) \to 0.
    $$
\end{lemma}

Actually, we can get a more accurate result:
\begin{theorem} \label{main}
    Suppose that $f_n, f \in \LL^+$ and $f_n \to f$ (a.e.), then
    $$
        \mu(f_n) - \mu(f) - \mu(|f_n - f|) \to 0, \text{ as $n \to \infty $}.
    $$
\end{theorem}
\begin{proof}
    Let $g_n := \min(f_n,f), h_n := \max(f_n,f)$. 
    Clearly, $g_n,h_n \in \LL^+$.
    Since $|f - f_n| = h_n - g_n, f_n = h_n + g_n - f$, we have
    \begin{equation} \label{eq1}
        \begin{aligned}
            \mu(f_n) - \mu(f) - \mu(|f_n - f|) 
            &=\mu(h_n + g_n - f) - \mu(f)  - \mu(h_n - g_n) \\
            &= 2[\mu(g_n) - \mu(f)].
        \end{aligned}
    \end{equation}
    Note that $g_n \leq f, g_n \to f$,  
    and thus $\mu(g_n) \to \mu(f)$ by DOM, that is, $\mu(g_n) - \mu(f) \to 0$.
    On plugging this into \cref{eq1} we get what we set out to prove.
\end{proof}

Of course we also have the second part:
\begin{theorem}
    Suppose that $f_n, f \in \LL$ and $f_n \to f$ (a.e.), then
    $$
        \mu(|f_n|) - \mu(|f|) - \mu(|f_n - f|) \to 0, \text{ as $n \to \infty $}.
    $$
\end{theorem}
\begin{proof}
    Applying \cref{main} to $|f_n|,|f|$ yields
    \begin{equation}\label{lowerbound}
        \mu(|f_n|) - \mu(|f|) - \mu(||f_n| - |f||) \to 0.
    \end{equation}
    Note that $f_n^+ \to f^+, f_n^- \to f^-$, and by \cref{main}
    \begin{equation} \label{part}
        \mu(f_n^\pm) - \mu(f^\pm) - \mu(|f_n^\pm - f^\pm|) \to 0.
    \end{equation}
    Rewrite $|f|, |f_n|$ as $f^+ + f^-$ and $f_n^+ + f_n^-$, and by \cref{part} we have 
    \begin{equation} \label{upperbound}
        \mu(|f_n|) - \mu(|f|) - [\mu(|f_n^+ - f^+) + \mu(|f_n^- - f^-|)] \to 0.
    \end{equation}
    Since $||f_n| - |f|| \leq |f_n - f| \leq |f_n^+ - f^+| + |f_n^- - f^-|$, 
    the theorem follows from \cref{lowerbound} and \cref{upperbound}.
\end{proof}

\begin{remark}
    \cref{scheffe} immediately follows from the theorem above.
    Informally, the theorem says some mass is missing when taking limit 
    and the loss (i.e. difference between $\mu(\lim_{n \to \infty}f_n)$ 
    and $\lim_{n \to \infty} \mu(f_n)$) can be measured by $\lim_{n \to \infty} \mu(|f - f_n|)$.
    I learned about \cref{main} while glancing over \cite{tao2011introduction}(see Exercise 1.4.48).
\end{remark}

\bibliographystyle{unsrt}
\bibliography{ref.bib}
\end{document}