\documentclass[UTF8, a4paper, linespread=1.5]{article}

\usepackage{tcolorbox, listings}
\usepackage{geometry, amsmath, enumerate, indentfirst}
\usepackage{color, bm, extarrows, ulem}
\usepackage{amsthm}
\usepackage{amssymb}
\usepackage{nameref, hyperref}
\usepackage{algorithm}
\usepackage{algorithmic}
 % \geometry{top=3cm, bottom=3cm, left=1.5cm, right=1.5cm}

\usepackage{enumitem}
\setenumerate[1]{itemsep=0pt,partopsep=0pt,parsep=\parskip,topsep=5pt}
\setitemize[1]{itemsep=0pt,partopsep=0pt,parsep=\parskip,topsep=5pt}

% \usepackage{adjustbox}

\renewcommand\contentsname{Contents}

\tcbuselibrary{skins, breakable, theorems}

% \setlength{\leftskip}{10pt}
\setlength{\parindent}{10pt}
% \setlength{\parskip}{2em}
\renewcommand{\baselinestretch}{1.3}

\newcounter{RomanNumber}
\newcommand{\mrm}[1]{(\setcounter{RomanNumber}{#1}\Roman{RomanNumber})}

\renewcommand{\proofname}{\indent \textbf {proof}}

\newtheorem{theorem}{Theorem}
\newtheorem{lemma}[theorem]{Lemma}
\newtheorem{proposition}[theorem]{Proposition}
\newtheorem{corollary}[theorem]{Corollary}
\newtheorem{fact}[theorem]{Fact}
\newtheorem{definition}[theorem]{Definition}
\newtheorem{remark}[theorem]{Remark}
\newtheorem{question}[theorem]{Question}
\newtheorem{answer}[theorem]{Answer}
\newtheorem{exercise}[theorem]{Exercise}
\newtheorem{example}[theorem]{Example}
%\newenvironment{proof}{\noindent \textbf{Proof:}}{$\Box$}
\newtheorem{observation}[theorem]{Observation}

\newtcbtheorem{thm}{}
  {enhanced, theorem name and number, code={\edef\@currentlabelname{#2}}, 
  frame code={
        % \path[thick, draw] (frame.north west) -| (frame.north east) -| (frame.south east) -| (frame.south west) -| (frame.north west);
        \path[thick, draw] (frame.north west)  +(.5\baselineskip,0) -| +(0,-.5\baselineskip);
        % \path[thick, draw] (frame.north east) +(-.5\baselineskip,0) -| +(0,-.5\baselineskip);
        % \path[thick, draw] (frame.south west) +(.5\baselineskip,0) -| +(0,.5\baselineskip);
        \path[thick, draw] (frame.south east) +(-.5\baselineskip,0) -| +(0,.5\baselineskip);
    },
    left=1mm, right=1mm, top=1mm, bottom=1mm,
    colback=black!5,
    colframe=red!75!black,
    colbacktitle=black!0,
    coltitle=black!100,
    fonttitle=\bfseries}{thm}


\usepackage{xparse}
\NewDocumentEnvironment{qte}{m}{\begin{tcolorbox}[breakable, leftrule=2mm, rightrule=-0.1mm, toprule=-0.1mm, bottomrule=-0.1mm, arc=0mm, colframe=black!30!white, colback=white, coltext=white!50!black]}{\\\rightline{#1}\end{tcolorbox}}

\usepackage{environ}
\RenewEnviron{math}{%
\begin{align*}
\BODY
\end{align*}
}

\title{Incomplete subspace of L\^2}
\date{\today}
\author{庄永昊}

\begin{document}
\maketitle
    Here we give an example of an incomplete subspace of $L^2$: 

    First we construct the subspace of $L^2$ that: 
    $$l_2=\{(x_n):\sum_0^\infty x_n^2=0\}$$
    Here the function $f$ corresponding to sequence $(x_n)$ is that: 
    \begin{equation*}
        f(x)=\left\{
            \begin{aligned}
                &x_n&n\leq x<n+1\\
                &0&x<0\\
            \end{aligned}
            \right.
    \end{equation*}
    Then all such functions are in $L_2$ since they are continuous almost everywhere. 
    And it is obviously a vector space(closed in addition and scalar multiplication). 

    Now construct a Cauchy sequence $(S_N)$ in the subset that: for each element $S_N$, let $x_0=-1$, $x_{N,n}=\frac{1}{N}$ for any $N+1\leq n\leq 2N$, zero otherwise. 

    For any two element in the sequence $S_N,S_M$ there is: 
    $$\|S_N-S_M\|_2^2\leq \frac{1}{N}+\frac{1}{M}\leq \frac{2}{\min{N,M}}$$
    So it is obvious that this sequence is a Cauchy sequence. 

    However, there is no $\lim_{n\to\infty}S_N$ in $l_2$, since it is not a zero-sum sequence. 

    The main idea of this example is that, the most trivial and intuitive elements in Hilbert space are those like a sequence. 
    Giving the sequence a constraint(in this example is the sum of all elements) can easily form a vector subspace. 
\end{document}
