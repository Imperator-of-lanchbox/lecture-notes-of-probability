\documentclass[UTF8]{ctexart}
\usepackage{amsmath}
\usepackage{amssymb}
\usepackage{amsthm}
\usepackage{graphicx}
\usepackage{bm}
\usepackage{CJK}
\usepackage{float}
\usepackage{mdframed}

\usepackage{indentfirst}
\setlength{\parindent}{2em}

\providecommand{\abs}[1]{\lvert#1\rvert}
\providecommand{\norm}[1]{\lVert#1\rVert}
\providecommand{\ud}[1]{\underline{#1}}

\newmdtheoremenv{thm}{Theorem}
\newmdtheoremenv{lemma}[thm]{Lemma}
\newmdtheoremenv{fact}[thm]{Fact}
\newmdtheoremenv{cor}[thm]{Corollary}
\newtheorem{eg}{Example}
\newtheorem{ex}{Exercise}
\newmdtheoremenv{defi}{Definition}
\newenvironment{sol}
  {\par\vspace{3mm}\noindent{\it Solution}.}
  {\qed \\ \medskip}

\newcommand{\ov}{\overline}
\newcommand{\ca}{{\cal A}}
\newcommand{\cb}{{\cal B}}
\newcommand{\cc}{{\cal C}}
\newcommand{\cd}{{\cal D}}
\newcommand{\ce}{{\cal E}}
\newcommand{\cf}{{\cal F}}
\newcommand{\ch}{{\cal H}}
\newcommand{\cl}{{\cal L}}
\newcommand{\cm}{{\cal M}}
\newcommand{\cp}{{\cal P}}
\newcommand{\cs}{{\cal S}}
\newcommand{\cz}{{\cal Z}}
\newcommand{\eps}{\varepsilon}
\newcommand{\ra}{\rightarrow}
\newcommand{\la}{\leftarrow}
\newcommand{\Ra}{\Rightarrow}
\newcommand{\dist}{\mbox{\rm dist}}
\newcommand{\bn}{{\mathbb N}}
\newcommand{\bz}{{\mathbb Z}}

\newcommand{\expe}{{\mathsf E}}
\newcommand{\pr}{{\mathsf{Pr}}}


\setlength{\parindent}{0pt}
%\setlength{\parskip}{2ex}
\newenvironment{proofof}[1]{\bigskip\noindent{\itshape #1. }}{\hfill$\Box$\medskip}
\usepackage{amsthm,amsmath,amssymb}

\theoremstyle{definition}
\newtheorem{problem}{Problem}
\newtheorem*{problem*}{Problem}

\pagenumbering{gobble}

\begin{document}

\title{A Trivial Idea of Exercise 4.1}
\date{Apr. 28, 2020}

\maketitle
\paragraph{}By imitating the method we used in the proof of Lemma 4.2, we can easily get the proof of this exercise.
\paragraph{}By fixing $I_2\in \mathcal{I}_2$ and $I_3\in \mathcal{I}_3$, the two measures on $\sigma(\mathcal{I}_1)$ agree on $\mathcal{I}_1$, and they have the same total mass:
\begin{align*}
	&\mathbb{P}(I_2\cap I_3) \\&= \mathbb{P}(\Omega\cap I_2\cap I_3)\\ &=\mathbb{P}(\Omega)\mathbb{P}(I_2)\mathbb{P}(I_3)\\&=\mathbb{P}(I_2)\mathbb{P}(I_3)
\end{align*}
\paragraph{}Hence, they agree on $\sigma(\mathcal{I}_1)$
\paragraph{}By fixing $H_1 \in \sigma(\mathcal{I}_1)$ and $I_3 \in \mathcal{I}_3$, the two measures on $\sigma(\mathcal{I}_2)$ agree on $\mathcal{I}_2$, and they have the same total mass:
\begin{align*}
	&\mathbb{P}(H_1\cap I_3) \\&=\mathbb{P}(H_1)\mathbb{P}(I_3)
\end{align*}
\paragraph{}Similarly, by fixing $H_1 \in \sigma(\mathcal{I}_1)$ and $H_2 \in \sigma(\mathcal{I}_2)$, the two measures agree on $\sigma(\mathcal{I}_3)$.
\paragraph{}Then we conclude that $\sigma(\mathcal{I}_1)$ $\sigma(\mathcal{I}_1)$ $\sigma(\mathcal{I}_3)$ are independent.
\\
\paragraph{}We need $\Omega \in \mathcal{I}_k$ because we need the equation for the total mass in each case. Otherwise, consider one $\pi$-system being a set of measure 0.
\end{document}

