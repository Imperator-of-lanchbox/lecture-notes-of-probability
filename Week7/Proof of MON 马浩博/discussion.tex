% This is a template for lecture notes.
\documentclass[12pt]{article}
\usepackage{amssymb}
\usepackage[UTF8]{ctex}
\usepackage{amsmath}
\usepackage{amsthm}
\usepackage{geometry}
\usepackage{booktabs}
\usepackage{bm}
\usepackage{cite}
%\usepackage{CJK}
\usepackage[many]{tcolorbox}
%\tcbuselibrary{listingsutf8}
%\tcbuselibrary{skins, breakable, theorems, most}
%\geometry{a4paper,bottom = 3cm,left = 3cm, right = 3cm}
\CTEXoptions[today=old]
%for reference
\usepackage{hyperref}
\usepackage[capitalise]{cleveref}
\crefname{enumi}{}{}


\newtheoremstyle{mythm}{1.5ex plus 1ex minus .2ex}{1.5ex plus 1ex minus .2ex} 
    {}{\parindent}{\bfseries}{}{1em}{} 
\theoremstyle{mythm}
\newtheorem{theorem}{Theorem}
\newtheorem{lemma}[theorem]{Lemma}
\newtheorem{corollary}[theorem]{Corollary}
\newtheorem{fact}[theorem]{Fact}
\newtheorem{definition}[theorem]{Definition}
\newtheorem*{remark}{Remark}

%\newenvironment{proof}{\noindent \textbf{Proof:}}{$\Box$}

%to use newcommand for convenience
\newcommand\field{\mathbb{F}}
\newcommand\Real{\mathbb{R}}
\newcommand\Q{\mathbb{Q}}
\newcommand\Z{\mathbb{Z}}
\newcommand\complex{\mathbb{C}}
\newcommand\cc{\mathcal{C}}
\newcommand\uu{\mathcal{U}}
\newcommand\pp{\mathcal{P}}
\newcommand\ff{\mathcal{F}}
\renewcommand\refname{Reference}
\renewcommand{\proofname}{Proof}
\DeclareMathOperator{\range}{range}   

\title{Proof of MON}
\author{马浩博 518030910428}
\date{\today}
\begin{document}
\maketitle

  \begin{lemma}
	Suppose $A$ is a measurable set and $f_k(k\in N)$ is a nondecreasing sequence of non-negative measurable functions on $S$ such that
	
	$$\lim _{k}f_{k}(x)\geq 1$$
	for almost all $x \in A.$ Then
	
	$$\lim _{k}\int f_{k}\,d\mu \geq \mu (A)$$
  \end{lemma}
\begin{proof}
	Let's fix $\varepsilon > 0$ and define the sequence of measurable sets
	
	$$B_{k}=\{x\in A:f_{k}(x)\geq 1-\varepsilon \}$$
	
	By monotonicity of the integral, it follows that for any $k \in N$
	
	$$(1-\varepsilon )\mu (B_{k})=\int (1-\varepsilon )1_{B_{k}}\,d\mu \leq \int f_{k}\,d\mu$$
	
	Because almost every $x$ is in $B_k$ for large enough $k$, we have
	
	$$(\bigcup _{k}B_k) \cup C=A$$
	with a set $C$ of measure 0. Thus by countable additivity of $\mu$, and because $B_k$ increases with $k$, we have
	
	$$\mu (A)=\lim _{k}\mu (B_{k})\leq \lim _{k}(1-\varepsilon )^{-1}\int f_{k}\,d\mu$$
	
	As this is true for any positive $\varepsilon$, then the result follows.
	
\end{proof}

  \begin{theorem}[Lebesgue monotone-convergence theorem]
  If $(f_n)$ is a sequence of elements of $(m\Sigma)^+$ such that $f_n \uparrow f$,
  then
  $$\mu(f_n)=\mu(f) \leq \infty$$
  \end{theorem}
    
\begin{proof}

By the monotonicity property of the integral, it is immediate that:

$$\displaystyle \int f\,d\mu \geq \lim _{k}\int f_{k}\,d\mu$$
and the limit on the right exists, because the sequence is monotonic. We now prove the inequality in the other direction. That is,
$$\int fd\mu \leq \lim _{k}\int f_{k}d\mu$$

It follows from the definition of integral that there is a non-decreasing sequence $(g_n)$ of non-negative simple functions such that $g_n \leq f$ and

$$\displaystyle \lim _{n}\int g_{n}\,d\mu =\int f\,d\mu .$$

Therefore, we just need to prove that for all $n \in N$,

$$\displaystyle \int g_{n}\,d\mu \leq \lim _{k}\int f_{k}\,d\mu$$

We know that

$$\lim _{k}f_{k}(x)\geq g_n(x)$$
almost everywhere, then we can break up the function $g_n$ into its constant value parts, this reduces to the case in which $g_n$ is the indicator function of a set. So we can use Lemma 1 and get that

$$\lim _{k}\int f_{k}\,d\mu \geq \int g_n\,d\mu$$

This result is for all $n \in N$, so we finish the proof.

\end{proof} 

\section*{Reference}
	[1] Probability with martingales(2014)

\end{document}