% This is a template for lecture notes.
\documentclass{article}
\usepackage[UTF8]{ctex}
\usepackage{amssymb}
\usepackage{amsmath}
\usepackage{amsthm}
\usepackage{geometry}
\usepackage{booktabs}
\usepackage{bm}
\usepackage{tcolorbox}
\CTEXoptions[today=old]
%Some commonly used notations
%\geometry{a4paper,bottom = 3cm,left = 3cm, right = 3cm}

%for reference
\usepackage{hyperref}
\usepackage[capitalise]{cleveref}
\crefname{enumi}{}{}

\newtheorem{theorem}{Theorem}
\newtheorem{lemma}[theorem]{Lemma}
\newtheorem{proposition}[theorem]{Proposition}
\newtheorem{corollary}[theorem]{Corollary}
\newtheorem{fact}[theorem]{Fact}
\newtheorem{definition}[theorem]{Definition}
\newtheorem{remark}[theorem]{Remark}
\newtheorem{question}[theorem]{Question}
\newtheorem{answer}[theorem]{Answer}
\newtheorem{exercise}[theorem]{Exercise}
\newtheorem{example}[theorem]{Example}
\newtheorem{observation}[theorem]{Observation}

%to use newcommand for convenience
\newcommand\field{\mathbb{F}}
\newcommand\Real{\mathbb{R}}
\newcommand\Q{\mathbb{Q}}
\newcommand\Z{\mathbb{Z}}
\newcommand\complex{\mathbb{C}}
\renewcommand{\proofname}{Proof}
\renewcommand\refname{Reference}

%this is how we define operators.
\DeclareMathOperator{\rank}{rank} % rank

\title{Notes on Lebesgue Dominated Convergence Theorem}
\author{Tong Chen}
\date{\today}

\begin{document}
    \maketitle
\section{Statement of the theorem}
\begin{theorem}\label{thm:DCT}
Lebesgue's Dominated Convergence Theorem. Let ($f_n$) be a sequence of complex-valued measurable functions on a measure space $S, \Sigma, \mu$. Suppose that the sequence converges pointwise to a function f and is dominated by some integrable function g in the sense that

$${\displaystyle |f_{n}(x)|\leq g(x)}    |f_n(x)| \le g(x)$$
for all numbers n in the index set of the sequence and all points x ∈ S. Then f is integrable and

$${\displaystyle \lim _{n\to \infty }\int _{S}|f_{n}-f|\,d\mu =0} \lim_{n\to\infty} \int_S |f_n-f|\,d\mu = 0$$
which also implies

$${\displaystyle \lim _{n\to \infty }\int _{S}f_{n}\,d\mu =\int _{S}f\,d\mu }\lim_{n\to\infty} \int_S f_n\,d\mu = \int_S f\,d\mu$$
\end{theorem}
\section{Proof of the theorem}
\begin{proof}
Without loss of generality, one can assume that $f$ is real, because one can split f into its real and imaginary parts (remember that a sequence of complex numbers converges if and only if both its real and imaginary counterparts converge) and apply the triangle inequality at the end.

Lebesgue's dominated convergence theorem is a special case of the Fatou–Lebesgue theorem. Below, however, is a direct proof that uses Fatou’s lemma as the essential tool.

Since f is the pointwise limit of the sequence ($f_n$) of measurable functions that are dominated by $g$, it is also measurable and dominated by g, hence it is integrable. Furthermore, (these will be needed later),

$${\displaystyle |f-f_{n}|\leq |f|+|f_{n}|\leq 2g}    |f-f_n| \le |f| + |f_n| \leq 2g$$
for all n and

$${\displaystyle \limsup _{n\to \infty }|f-f_{n}|=0.}    \limsup_{n\to\infty} |f-f_n| = 0.$$
The second of these is trivially true (by the very definition of $f$). Using linearity and monotonicity of the Lebesgue integral,

$${\displaystyle \left|\int _{S}{f\,d\mu }-\int _{S}{f_{n}\,d\mu }\right|=\left|\int _{S}{(f-f_{n})\,d\mu }\right|\leq \int _{S}{|f-f_{n}|\,d\mu }.}    \left | \int_S{f\,d\mu} - \int_S{f_n\,d\mu} \right|=   \left| \int_S{(f-f_n)\,d\mu} \right|\le \int_S{|f-f_n|\,d\mu}.$$
By the reverse Fatou lemma (it is here that we use the fact that |f−fn| is bounded above by an integrable function)

$${\displaystyle \limsup _{n\to \infty }\int _{S}|f-f_{n}|\,d\mu \leq \int _{S}\limsup _{n\to \infty }|f-f_{n}|\,d\mu =0,}\limsup_{n\to\infty} \int_S |f-f_n|\,d\mu \le \int_S \limsup_{n\to\infty} |f-f_n|\,d\mu = 0,$$
which implies that the limit exists and vanishes i.e.

$${\displaystyle \lim _{n\to \infty }\int _{S}|f-f_{n}|\,d\mu =0.}\lim_{n\to\infty} \int_S |f-f_n|\,d\mu= 0.$$
Finally, since

$${\displaystyle \lim _{n\to \infty }\left|\int _{S}fd\mu -\int _{S}f_{n}d\mu \right|\leq \lim _{n\to \infty }\int _{S}|f-f_{n}|\,d\mu =0.}{\displaystyle \lim _{n\to \infty }\left|\int _{S}fd\mu -\int _{S}f_{n}d\mu \right|\leq \lim _{n\to \infty }\int _{S}|f-f_{n}|\,d\mu =0.}$$
we have that

$${\displaystyle \lim _{n\to \infty }\int _{S}f_{n}\,d\mu =\int _{S}f\,d\mu .}{\displaystyle \lim _{n\to \infty }\int _{S}f_{n}\,d\mu =\int _{S}f\,d\mu .}$$
The theorem now follows.

If the assumptions hold only $\mu$-almost everywhere, then there exists a $\mu$-null set $N \in \Sigma$ such that the functions $f_n \mathbf{1}_{S \backslash N}$ satisfy the assumptions everywhere on $S$. Then the function $f(x)$ defined as the pointwise limit of $f_n(x)$ for $x \in S \backslash N$ and by $f(x) = 0$ for $x \in N$, is measurable and is the pointwise limit of this modified function sequence. The values of these integrals are not influenced by these changes to the integrands on this $\mu$-null set $N$, so the theorem continues to hold.

DCT holds even if $f_n$ converges to $f$ in measure (finite measure) and the dominating function is non-negative almost everywhere.
\end{proof}

\section{Reference}
[1] Wikipedia-Dominated\_convergence\_theorem : \url{https://en.wikipedia.org/wiki/Dominated_convergence_theorem}
\end{document}