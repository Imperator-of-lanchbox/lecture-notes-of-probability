% This is a template for lecture notes.
\documentclass{article}
%\usepackage[UTF8]{ctex}
\usepackage{amssymb}
\usepackage{amsmath}
\usepackage{amsthm}
\usepackage{geometry}
\usepackage{booktabs}
\usepackage{bm}
\usepackage{tcolorbox}
\usepackage{indentfirst}
%\CTEXoptions[today=old]

%Some commonly used notations
%\geometry{a4paper,bottom = 3cm,left = 3cm, right = 3cm}

%for reference
\usepackage{hyperref}
\usepackage[capitalise]{cleveref}
\crefname{enumi}{}{}

\newtheorem{theorem}{Theorem}
\newtheorem{lemma}[theorem]{Lemma}
\newtheorem{proposition}[theorem]{Proposition}
\newtheorem{corollary}[theorem]{Corollary}
\newtheorem{fact}[theorem]{Fact}
\newtheorem{definition}[theorem]{Definition}
\newtheorem{remark}[theorem]{Remark}
\newtheorem{question}[theorem]{Question}
\newtheorem{answer}[theorem]{Answer}
\newtheorem{exercise}[theorem]{Exercise}
\newtheorem{example}[theorem]{Example}
%\newenvironment{proof}{\noindent \textbf{Proof:}}{$\Box$}
\newtheorem{observation}[theorem]{Observation}

%to use newcommand for convenience
\newcommand\field{\mathbb{F}}
\newcommand\Real{\mathbb{R}}
\newcommand\Q{\mathbb{Q}}
\newcommand\Z{\mathbb{Z}}
\newcommand\complex{\mathbb{C}}

%this is how we define operators.
\DeclareMathOperator{\rank}{rank} % rank

\title{Borel set on [a,b] is generated by $\pi$[a,b]}
\author{Ji Jiabao}
\date{\today}

\begin{document}
\maketitle

\begin{proof}
    \hspace*{1em} \newline
    \hspace*{1em} $\Leftarrow:$ prove $\sigma(\pi[a,b]) \subseteq \mathcal{B}[a,b]$\\
    \hspace*{1em} $$\pi[a,b] = \{ A \cap [a,b]: A \in \pi(\mathbb{R}) \} = \{ (-\infty , x] \cap [a,b] : x \in \mathbb{R} \} = \{ [a, x]: x \in [a,b] \}$$
    \hspace*{1em} Simlilar to the proof in class, we need to show that any $[a,x]$ is a union of countable open sets.
    Observing that $[a, x] = \cap_{n \in \mathbb{N}} = (a - \frac{1}{n}, x + \frac{1}{n})$, $[a, x] \in \mathcal{B}[a,b]$, which leads to $\sigma(\pi [a,b]) \subseteq \mathcal{B}[a,b]$\\


    $\Rightarrow:$ prove $\mathcal{B}([a,b]) \subseteq \sigma(\pi[a,b])$ \\
    \hspace*{1em} Similar to the proof in class, we need to show that any open set of $[a,b]$ is contained in 
    $\sigma(\pi[a,b])$. Still similar to the proof, each of these open sets is a countably union of open intervals,
    we need to show that every $s = (x, y) \cap [a,b] \in \sigma(\pi([a,b]))$.\\
    \hspace*{1em} We can see such $s$ is in the four classes below, we prove $s \in \sigma(\pi([a,b]))$ for each case .
    \begin{enumerate}   
        \item $s = [a, y)$, $s = \cap_{n > \frac{1}{b - y}, n \in \mathbb{N}}[a, y + \frac{1}{n}]$ and $[a, y + \frac{1}{n}] \in \pi([a,b])$, so $s \in \sigma(\pi([a,b]))$
        \item $s = (x, b]$, $s = \cap_{n > \frac{1}{x - a}, n \in \mathbb{N}}[x - \frac{1}{n}, b]$ and $[x - \frac{1}{n}, b] \in \pi([a,b])$, so $s \in \sigma(\pi([a,b]))$
        \item $s = [a, b]$, $[a, b] \in \pi([a,b])$ so $s \in \sigma(\pi([a,b]))$
        \item $s = (x, y)$, $(x, y) = (x, b] \cap [a, y) \in \sigma(\pi([a,b]))$
    \end{enumerate}
    \hspace*{1em}Therefore, $\mathcal{B}([a,b]) \subseteq \sigma(\pi[a,b])$
\end{proof}

\end{document}