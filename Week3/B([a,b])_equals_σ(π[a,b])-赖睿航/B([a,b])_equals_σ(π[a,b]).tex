\documentclass[a4paper, linespread=1.5]{article}
%\usepackage[UTF8]{ctex}
\usepackage{xeCJK}
\usepackage{geometry}
\usepackage{amsmath}
\usepackage{amssymb}
\usepackage{amsthm}
\usepackage{graphicx}
\usepackage{keyval}
\usepackage[dvipsnames,svgnames,x11names]{xcolor}
\usepackage{float}
\usepackage{ifthen}
\usepackage{calc}
\usepackage{ifplatform}
\usepackage{fancyvrb}
\usepackage{minted}
\usepackage{hyperref}
\usepackage{enumerate}
\usepackage{multicol}
\usepackage[all]{xy}
\usepackage{ulem}
\usepackage{epstopdf}
\usepackage{mathrsfs}
\usepackage{cancel}
\usepackage{algorithm}
\usepackage{algorithmic}
\setlength{\parskip}{0.2\baselineskip}
\setlength{\parindent}{2em}
%\geometry{left=2.7cm,right=2.7cm,top=2.7cm,bottom=2.7cm}


\newtheorem{theorem}{Theorem}
\newtheorem{proposition}[theorem]{Proposition}
\newtheorem{lemma}[theorem]{Lemma}
\newtheorem{corollary}[theorem]{Corollary}
\newtheorem{definition}[theorem]{Definition}
\newtheorem{exercise}[theorem]{Exercise}

\newtheorem{innercustom}{\customname}
\providecommand{\customname}{}
\newcommand{\newcustomtheorem}[2]{
    \newenvironment{#1}[1]
    {
        \renewcommand\customname{#2}
        \renewcommand\theinnercustom{##1}
        \innercustom
    }
    {\endinnercustom}
}
\newcustomtheorem{customthm}{Theorem}
\newcustomtheorem{customprop}{Proposition}
\newcustomtheorem{customlemma}{Lemma}
\newcustomtheorem{customcorollary}{Corollary}
\newcustomtheorem{customdef}{Definition}
\newcustomtheorem{customex}{Exercise}
\newcustomtheorem{customremark}{Remark}

\newcommand{\Natural}{\mathbb{N}}
\newcommand{\Real}{\mathbb{R}}
\newcommand{\BorelSet}{\mathcal{B}}
\newcommand{\addbigcup}{\bigcup{\kern-1.12em{+}}\kern0.3em}
\newcommand{\nth}[1]{#1\textsuperscript{th}}

\begin{document}
    \title{Show that $\BorelSet([a, b]) = \sigma(\pi[a, b])$}
    \author{赖睿航\ 518030910422}
    \date{\today}
    \maketitle
    
    \begin{proof}
        According to the definition of $\pi(\Real)$ and $\pi[a, b]$, we have:
        \begin{align*}
            \pi[a, b] &= \{A \cap [a, b] \mid A \in \pi(\Real)\} \\
            &= \{(-\infty, x] \cap [a, b] \mid x \in \Real \} \\
            &= \{[a, x] \mid x \in [a, b]\}.
        \end{align*}
        Let $\tau$ be all the open sets on $\Real$. By the definition of subspace topology, we have
        \begin{align*}
        \BorelSet([a, b]) &= \sigma(\{\textrm{all open sets on } [a, b]\}) \\
        &= \sigma(\{[a, b] \cap U \mid U \in \tau \}).
        \end{align*}
        Hence intervals such as $[a, b], [a, x), (x, b], x \in (a, b)$ are all open intervals on $[a, b]$.
        
        We first show that $\sigma(\pi[a, b]) \subseteq \BorelSet([a, b])$. By definition of $\sigma$-algebra, we only need to show that $\pi[a, b] \subseteq \BorelSet([a, b])$. For any interval $[a, x] \in \pi[a, b]$ with $x \in [a, b]$, there are two cases:
        \begin{enumerate}
            \item If $x \in [a, b)$, then $[a, x] = \bigcap_{n \in \Natural} [a, x + \frac{b - x}{2n}) \in \BorelSet([a, b])$.
            \item If $x = b$, then $[a, x] = [a, b] \in \BorelSet([a, b])$ since $[a, b]$ is an open set.
        \end{enumerate}
        Thus $\pi[a, b] \subseteq \BorelSet([a, b])$.
        
        Then we show that $\BorelSet([a, b]) \subseteq \sigma(\pi[a, b])$. Equivalently we show that every open set on $[a, b]$ is contained in $\sigma(\pi[a, b])$. Every open set on $\sigma(\pi[a, b])$ is a countable union of open intervals. So we only need to show that every open interval on $[a, b]$ belongs to $\sigma(\pi[a, b])$. For any open interval $I$, there are four cases:
        \begin{enumerate}
            \item If $I = [a, b]$, then $I = [a, b] \in \sigma(\pi[a, b])$ is trivial.
            \item If $I = [a, y), y \in (a, b]$, then $I = [a, y) = \bigcup_{n \in \Natural} [a, y - \frac{y - a}{2n}] \in \sigma(\pi[a, b])$.
            \item If $I = (x, b], x \in [a, b)$, then $I = (x, b] = [a, b] \setminus [a, x] \in \sigma(\pi[a, b])$.
            \item If $I = (x, y), a \leqslant x < y \leqslant b$, then $I = (x, y) = [a, y) \setminus [a, x] = (\bigcup_{n \in \Natural} [a, y - \frac{y - a}{2n}]) \setminus [a, x] \in \sigma(\pi[a, b])$.
        \end{enumerate}
        Hence every open interval on $[a, b]$ belongs to $\sigma(\pi[a, b])$. So every open set on $[a, b]$ belongs to $\sigma(\pi[a, b])$. And therefore $\BorelSet([a, b]) = \sigma(\pi[a, b])$.
    \end{proof}
\end{document}
