\documentclass{article}
\usepackage[UTF8]{ctex}
\usepackage{amssymb}
\usepackage{amsmath}
\usepackage{amsthm}
\usepackage{geometry}
\usepackage{booktabs}
\usepackage{bm}
\usepackage{tcolorbox}
\usepackage{xunicode, mathrsfs, xltxtra, amsfonts, caption, latexsym}
\CTEXoptions[today=old]
%Some commonly used notations
%\geometry{a4paper,bottom = 3cm,left = 3cm, right = 3cm}

%for reference
\usepackage{hyperref}
\usepackage[capitalise]{cleveref}
\crefname{enumi}{}{}

\newtheorem{theorem}{Theorem}
\newtheorem{lemma}[theorem]{Lemma}
\newtheorem{proposition}[theorem]{Proposition}
\newtheorem{corollary}[theorem]{Corollary}
\newtheorem{fact}[theorem]{Fact}
\newtheorem{definition}[theorem]{Definition}
\newtheorem{remark}[theorem]{Remark}
\newtheorem{question}[theorem]{Question}
\newtheorem{answer}[theorem]{Answer}
\newtheorem{exercise}[theorem]{Exercise}
\newtheorem{example}[theorem]{Example}
%\newenvironment{proof}{\noindent \textbf{Proof:}}{$\Box$}
\newtheorem{observation}[theorem]{Observation}

%this is how we define operators.
\DeclareMathOperator{\rank}{rank} % rank

\newenvironment{myproof}{\ignorespaces\paragraph{Proof:}}{\hfill $\square$\par\noindent}

\title{Probability, Week 3, exerciese 2}
\author{庄永昊}
\date{\today}
\def\bab{\mathcal B[a,b]}
\def\spab{\sigma(\pi([a,b]))}
\begin{document}
\maketitle

0. By the definition of $\pi([a,b])$ and $\pi(\mathbb R)$, $\pi([a,b])$ can be written as: 
$$\pi([a,b])=\{[a,x]\mid x\in[a,b]\}$$. 

1. Now prove that $\spab\subseteq\bab$. 

As all open subsets of $[a,b]$ are in $\bab$, all closed subsets of $[a,b]$ are also in $\bab$, 
so $\forall x\in[a,b],[a,x]\in \bab$. 

So there is $\pi([a,b])\subset\bab$. As $\bab$ is also a $\sigma-$algebra, 
$\spab\subseteq\bab$. 

2. Now prove that $\bab\subseteq\spab$. 

Since $\bab$ is generated by all open subsets of $[a,b]$(also open subsets of $\mathbb R$), 
each one(said $s$), can is a countably union of open intervals on $\mathbb R$, 
denoted by $I_1,I_2,\cdots$. So there is $s=\cap_i I_i$ 

Now Let $I_i'=I_i\cap [a,b]$, it is obvious that $s=\cap_i I_i'$. Otherwise, there are some elements 
in $I_j$ but not $I_j'$ for some $j$, so it is not in $[a,b]$. But $s\subset [a,b]$. 

Now we prove that for each $i$, $I_i'\in\spab$: 

1)$I_i'$ can only be form like $[a,b],[a,x),(x,b],(x,y)$ where $x,y\in(a,b)$; 

2)$[a,b]\in\pi([a,b])$, so $[a,b]\in\spab$; 

3)as $x<b$, we have $\exists n\rightarrow\forall i>n,x+2^{-i}<b$. So 
$[a,x)=\cap_{i=n+1}^{\infty}[a,x+2^{-i}]$ and $[a,x+2^{-i}]\in\pi([a,b])$. So $[a,x)\in\spab$

4)as $[a,x],[a,b]\in\pi([a,b])$, there is $(x,b]=[a,b]/[a,x]\in\pi([a,b])\subseteq\spab$; 

5)By 3) and 4) $[a,y),(x,b]\in\spab$, there is $(x,y)=[a,y)\cap(x,b]\in\spab$. 

\end{document}