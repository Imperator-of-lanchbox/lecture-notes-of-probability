\documentclass[12pt]{article}

%\usepackage[UTF8]{ctex}
\usepackage{geometry}
\usepackage{amsthm}
\usepackage{amsmath}
\usepackage{amssymb}
\usepackage{mathtools}
\usepackage{enumerate}
\usepackage{hyperref} 
\usepackage{tcolorbox}

\geometry{a4paper, left = 2cm, right = 2cm, top = 2cm}

\newcommand\problem[1]{\section*{Problem #1}}

\newcommand\bE{\mathbb{E}}
\newcommand\bF{\mathbb{F}}
\newcommand\bN{\mathbb{N}}
\newcommand\bZ{\mathbb{Z}}
\newcommand\bQ{\mathbb{Q}}
\newcommand\bR{\mathbb{R}}
\newcommand\fC{\mathbf{C}}
\newcommand\fF{\mathbf{F}}
\newcommand\fN{\mathbf{N}}
\newcommand\fQ{\mathbf{Q}}
\newcommand\fR{\mathbf{R}}
\newcommand\fZ{\mathbf{Z}}
\newcommand\cU{\mathcal{U}}

\newcommand\ce{\coloneqq}
\newcommand\lproof{\item ``$\Leftarrow$'' :}
\newcommand\rproof{\item ``$\Rightarrow$'' :}

\newcommand{\leb}{\text{Leb}}
\newcommand*{\dif}{\mathop{}\!\mathrm{d}}
\newcommand\ord{\text{ord}}
\newcommand{\floor}[1]{\lfloor {#1}\rfloor}

\newtheorem{claim}{Claim}
\newtheorem{lemma}{Lemma}
\newtheorem{theorem}{Theorem}
\newtheorem{corollary}{Corollary}


\title{A Non-Lebesgue-Measurable Set in $\bR^n$}
\author{WU Runzhe\\
	Student ID : 518030910432\\
	\textsc{Shanghai Jiao Tong University}}
\date{\today}

\begin{document}
	\maketitle
	
	Our goal here is to construct a non-Lebesgue-measurable set in $\bR^n$ for $n\in \bZ_+$. I have to say that the desired set constructed here is quite similar to Vitali set on $\bR$.
	
	Let's consider the collection of cosets $A\ce\bR^n/\bQ^n$. And we limit the elements in $A$ on $[0,1]^n$, that is, we define $B\ce \{S\cap[0,1]^n:S\in A \} $.
	
	Using axiom of choice, we can obtain a choice function $ f : B \rightarrow [0,1]^n $, and we define $V\ce \{f(S):S \in B\}$. Undoubtedly, $V \subseteq [0,1]^n$.
	
	Furthermore, we construct a new set $W$ by translating $V$ in all directions with some limitations, namely, 
	\begin{equation}\label{w}
		W\ce \bigcup_{v\in  \bQ^n\cap[-1,1]^n} (V+v)
	\end{equation}
	
	\begin{lemma}\label{l0}
		For $u,v\in  \bQ^n\cap[-1,1]^n$ with $u\not=v$, $(V+v)\cap (V+u)=\emptyset$.
	\end{lemma}

	\begin{proof}[Proof of lemma \ref{l0}]
		Assume not, say, $z\in (V+v)\cap (V+u)$. Then  for some $x,y\in V$, we have
		$$x+v=y+u=z$$
		which means
		$$x-y=u-v.$$
		As $u-v\in \bQ^n$, we have $x-y\in \bQ^n$, which means $x,y\in \bQ^n+t$ for some $t\in \bR^n$. Therefore, by the definition of $V$, it is impossible for both $x$ and $y$ to be in $V$. It contracts our assumption.
	\end{proof}
	
	\begin{lemma}\label{l1}
		$[0,1]^n\subseteq W$.
	\end{lemma}

	\begin{proof}[Proof of lemma \ref{l1}]
		For each $x\in [0,1]^n$, we know that $x\in \bQ^n+t$ for some $t\in \bR$ as $\bQ^n$ is a subgroup of $\bR^n$. Consider the choice on $\bQ^n+t$, say, $y=f(\bQ^n+t)\in\bQ^n+t$. It is clear that $y-x\in \bQ^n$. And since $x,y\in[0,1]^n$, $y-x\in \bQ^n\cap[-1,1]^n$, which completes the proof.
	\end{proof}
	
	\begin{lemma}\label{l2}
		$W\subseteq [-1,2]^n$.
	\end{lemma}

	\begin{proof}[Proof of lemma \ref{l2}]
		This is quite obvious as we have $x\in[0,1]^n$ for each $x\in V$, and of course, $x+v\in [-1,2]^n$ since $v\in [-1,1]^n$.
	\end{proof}

	The following corollary is a direct result by combining lemma \ref{l1} and lemma \ref{l2}.
	\begin{corollary}\label{c1}
		$[0,1]^n\subseteq W\subseteq [-1,2]^n$
	\end{corollary}
	
	We assume all elements in $\bR^n$ is Lebesgue-measurable. By the properties of measure and corollary \ref{c1}, we have
	
	\begin{equation}\label{eq1}
		\leb([0,1]^n) \le \leb(W) \le \leb([-1,2]^n)
	\end{equation}
	
	Combining Eq.\ref{w} and lemma \ref{l0}, we have 
	
	$$\leb(W)=\leb(\bigcup_{v\in  \bQ^n\cap[-1,1]^n} (V+v))=\sum_{\bQ^n\cap[-1,1]^n}\leb(V+v)$$.
	
	According to the translation invariant of Lebesgue measure
	\footnote{Lebesgue measure can also be obtained by limiting the Lebesgue outer measure on Lebesgue $\sigma-$algebra, and we can easily tell from the definition of Lebesgue outer measure that it has the property of translation invariant.},
	we know that $\leb(V+v)=\leb(V+u)$ for $u,v\in\bQ^n\cap[-1,1]^n$. 
	
	Now we pick an arbitrary $v\in\bQ^n\cap[-1,1]^n$. If $\leb(V+v)=0$, then $\leb(W)=0$, too. And by Eq.\ref{eq1}, we have $\leb([0,1]^n)=0$, which is not Lebesgue measure means to do. On the other hand, if $\leb(V+v)>0$, it implies $\leb(W)=\infty$, which means $\leb([-1,2]^n)=\infty$. This also contradicts the definition of Lebesgue measure.
	
	In conclusion, $W$ is not a Lebesgue-measurable set in $\bR^n$.
	
\end{document}
