\documentclass[12pt]{article}

%\usepackage[UTF8]{ctex}
\usepackage{geometry}
\usepackage{amsthm}
\usepackage{amsmath}
\usepackage{amssymb}
\usepackage{mathtools}
\usepackage{enumerate}
\usepackage{hyperref} 
\usepackage{tcolorbox}

\geometry{a4paper, left = 2cm, right = 2cm, top = 2cm}

\newcommand\problem[1]{\section*{Problem #1}}

\newcommand\bE{\mathbb{E}}
\newcommand\bF{\mathbb{F}}
\newcommand\bN{\mathbb{N}}
\newcommand\bZ{\mathbb{Z}}
\newcommand\bQ{\mathbb{Q}}
\newcommand\bR{\mathbb{R}}
\newcommand\fC{\mathbf{C}}
\newcommand\fF{\mathbf{F}}
\newcommand\fN{\mathbf{N}}
\newcommand\fQ{\mathbf{Q}}
\newcommand\fR{\mathbf{R}}
\newcommand\fZ{\mathbf{Z}}
\newcommand\cU{\mathcal{U}}

\newcommand\ce{\coloneqq}
\newcommand\lproof{\item ``$\Leftarrow$'' :}
\newcommand\rproof{\item ``$\Rightarrow$'' :}

\newcommand{\leb}{\text{Leb}}
\newcommand*{\dif}{\mathop{}\!\mathrm{d}}
\newcommand{\ord}{\text{ord}}
\newcommand{\floor}[1]{\lfloor {#1}\rfloor}
\newcommand{\ind}[1]{\mathbf{1}_{#1}}

\newtheorem{claim}{Claim}
\newtheorem{definition}{Definition}
\newtheorem{lemma}{Lemma}
\newtheorem{theorem}{Theorem}
\newtheorem{corollary}{Corollary}
\newtheorem{remark}{Remark}


\title{A Proof of Doob-Dynkin Lemma}
\author{WU Runzhe\\
	Student ID : 518030910432\\
	\textsc{Shanghai Jiao Tong University}}
\date{\today}

\begin{document}
	\maketitle
	
	\begin{tcolorbox}
		\textbf{Doob-Dynkin Lemma.}
			Let $X,Y:S\rightarrow \bR$ be random variables. Then $\sigma(Y)\subseteq\sigma(X)$ implies there exists a $\mathcal{B}/\mathcal{B}$-measurable function $f$ such that $f\circ X=Y$.
	\end{tcolorbox}

	It seems that there is a common technology to prove the validity of a certain property for some functions that breaks proof to four easier steps as follows
	
	\begin{enumerate}[(1)]
		\item Prove the property for indicator functions.
		\item Using linearity, extend the proof to simple functions.\footnote{A simple function is a finite linear combination of indicator functions of measurable sets.}
		\item Using monotone approximation, extend the property to non-negative functions.
		\item Extend the property to some real-valued functions $f$ by writing $f=f^+-f^-$.
	\end{enumerate}

	It's called \textit{Standard Machine}. Anyway, the following proof will follow this idea.
	
	Let's do the first step---show the validity of Doob-Dynkin Lemma for indicator functions.

	\begin{lemma}\label{indic_func}
		Doob-Dynkin Lemma holds when $Y$ is an indicator function.
	\end{lemma}

	\begin{proof}[Proof of Lemma \ref{indic_func}]
		Suppose $Y=\ind{A}$ where $A\subseteq S$. Then we have  $\sigma(Y)=\{\emptyset,S,A,A^C\}\subseteq\sigma(X)$. Clearly, $f=\ind{B}$ where $X^{-1}(B)=A$ will suffice, and trivially, $f$ is $\mathcal{B}/\mathcal{B}$-measurable.
	\end{proof}

	Then we extend it to some simple functions.

	\begin{lemma}\label{sim}
		Doob-Dynkin Lemma holds when $Y$ is a finite linear combination of disjoint indicator functions in $\bR$. Formally, 
		$$Y=\sum_{i\in I}\lambda_i\ind{A_i}$$
		where $\lambda_i\in\bR$ and $A_i\cap A_j=\emptyset$ for any $i,j\in I$ with $i\not=j$.
	\end{lemma}

	\begin{proof}[Proof of Lemma \ref{sim}]
		Without loss of generality, we assume all $\lambda_i$ are pairwise distinct. It's easy to see that $\sigma(\ind{A_i})\subseteq\sigma(Y)\subseteq \sigma(X)$, so by \textit{Lemma \ref{indic_func}}, we can obtain some measurable functions $f_i,i\in I$. And $f=\sum_{i\in I} \lambda_i f_i$ will suffice.
	\end{proof}

	\begin{remark}
		Note that \textit{Lemma \ref{sim}} is true only for \textbf{finite} combination of indicator functions. Otherwise, the measurability of $f$ won't be guaranteed.
	\end{remark}

	Let's define the dyadic function $d_n\in\bR^{\bR}$ for $n\in \bN$ as follows.
	$$d_n=\sum_{k=1}^{n2^n}\frac{k-1}{2^n}\ind{[\frac{k-1}{2^n},\frac{k}{2^n})}+n\ind{[n,+\infty)}$$
	
	\begin{lemma}[Monotone Approximation]\label{ma}
		Let $f$ be a non-negative measurable function. Then there exists a sequence $\{f_n\}_{n=1}^\infty$ of non-negative simple functions such that $f_n\uparrow f$.
	\end{lemma}

	\begin{proof}[Proof of Lemma \ref{ma}]
		Let $f_n=d_n\circ f$. The proof of why $f_n\uparrow f$ has been shown in previous work by other students (maybe in folder \textit{week 4 or 5} in our course website), so I won't repeat it here.
	\end{proof}

	By letting $f=f^+-f^-$, we obtain the following corollary.

	\begin{corollary}[Approximation]\label{app}
		Let $f$ be a measurable function. Then there exists a sequence $\{f_n\}_{n=1}^\infty$ of simple functions such that $f_n\rightarrow f$.
	\end{corollary}

	Now we can prove the Doob-Dynkin Lemma. Let $X,Y$ be random variables, and by \textit{corollary \ref{app}}, there exists a sequence of simple function $\{Y_n\}_{n=1}^\infty$ where $Y_n=d_n\circ Y^+-d_n\circ Y^-$ such that $Y_n\rightarrow Y$. Now we should prove $\sigma(Y_n)\subseteq\sigma(Y)$ in order to use \textit{Lemma \ref{sim}}. By definition, if $A\in \sigma(Y_n)$, then there exists $B\in \mathcal{B}$ such that $A=Y_n^{-1}B$. As $Y_n$ is a simple function, we can somehow extend $B$ by finite number of subset in $\bR$ to $B'$, formally, $B'=B\cup B_1\cup B_2\cup\cdots \cup B_k$ such that $A=Y^{-1}B'$. That means $\sigma(Y_n)\subseteq\sigma(Y)\subseteq\sigma(X)$. By \textit{Lemma 2}, each $Y_n$ has one associated $f_n$ to satisfy the Doob-Dynkin Lemma. Then $f=\lim\sup f_n$ is again measurable by property of measurable functions. And $f\circ X=(\lim\sup f_n)\circ X=\lim\sup(f_n\circ X)=\lim \sup Y_n=\lim Y_n=Y$.
	
	
	
\end{document}
