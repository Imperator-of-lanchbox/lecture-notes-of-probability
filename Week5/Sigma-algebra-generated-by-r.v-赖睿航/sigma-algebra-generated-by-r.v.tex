\documentclass[a4paper, linespread=1.5]{article}
%\usepackage[UTF8]{ctex}
\usepackage{xeCJK, geometry, amsmath, amssymb, amsthm}
\usepackage{graphicx, keyval}
\usepackage[dvipsnames,svgnames,x11names]{xcolor}
\usepackage{float, ifthen, calc, ifplatform, fancyvrb}
\usepackage{minted, hyperref, enumerate, multicol}
\usepackage[all]{xy}
\usepackage{ulem}
%\usepackage{epstopdf}
\usepackage{mathrsfs}
\usepackage{cancel}
\usepackage{algorithm, algorithmic}

%\setlength{\parskip}{0.2\baselineskip}
\setlength{\parindent}{2em}
%\geometry{left=2.7cm,right=2.7cm,top=2.7cm,bottom=2.7cm}


\newtheorem{theorem}{Theorem}
\newtheorem{proposition}[theorem]{Proposition}
\newtheorem{lemma}[theorem]{Lemma}
\newtheorem{corollary}[theorem]{Corollary}
\newtheorem{definition}[theorem]{Definition}
\newtheorem{exercise}[theorem]{Exercise}

\newtheorem{innercustom}{\customname}
\providecommand{\customname}{}
\newcommand{\newcustomtheorem}[2]{
    \newenvironment{#1}[1]
    {
        \renewcommand\customname{#2}
        \renewcommand\theinnercustom{##1}
        \innercustom
    }
    {\endinnercustom}
}
\newcustomtheorem{customthm}{Theorem}
\newcustomtheorem{customprop}{Proposition}
\newcustomtheorem{customlemma}{Lemma}
\newcustomtheorem{customcorollary}{Corollary}
\newcustomtheorem{customdef}{Definition}
\newcustomtheorem{customex}{Exercise}
\newcustomtheorem{customremark}{Remark}

\newcommand{\Natural}{\mathbb{N}}
\newcommand{\Real}{\mathbb{R}}
\newcommand{\BorelSet}{\mathcal{B}}
\newcommand{\addbigcup}{\bigcup{\kern-1.12em{+}}\kern0.3em}
\newcommand{\nth}[1]{#1\textsuperscript{th}}
\newcommand{\IndicatorFunc}[1]{\mathbf{1}_{#1}}

\begin{document}
    \title{$\sigma(Y) = Y^{-1}(\BorelSet)$\\ \& \\ $\sigma(Y)$ can be generated by $\pi(Y)$}
    \author{赖睿航 518030910422}
    \date{\today}
    \maketitle

    \begin{exercise}
        Show that $\sigma(Y) = Y^{-1}(\BorelSet) := (\{\omega \colon Y(\omega) \in B\} \colon B \in \BorelSet)$.
    \end{exercise}

    \begin{proof}
        Let $(S, \Sigma)$ be a measurable space. It is already known that 
        \begin{align*}
            \sigma(Y) &:= \sigma(Y^{-1}(\BorelSet)) \\
            &= \sigma(Y^{-1}(B) \colon B \in \BorelSet) \\
            &= \sigma(\{\omega \colon Y(\omega) \in B\} \colon B \in \BorelSet).
        \end{align*}
        So we only need to show that $\Sigma_0 := Y^{-1}(\BorelSet) = \{Y^{-1}(B) \colon B \in \BorelSet\}$ itself is a $\sigma$-algebra. By definition we need to prove two properties of $\Sigma_0$:
        \begin{enumerate}
            \item $S_0 \in \Sigma_0 \Rightarrow S_0^c \in \Sigma_0$, and
            \item $(S_i)_{i \in \Natural} \subseteq \Sigma_0 \Rightarrow \bigcup_{i \in \Natural}S_i \in \Sigma_0$.
        \end{enumerate}
        Since $Y$ is a random variable which is a $\Sigma$-measurable function by definition, the mapping $Y^{-1}$ satisfies that
        $$
        Y^{-1}(A^c) = (Y^{-1}(A))^c, Y^{-1}(\bigcup_\alpha A_\alpha) = \bigcup_\alpha Y^{-1}(A_\alpha)
        $$
        where $A, A_\alpha \in \BorelSet$. Thus we have
        \begin{align*}
            S_0 \in \Sigma_0 &\Rightarrow S_0 \in \{Y^{-1}(B) \colon B \in \BorelSet\} \\
            &\Rightarrow \exists B_{S_0} \in \BorelSet \textrm{ s.t. } Y^{-1}(B_{S_0}) = S_0 \\
            &\Rightarrow Y^{-1}({B_{S_0}}^c) = (Y^{-1}(B_{S_0}))^c = {S_0}^c \\
            &\Rightarrow {S_0}^c \in \Sigma_0,
        \end{align*}
        and
        \begin{align*}
            (S_i)_{i \in \Natural} \subseteq \Sigma_0 &\Rightarrow \exists (B_i)_{i \in \Natural} \textrm{ s.t. } \forall i \in \Natural, Y^{-1}(B_i) = S_i \\
            &\Rightarrow Y^{-1}(\bigcup_{i \in \Natural} B_i) = \bigcup_{i \in \Natural} Y^{-1}(B_i) = \bigcup_{i \in \Natural} S_i \\
            &\Rightarrow \bigcup_{i \in \Natural} S_i \in \Sigma_0.
        \end{align*}
        Therefore, $\Sigma_0 = \{Y^{-1}(B) \colon B \in \BorelSet\}$ is a $\sigma$-algebra, which means $\sigma(Y) = \sigma(Y^{-1}(\BorelSet)) = Y^{-1}(\BorelSet)$.
    \end{proof}

    \begin{exercise}
        $\sigma(Y)$ can be generated by the $\pi$-system
        $$
        \pi(Y) := (\{\omega \colon Y(\omega) \leqslant x\} \colon x \in \Real) = Y^{-1}(\pi(\Real)).
        $$
    \end{exercise}

    \begin{proof}
        Let $(S, \Sigma)$ be a measurable space. We know that
        \begin{align*}
            &\sigma(Y) = Y^{-1}(\BorelSet) = (\{\omega \in S \colon Y(\omega) \in B\} \colon B \in \BorelSet), \\
            &\sigma(\pi(Y)) = \sigma(Y^{-1}(\pi(\Real))) = \sigma(\{\omega \in S \colon Y(\omega) \leqslant x\} \colon x \in \Real).
        \end{align*}
        From the $\Sigma$-measurability of $Y$, we know that $Y^{-1}$ preverves all set operations, which is so powerful for us to finish our proof.
        
        Since $\pi(\Real) \subseteq \BorelSet$ and $\BorelSet = \sigma(\pi(\Real))$, it follows that $Y^{-1}(\pi(\Real)) \subseteq Y^{-1}(\BorelSet)$. And from the result of the previous exercise we know that $Y^{-1}(\BorelSet)$ is a $\sigma$-algebra. So the $\sigma$-algebra generated by $Y^{-1}(\pi(\Real))$ is a sub-$\sigma$-algebra of $Y^{-1}(\BorelSet)$, i.e., $\sigma(\pi(Y)) = \sigma(Y^{-1}(\pi(\Real))) \subseteq \sigma(Y)$.
        
        Next we show that $Y^{-1}(\BorelSet) \subseteq \sigma(Y^{-1}(\pi(\Real)))$. Equivalently we show that for $\forall B \in \BorelSet, Y^{-1}(B) = \{\omega \in S \colon Y(\omega) \in B\} \in \sigma(Y^{-1}(\pi(\Real))) = \sigma(\{\omega \in S \colon Y(\omega) \leqslant x\} \colon x \in \Real)$. In the next several steps we use the $\Sigma$-measurability of $Y$ implicitly or explicitly.
        
        Note that any Borel set can be obtained by a set of countable open sets of the usual topology on $\Real$. And every open set on $\Real$ is a countable union of open intervals. Any open interval $I = (a, b)$ can be written as $\bigcup_{n \in \Natural}(a, b - \frac{b - a}{2n}]$. Then we have
        \begin{align*}
            Y^{-1}(I) &= Y^{-1}(\bigcup_{n \in \Natural}(a, b - \frac{b - a}{2n}]) \\
            &= \bigcup_{n \in \Natural} Y^{-1}((a, b - \frac{b - a}{2n}]) \\
            &= \bigcup_{n \in \Natural} Y^{-1}((-\infty, b - \frac{b - a}{2n}] \setminus (-\infty, a]) \\
            &= \bigcup_{n \in \Natural} (Y^{-1}((-\infty, b - \frac{b - a}{2n}]) \setminus Y^{-1}((-\infty, a])) \\
            &\in \sigma(Y^{-1}(\pi(\Real))).
        \end{align*}
        
        Hence every open interval belongs to $\sigma(Y^{-1}(\pi(\Real)))$, which equivalently means $Y^{-1}(\BorelSet) \subseteq \sigma(Y^{-1}(\pi(\Real)))$ by our previous analysis.
        
        Therefore, $\sigma(Y) = \sigma(\pi(Y))$, i.e., $\sigma(Y)$ can be generated by $\pi(Y)$.
    \end{proof}
\end{document}
