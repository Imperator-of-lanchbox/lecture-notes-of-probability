\documentclass[UTF8]{article}

\usepackage[T1]{fontenc}
\usepackage{textcomp}
\usepackage{theorem}
% \usepackage[dutch]{babel}
\usepackage{amsmath, amssymb}
\usepackage{import}
\usepackage{pdfpages}
\usepackage{transparent}
\usepackage{xcolor}
\usepackage{enumerate}
\usepackage{setspace} 
%\usepackage{ebgaramond}
%\fontfamily{ebgaramond}
% ------------- coding style setting --------------%
%\usepackage{libertine}
\usepackage{listings}
\usepackage{enumitem}
\setlist{nosep}
\definecolor{codegreen}{rgb}{0,0.6,0}
\definecolor{codegray}{rgb}{0.5,0.5,0.5}
\definecolor{codepurple}{rgb}{0.58,0,0.82}
\definecolor{backcolour}{rgb}{0.95,0.95,0.92}

\lstdefinestyle{mystyle}{
    backgroundcolor=\color{backcolour},
    commentstyle=\color{codegreen},
    keywordstyle=\color{magenta},
    numberstyle=\tiny\color{codegray},
    stringstyle=\color{codepurple},
    basicstyle=\ttfamily\footnotesize,
    breakatwhitespace=false,
    breaklines=true,
    captionpos=b,
    keepspaces=true,
    numbers=left,
    numbersep=5pt,
    showspaces=false,
    showstringspaces=false,
    showtabs=false,
    tabsize=2
}

\lstset{style=mystyle}

% ----------------- geometry and fancy head -----------

\usepackage{geometry}
\geometry{left=2.5cm,right=2.5cm,top=3cm,bottom=3cm}
\usepackage[many]{tcolorbox}
\tcbuselibrary{skins, breakable, theorems}

\usepackage{fancyhdr}
\usepackage{syntonly} % dubugging
% \syntaxonly
\fancypagestyle{mainFancy}{
    \fancyhf{}
    %\renewcommand\headrulewidth{0pt}       % 页眉横线
    %\renewcommand\footrulewidth{0pt}
    
    \fancyhead[L]{Probability Theory}       % 页眉章标题
    \fancyhead[R]{Assignment}         % 页眉文章题目
    \fancyfoot[C]{\thepage}                 % 页眉编号
}
\pagestyle{mainFancy}


% --------------- environment setting ------------------

\newtheorem{thm}{Theorem}
\newtheorem{pro}{Problem}
\newtheorem{lemma}{Lemma}
\newtheorem{defi}{Definition}
\newtheorem{li}{Example}
\newenvironment{proof}{\paragraph{Proof:}}{\hfill$\square$}
\newenvironment{jie}{\paragraph{Show:}}{\hfill$\square$}

\numberwithin{pro}{section}
\numberwithin{thm}{section}
\numberwithin{defi}{section}
\numberwithin{lemma}{section}


\tcolorboxenvironment{pro}{
  enhanced,
  borderline={0.4pt}{0.4pt}{black},
  boxrule=0.4pt,
  colback=white,
  coltitle=black,
  sharp corners,
}
\tcolorboxenvironment{thm}{
  enhanced,
  borderline={0.4pt}{0.4pt}{black},
  boxrule=0.4pt,
  colback=white,
  coltitle=black,
  sharp corners,
}
\tcolorboxenvironment{lemma}{
  enhanced,
  borderline={0.4pt}{0.4pt}{black},
  boxrule=0.4pt,
  colback=white,
  coltitle=black,
  sharp corners,
}
\tcolorboxenvironment{defi}{
  enhanced,
  borderline={0.4pt}{0.4pt}{black},
  boxrule=0.4pt,
  colback=white,
  coltitle=black,
  sharp corners,
}

% ----------------- macros and command -----------------
\usepackage{stmaryrd} 
\newcommand\contra{\scalebox{1.5}{$\lightning$}}
\definecolor{correct}{HTML}{009900}
\newcommand\correct[2]{\ensuremath{\:}{\color{red}{#1}}\ensuremath{\to }{\color{correct}{#2}}\ensuremath{\:}}
\newcommand\green[1]{{\color{correct}{#1}}}

% horizontal rule
\newcommand\hr{
		    \noindent\rule[0.5ex]{\linewidth}{0.5pt}
	}
\def\mf(#1){\mathfrak{#1}} 
\def\setn(#1,#2){\left\{#1_1,#1_2,\cdots, #1_#2 \right\}  }


\let\implies\Rightarrow
\let\impliedby\Leftarrow
\let\iff\Leftrightarrow
\let\ldots\cdots


\newcommand\dif{\,\mathrm{d}}
\newcommand\e{\,\mathrm{e}}
\newcommand\R{\,\mathbb{R}}
\newcommand\Q{\,\mathbb{Q}}
\newcommand\C{\,\mathbb{C}}
\newcommand\N{\,\mathbb{N}}
\newcommand\A{\,\mathbb{A}}
\newcommand\Z{\,\mathbb{Z}}
\newcommand\ep{\,\varepsilon}
\newcommand\F{\,\varphi}
\newcommand\T{\,\mathbb{T}}
\newcommand\HH{\,\mathbb{H}}
\author{Yujie Lu \quad Haichen Dong \\ \textsc{ACM Class 18} }

\linespread{1.5} \selectfont
\title{\textsc{Depicting Independence of Random Variables by Their Distribution Functions} }
\begin{document}
\maketitle
First we give a definition for distribution for multiple random variables
\begin{defi}[Joint distribution function]
		For $n$ random variables $X_1,\ldots,X_{n}$, the joint cumulative distribution function $F_{X_1,X_2\ldots,X_{n}}$ is given by
		\[
				F_{X_1,\ldots,X_{n}} = P\left( X_1\le x_1,\ldots,
				X_{n} \le  x_{n}\right) 
		.\] 
		Interpreting the $n$ random variables as a random vector 
		$\bm{X} = \left( X_1,\ldots,X_{n} \right) ^{\top }$ yields a shorter
		notation 
		\[
				F_{\bm{X}}\left( \bm{x} \right)  = 
				P\left( X_1\le x_1,\ldots,X_{n}\le x_{n} \right) 
		.\] 
\end{defi}
And we also need the lemma for independence of sigma algebra, which is proved 
in class. Here I give a generalization form of it.
\begin{lemma}
		Suppose $\mathcal{A}_{1}, \ldots, \mathcal{A}_{n}$ are independent 
		and each $\mathcal{A}_{i}$ is a $\pi$-system, then 
		$\sigma\left( \mathcal{A}_{1} \right) , \ldots, 
		\sigma\left( \mathcal{A}_{n} \right) $ are independent.
\end{lemma}
		The essence is the same with the case of 2 sigma algebras.

With the preparation done, we can depict the independence of 
random variables by their distribution functions as follows
\begin{thm}
		Random variables $X_1,X_2,\ldots,X_{n}$ are independent, if and only if
		$\forall \bm{x}=\left( x_1,\ldots,x_{n} \right) \in \R^{n}$ 
		\[
			 P\left( X_1\le x_1,\ldots,X_{n}\le x_{n} \right) 
				= \prod_{i=1}^{n} F_{i}\left( x_{i} \right)  
		.\] 
\end{thm}
\begin{proof}
		Using the definition of joint distribution function, the 
		formula above is equivalent with 
	\[
			F_{\bm{X}}\left( \bm{x} \right) = \prod_{i=1}^{n} F_{i}\left( x_{i} \right)  
	.\] 
	Which is clean and beautiful. First we prove $\implies$, the result 
	is obvious by definition. Since 
	\[
	\forall i \le n, B_{i} :=\{X_i | X_i \le  x_i\} \subset \mathcal{B}
	.\] 
	By definition, we obtain that 
	\[
			P\left( \bigcap_{i\in [n] } B_{i}  \right) = 
			\prod_{i=1}^{n} P\left( B_{i} \right)  
	.\] 

	Now we prove $\impliedby$. Consider $\mathcal{A}_{i} $ be the set of form 
	$\{X_{i} \le  x_{i}\} $. Hence by
\[
		\{ X_{i} \le r \} \cap \{X_{i} \le  s\} = \{ X_{i} \le  \min\left( r,s \right) \} 
.\] 
	We know that $\mathcal{A}_{i}$ is a $\pi$-system plus we 
	can write it as $X^{-1}\left( \pi\left( \R \right)  \right) $, and they are 
	independent (from the definition of independence).
	Moreover by the conclusion (the second half) proved by Ruihang Lai,
we have
\[
		\sigma\left( \mathcal{A}_{i} \right)= \sigma\left(
		X_{i}^{-1}\left( \pi\left( \R \right)  \right)\right)  =X_{i}^{-1}\left( \mathcal{B} \right)  =  \sigma\left( X_{i} \right) 
.\]
	It's obvious that the independence of random variables 
	is equivalent with the independence of the sigma algebra they generated.
	So by \textbf{Lemma 1}  we know that $\sigma\left( \mathcal{A}_1 \right) , 
	\ldots, \sigma\left( \mathcal{A}_{n} \right) $ are independent, which 
	means $\sigma\left( X_1 \right) ,\ldots,\sigma\left( X_{n} \right) $ are
	independent,
	hence $X_1,\ldots,X_{n}$ are independent.
\end{proof}

\end{document}
