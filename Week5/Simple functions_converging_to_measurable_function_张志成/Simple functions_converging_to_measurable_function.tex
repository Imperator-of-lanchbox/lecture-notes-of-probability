\documentclass[UTF8, 12pt]{article}
\fontfamily{DENGL.TTF}
\usepackage{xeCJK}
\usepackage{enumitem}
\usepackage{amsmath}
\usepackage{amssymb}
\usepackage{amsfonts}
\usepackage{mathrsfs}
\usepackage{XCharter}
\usepackage{fancyhdr}
\usepackage{eulervm}
\usepackage{graphicx}
\usepackage{mdframed}
\usepackage{ntheorem}

\renewcommand{\baselinestretch}{1.2}

\topmargin -.5in
\oddsidemargin -.25in
\evensidemargin -.25in
\textwidth 7in
\pagestyle{fancy}

\newenvironment{proof}{\noindent\ignorespaces\textbf{Proof:}}{\hfill $\square$\par\noindent}
\newenvironment{solution}{\noindent\ignorespaces\textbf{Solution:}}{\hfill $\square$\par\noindent}
\newenvironment{definition}{\noindent\ignorespaces\textbf{Definition:}}{\hfill \noindent}

\newtheorem{claim}{Claim}
\newtheorem{problem}{Problem}
\newtheorem{lemma}{Lemma}
\newtheorem*{theorem*}{Theorem}
\newtheorem*{remark*}{Remark}

\title{A simple way of understanding the construction of simple functions converging to a measurable function}
\author{张志成 518030910439}
\date{\today}

\begin{document}
    \maketitle
    \begin{definition} \\
        Given function $f$, we define
        \begin{align*}
            f^+(x) &= max(f(x), 0) \\
            f^-(x) &= max(-f(x), 0)
        \end{align*}
        $f$ is a simple function with respect to $(S,\Sigma)$ provided it falls into the linear subspace
        of $\mathbb{R}^S$ spanned by $\{\textbf{1}_A\ |\ A \in \Sigma \}$. 
    \end{definition}
    \begin{theorem*}
        Let $$ d_n = \sum_{k=1}^{n2^n} \frac{k-1}{2^n}\textbf{1}_{[\frac{k-1}{2^n}, \frac{k}{2^n})} + n\textbf{1}_{[n,\infty)} \in \mathbb{R}^\mathbb{R} $$
        Then $f_n = d_n \circ f^+ - d_n \circ f^-$ is a simple function and $\{f_n\}$ converges to $f$.
    \end{theorem*}

    \begin{proof}

        For simplicity, it suffices to prove the case when $f$ is nonnegative. Because $f = f^+ - f^-$, the same proof would also work if applied to $f^+$ and $f^-$.
        
        Then
        \begin{align*} 
            f_n &= d_n \circ f \\
            & = \sum_{k=1}^{n2^n} \frac{k-1}{2^n}\textbf{1}_{f^{-1}([\frac{k-1}{2^n}, \frac{k}{2^n}))} + n\textbf{1}_{f^{-1}({[n,\infty)})} \in \mathbb{R}^\mathbb{R}
        \end{align*}

        $f_n$ is obviously a simple function because it falls into the linear space of $\mathbb{R}^\mathbb{R}$ spanned by $\{\textbf{1}_{f^{-1}([\frac{k-1}{2^n}, \frac{k}{2^n})}\ |\ k \leq n2^n\}  \cup \textbf{1}_{[n,\infty)} $. 

        What this construction does intuitively is that it partitions $[0, \infty)$ to $n2^n$ intervals of length $\frac{1}{2^n}$ and another interval of $[n,\infty)$, which together constitutes $[0, \infty)$. 

        Explicitly, 
        \[
        f_n(x) =  
            \begin{cases}
                \frac{k-1}{2^n}, \quad \ f(x) \in [\frac{k-1}{2^n}, \frac{k}{2^n}) \\
                n, \quad \  f(x) \in [n, \infty)
            \end{cases}
        \]

        This construction of $f_n$ makes sure that $\forall x\ f_n(x) \leq f(x)$.
        It also makes sure that $f_n(x)$ is monotonically increasing, 
        $$
        [\frac{k-1}{2^n}, \frac{k}{2^n}) = [\frac{2k-2}{2^{n+1}}, \frac{2k-1}{2^{n+1}}) \cup [\frac{2k-1}{2^{n+1}}, \frac{2k}{2^{n+1}})
        $$
        Thus, if $f(x) \in [\frac{k-1}{2^n}, \frac{k}{2^n})$, 
        \begin{align*}
            f_{n+1}(x) &\geq \text{min}(\frac{2k-2}{2^{n+1}}, \frac{2k-1}{2^{n+1}}) = \frac{2k-2}{2^{n+1}} = f_n(x)
        \end{align*}
        Finally, convergence to $f$ is guaranteed by $$ f_n(x) - f(x) \leq \frac{k}{2^n} - \frac{k-1}{2^n} = \frac{1}{2^n} $$
    
        Therefore, we have proved that $f_n \uparrow f$. 
    \end{proof}

    \begin{remark*}
        In fact, the essense of this construction is that for any interval $(I_{n,k} = [\frac{k-1}{2^n}, \frac{k}{2^n}))$, it satisfies:
        \begin{enumerate}
            \item  
                $
                f_n(x) = \text{inf}\ \{f(x)\ |\ f(x) \in I_{n,k}\} \leq f(x)
                $
            \item $\exists t\ I_{n+1,k} \subset I_{n,t}$
        \end{enumerate}
        These are not difficult to achieve, for example we can partition $[0,\infty)$ to intervals of length $\frac{1}{n}$, 
        $$
            f_n = \sum_{k=1}^{n^2-1} \frac{k}{n}\textbf{1}_{f^{-1}([\frac{k}{n}, \frac{k + 1}{n}))} + n\textbf{1}_{f^{-1}({[n,\infty)})}
        $$
    \end{remark*}
\end{document}