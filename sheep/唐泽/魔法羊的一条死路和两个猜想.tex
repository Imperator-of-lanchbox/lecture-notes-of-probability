% !Mode:: "TeX:UTF-8"

\documentclass{article}
\usepackage[UTF8]{ctex}
\usepackage{amssymb}
\usepackage{amsmath}
\usepackage{amsthm}

\usepackage{geometry}
\geometry{a4paper,left = 3cm, right = 3cm, bottom = 2cm}

\CTEXoptions[today=old]
%Some commonly used notations
\newcommand\Real{\mathbb{R}}
\newcommand\rational{\mathbb{Q}}
\newcommand\Rational{\mathbb{Q}^*}
\newcommand\integer{\mathbb{Z}}
\newcommand\Interger{\mathbb{Z}^*}

\title{魔法羊的一条死路和两个猜想}
\author{唐泽}
\date{\today}

\begin{document}

\maketitle

定义1:$[i,j]$为$i$只黑羊,$j$只白羊的状态

转移概率:$[i,j]$
$\begin{cases}
[i+1,j-1] \ (Possibility=\frac{ip}{i+j})\\
[i,j] \ (Possibility=\frac{i(1-p)+j(1-q)}{i+j})\\
[i-1,j+1] \ (Possibility=\frac{jq}{i+j})
\end{cases}$

定义2:$f(i,j)$为$i$只黑羊,$j$只白羊,剩下黑羊数量的最大期望

定义3:$F(i,j)$为$i$只黑羊,$j$只白羊,本次不可移除白羊,剩下黑羊数量的最大期望

于是有:
$$f(i,j)=max_{0\le k\le j}\{F(i,k)\}$$
$$F(i,j)=\frac{ip}{i+j}f(i+1,j-1)+\frac{i(1-p)+j(1-q)}{i+j}f(i,j)+\frac{jq}{i+j}f(i-1,j+1)$$

边界条件:
$$f(i,0)=i$$
$$f(0,j)=0$$

但这几个式子里有个max,无法通过Gauss消元求解,所以这条路目前到此为止了。

于是我想尝试一些感性的想法。看到有变化的两个转移概率看上去像正反馈,变化对方的优势会随着个数增多而变大。于是就有了下面两个猜想:

结论1:$\forall i\exists_1 k ((\forall j\ge k,f(i,j)=f(i,k)) \wedge (\forall j<k,f(i,j)<f(i,k)))$

简略证明:$f(i,j)$在$j$变化时最大值存在是因为:黑羊期望要增多,有个必要条件是$\frac{ip}{i+j}>\frac{jq}{i+j}$,这个条件约束了白羊数量$j$有个上限,最大值必然在这个上限之前到来。$j$足够大后函数值不变是由于取F函数的前缀max.

定义4:将结论1中$i$所对应的唯一的$k$定义为$g(i)$

猜想1(白羊单调性):$\forall i\forall j_1<j_2\le g(i), f(i,j_1)\le f(i,j_2)$ 

猜想2(决策单调性):对于固定的$p$,$q$,有 $\forall i_1<i_2, g(i_1)\le g(i_2)$

如果这两个单调性猜想正确的话就产生了一种策略:当前白羊数目超过决策点$g(i)$时,减少白羊数量至决策点;当前白羊数目不到决策点$g(i)$时,继续魔法就行。

但是还有个问题是$g(i)$的求解,拓展一点也可以说是$g(i,p,q)$的求解(可以用作$p$,$q$随着$i$变化的模型)。我没有求解出来的思路,但我觉得如果只是求解较优解的话可以假设决策点在黑羊获取优势的边界处,即减少$j$至刚好$\frac{ip}{i+j}>\frac{jq}{i+j}$.这本可以用C++模拟的,但我和毛昕渝交流后发现他已经对于这个模型模拟过了,我就不再做一遍了。
\end{document}

