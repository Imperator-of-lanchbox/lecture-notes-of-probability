% This is a template for lecture notes.
\documentclass[12pt]{article}
\usepackage{amssymb}
\usepackage[UTF8]{ctex}
\usepackage{amsmath}
\usepackage{amsthm}
\usepackage{geometry}
\usepackage{booktabs}
\usepackage{bm}
\usepackage{cite}
%\usepackage{CJK}
\usepackage[many]{tcolorbox}
%\tcbuselibrary{listingsutf8}
%\tcbuselibrary{skins, breakable, theorems, most}
%\geometry{a4paper,bottom = 3cm,left = 3cm, right = 3cm}
\CTEXoptions[today=old]
%for reference
\usepackage{hyperref}
\usepackage[capitalise]{cleveref}
\crefname{enumi}{}{}


\newtheoremstyle{mythm}{1.5ex plus 1ex minus .2ex}{1.5ex plus 1ex minus .2ex} 
    {}{\parindent}{\bfseries}{}{1em}{} 
\theoremstyle{mythm}
\newtheorem{theorem}{Theorem}
\newtheorem{lemma}[theorem]{Lemma}
\newtheorem{corollary}[theorem]{Corollary}
\newtheorem{fact}[theorem]{Fact}
\newtheorem{definition}[theorem]{Definition}
\newtheorem*{remark}{Remark}

%\newenvironment{proof}{\noindent \textbf{Proof:}}{$\Box$}

%to use newcommand for convenience
\newcommand\field{\mathbb{F}}
\newcommand\Real{\mathbb{R}}
\newcommand\Q{\mathbb{Q}}
\newcommand\Z{\mathbb{Z}}
\newcommand\complex{\mathbb{C}}
\newcommand\cc{\mathcal{C}}
\newcommand\uu{\mathcal{U}}
\newcommand\pp{\mathcal{P}}
\newcommand\ff{\mathcal{F}}
\renewcommand\refname{Reference}
\renewcommand{\proofname}{Proof}
\DeclareMathOperator{\range}{range}   

\title{Proof of Extended BC1}
\author{马浩博 518030910428}
\date{\today}
\begin{document}
\maketitle

  \begin{theorem}
  If $\lim_{n\to \infty}A_n=0$ and $\sum_{n = 1}^{\infty}P(A_n^c\cap A_{n+1})<\infty$ , then $P(A_n\ i.o.)=0$
  \end{theorem}
    
\begin{proof}
Let $G_m=\cup_{n>m} A_n$ and $G_m \downarrow G$, where $G := lim sup A_n$
By the proof of BC1, we know that: 
$$P(A_n i.o.)=P(G) \leq \lim_{m\to \infty}P(G_m)$$
Because $\sum_{n = 1}^{\infty}P(A_n^c\cap A_{n+1})<\infty$, we know that $\lim_{m\to \infty}\sum_{m}^{\infty}P(A_n^c\cap A_{n+1})=0$.
And by $\cup_{n\geq m} (A_n^c\cap A_{n+1})=G_m/(A_m\cap A_{m+1})$, we can get that:
$$\lim_{m\to \infty}P(G_m) \leq \lim_{m\to \infty}P(\cup_{n\geq m} (A_n^c\cap A_{n+1}))+\lim_{n\to \infty}A_n \leq \lim_{m\to \infty}\sum_{m}^{\infty}P(A_n^c\cap A_{n+1})=0$$

So $P(A_n i.o.)=0$.
\end{proof} 

\end{document}