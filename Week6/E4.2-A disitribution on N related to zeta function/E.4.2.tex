\documentclass[12pt]{article}
\usepackage{amssymb}
%\usepackage[UTF8]{ctex}
\usepackage{amsmath}
\usepackage{amsthm}
\usepackage{geometry}
\usepackage{booktabs}
\usepackage{enumerate}
\usepackage{bm}
\usepackage{cite}
%\usepackage{CJK}
\usepackage[many]{tcolorbox}
%\tcbuselibrary{listingsutf8}
%\tcbuselibrary{skins, breakable, theorems, most}
%\geometry{a4paper,bottom = 3cm,left = 3cm, right = 3cm}

%for reference
\usepackage{hyperref}
\usepackage[capitalise]{cleveref}
\crefname{enumi}{}{}

\newtheorem{theorem}{Theorem}
\newtheorem{lemma}[theorem]{Lemma}
\newtheorem{proposition}[theorem]{Proposition}
\newtheorem{question}[theorem]{Question}
\newtheorem{definition}[theorem]{Definition}
\newtheorem*{remark}{Remark}
%\newenvironment{proof}{\noindent \textbf{Proof:}}{$\Box$}

%to use newcommand for convenience
\newcommand\field{\mathbb{F}}
\newcommand\R{\mathbb{R}}
\newcommand\Q{\mathbb{Q}}
\newcommand\Z{\mathbb{Z}}
\newcommand\N{\mathbb{N}}
\newcommand\cc{\mathcal{C}}
\newcommand\bb{\mathcal{B}}
\newcommand\pp{\mathcal{P}}
\newcommand\PP{\mathbb{P}}
\newcommand\nn{\mathcal{N}}
\newcommand\ff{\mathcal{F}}
\newcommand\one{\bm{1}}
\newcommand\eps{\varepsilon}
\newcommand\pr[1]{\mathcal{P} \left( #1\right)}
\DeclareMathOperator{\leb}{Leb}   
\DeclareMathOperator{\diff}{d}   

\title{E4.2: A distribution on $\N$ related to zeta function}
\author{Xinyu Mao}
\date{\today}
\begin{document}
\maketitle

Let $s > 1$, and define $\zeta(s) := \sum_{n \in \N} \frac{1}{n^s}$, as usual. 
Let $X$ and $Y$ be independent $\N$-valued random variables with
$$
    \pp(X = n) = \pp(Y = n) = \frac{n^{-s}}{\zeta(s)}.
$$
Define events $E_n := \{ \text{$X$ is divisible by $n$}\}$.
One can easily see that 
\begin{equation} \label{factor}
    \pp(E_n) = \sum_{i \in \N} \pp(X = ni) 
    = \sum_{i \in \N} \frac{(ni)^{-s}}{\zeta(s)}
    = \frac{n^{-s}}{\zeta(s)} \sum_{i \in \N} i^{-s}= n^{-s}.
\end{equation}
Everything is set up now, and we shall explore some interesting properties of 
$X$ and $Y$.

\begin{proposition} \label{ind}
    $(E_p : p \in \PP)$ are independent with $\PP$ denoting all prime numbers.
\end{proposition}
\begin{proof}
    Let $S := \{p_1,p_2,\dots,p_m\}$ be a finite subset of $\PP$.
    By definition of $E_n$ and \cref{factor}, we immediately have 
    $$
        \pr{\bigcap_{i = 1}^m E_i} = \pr{E_{p_1p_2\cdots p_m}}
        = (p_1p_2\cdots p_m)^{-s} = \prod_{i = 1}^m\pp(E_i),
    $$
    which implies  $(E_p : p \in \PP)$ are independent.
\end{proof}

We can use this observation to cast light on \textbf{Euler's formula}
\begin{equation} \label{euler}
    \frac{1}{\zeta(s)} = \prod_{p \in \PP} (1 - \frac{1}{p^s}).
\end{equation}
The LHS of \cref{euler} is obviously $\pp(X = 1)$. 
The event that $X = 1$ can also be described as 
$E := \{p \not{|} X, \forall p \in \PP\} = \bigcap_{p \in \PP}E_p^c$.
Combining \cref{factor} and \cref{ind}, 
we have $\pp(E) = \prod_{p\in \PP} \pp(E_p^c) = 
\prod_{p \in \PP} (1 - \frac{1}{p^s})$.

Clearly, $(E_{p^2},p \in \PP)$ is independent as well(the argument is similar to \cref{ind}).
Hence,
$$
    \pp(\text{no square other than 1 divides $X$}) 
    = \pr{\bigcap_{p \in \PP} E_{p^2}^c} 
    = \prod_{p \in \PP} (1 - \frac{1}{p^{2s}}) = \frac{1}{\zeta(2s)},
$$ 
where the last equality follows from \cref{euler}.

Define r.v. $H := \gcd(X,Y)$, and we shall show that
\begin{proposition} \label{gcd}
    $\pp(H = n) = \frac{n^{-2s}}{\zeta(2s)}$.
\end{proposition}
\begin{proof}
    We need the concept of conditional probability 
    from elementary theory. Note that 
    $$
        \pr{X/n = k| E_n}
        = \frac{(nk)^{-s}}{\sum_{i \geq 1}(ni)^{-s}} 
        = \frac{k^{-s}}{\zeta(s)}.
    $$
    Hence, $(X/n | E_n)$ has the same distribution as $X$.
    Let $F_n$ be the event that $n$ divides $Y$, 
    then $(Y/n|E_n)$ also has the same distribution.
    Thus,
    \begin{equation} \label{main}
    \begin{aligned}
        \pp(H = n) &= \pr{E_n \cap F_n \cap {\gcd (\frac{X}{n},\frac{Y}{n}) = 1}} \\
         &= \pp(E_n) \pp(F_n) \pr{\gcd (\frac{X}{n},\frac{Y}{n}) = 1 | E_n \cap F_n} \\
         &= n^{-s} \cdot n^{-s} \cdot \pp(\gcd(X,Y) = 1).
    \end{aligned}
    \end{equation}
    Since $(E_p,F_p: p\in\PP)$ is independent, 
    $$
        \pp(\gcd(X,Y) = 1) = \pp(\bigcap_{p \in \PP}(E_p \cap F_p)^c)
        = \prod_{p\in \PP} (1 - \frac{1}{p^{2s}}) = \frac{1}{\zeta(2s)}.
    $$
    Plugging this into \cref{main}, we get the desired result.
\end{proof}

However, we have not rigorously defined conditional probability yet.
Can we avoid using conditional probability?
The answer is yes, but we need a useful tool, \textbf{Möbius function},
which is defined as 
\begin{align*}
    &\mu : \N \to \{-1,0,1\} \\
     &n \mapsto \begin{cases}
        (-1)^{m}, \text{if $n = p_1p_2\cdots p_m$, where $(p_i)_{i \in [m]}$ are distinct prime numbers};\\
        0, \text{otherwise}.
    \end{cases}
\end{align*}

Here we list two facts about $\mu$ without giving detailed proof
\footnote{This kind of technique is called \textbf{Möbius inversion}.}.
\begin{proposition} \label{mu} 
    \begin{enumerate}[(i)]
        \item $\sum_{d|n} \mu(d) = [n = 1]$.
        \item $\sum_{n \in \N} \frac{\mu(n)}{n^s} 
        = \prod_{p \in \PP} \frac{1}{(1 - p^{s})} = \frac{1}{\zeta(s)}$.
    \end{enumerate}
\end{proposition} 
\begin{proof}
    (i) can be proved by counting the number of $-1$ and $+1$
    among all $\mu(d)$. (ii) is simply expansion of the product.
\end{proof}

Now we are well-equipped to give the coup de grace. 
\begin{proof}[Another proof of \cref{gcd}]
    Let $G_n$ be the event that $X,Y$ are both divisible by $n$,
    which is equivalent to $n | H$.
    Clearly, $\pp(G_n) = n^{-2s}$.
    By \cref{mu} (i), 
    \begin{equation}
        \tag{*}
        \pp(H = n) = \sum_{k = 1}^\infty \pp(H = kn) \sum_{d | k} \mu(d) 
        = \sum_{d = 1}^{\infty} \mu(d) \sum_{i = 1}^{\infty} \pp(H = idn).    
    \end{equation}
    Note that $\sum_{i = 1}^{\infty} \pp(H = in) = \pp(G_n)$ for all $n \in \N$.
    Hence, 
    \begin{align*}
        (*) = \sum_{d = 1}^{\infty} \mu(d) \pp(G_{dn}) 
        &= \sum_{d = 1}^{\infty} \mu(d) (dn)^{-2s} \\
        &= n^{-2s} \sum_{d = 1}^{\infty}  \mu(d) d^{-2s} \\
        &= \frac{n^{-2s}}{\zeta(2s)},
    \end{align*}
    where the last equality follows from \cref{mu} (ii).
\end{proof}

\begin{remark}
    If we take $s = 1$(this makes no sense) in \cref{gcd}, we get 
    the probability that $X,Y$ are relatively prime is $\frac{1}{\zeta(2)}$.
    It is a coincidence that the \textbf{arithmetic density} of relatively prime pairs 
    among $\N^2$ is indeed $\frac{1}{\zeta(2)}$! Formally, 
    $$
        \lim_{N \to \infty} \frac{\sum_{i,j \in [N]}[\gcd(i,j) = 1]}{N^2} = \frac{1}{\zeta(2)}.
    $$    
\end{remark}
\end{document}