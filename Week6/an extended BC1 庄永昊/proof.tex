\documentclass[UTF8, a4paper, linespread=1.5]{article}

\usepackage{tcolorbox, listings}
\usepackage{geometry, amsmath, enumerate, indentfirst}
\usepackage{color, bm, extarrows, ulem}
\usepackage{amsthm}
\usepackage{amssymb}
\usepackage{nameref, hyperref}
\usepackage{algorithm}
\usepackage{algorithmic}
 % \geometry{top=3cm, bottom=3cm, left=1.5cm, right=1.5cm}

\usepackage{enumitem}
\setenumerate[1]{itemsep=0pt,partopsep=0pt,parsep=\parskip,topsep=5pt}
\setitemize[1]{itemsep=0pt,partopsep=0pt,parsep=\parskip,topsep=5pt}

% \usepackage{adjustbox}

\renewcommand\contentsname{Contents}

\tcbuselibrary{skins, breakable, theorems}

% \setlength{\leftskip}{10pt}
\setlength{\parindent}{10pt}
% \setlength{\parskip}{2em}
\renewcommand{\baselinestretch}{1.3}

\newcounter{RomanNumber}
\newcommand{\mrm}[1]{(\setcounter{RomanNumber}{#1}\Roman{RomanNumber})}

\renewcommand{\proofname}{\indent \textbf {proof}}

\newtheorem{theorem}{Theorem}
\newtheorem{lemma}[theorem]{Lemma}
\newtheorem{proposition}[theorem]{Proposition}
\newtheorem{corollary}[theorem]{Corollary}
\newtheorem{fact}[theorem]{Fact}
\newtheorem{definition}[theorem]{Definition}
\newtheorem{remark}[theorem]{Remark}
\newtheorem{question}[theorem]{Question}
\newtheorem{answer}[theorem]{Answer}
\newtheorem{exercise}[theorem]{Exercise}
\newtheorem{example}[theorem]{Example}
%\newenvironment{proof}{\noindent \textbf{Proof:}}{$\Box$}
\newtheorem{observation}[theorem]{Observation}

\newtcbtheorem{thm}{}
  {enhanced, theorem name and number, code={\edef\@currentlabelname{#2}}, 
  frame code={
        % \path[thick, draw] (frame.north west) -| (frame.north east) -| (frame.south east) -| (frame.south west) -| (frame.north west);
        \path[thick, draw] (frame.north west)  +(.5\baselineskip,0) -| +(0,-.5\baselineskip);
        % \path[thick, draw] (frame.north east) +(-.5\baselineskip,0) -| +(0,-.5\baselineskip);
        % \path[thick, draw] (frame.south west) +(.5\baselineskip,0) -| +(0,.5\baselineskip);
        \path[thick, draw] (frame.south east) +(-.5\baselineskip,0) -| +(0,.5\baselineskip);
    },
    left=1mm, right=1mm, top=1mm, bottom=1mm,
    colback=black!5,
    colframe=red!75!black,
    colbacktitle=black!0,
    coltitle=black!100,
    fonttitle=\bfseries}{thm}


\usepackage{xparse}
\NewDocumentEnvironment{qte}{m}{\begin{tcolorbox}[breakable, leftrule=2mm, rightrule=-0.1mm, toprule=-0.1mm, bottomrule=-0.1mm, arc=0mm, colframe=black!30!white, colback=white, coltext=white!50!black]}{\\\rightline{#1}\end{tcolorbox}}

\usepackage{environ}
\RenewEnviron{math}{%
\begin{align*}
\BODY
\end{align*}
}

\title{Extended BC1}
\date{\today}
\author{庄永昊}

\begin{document}
\maketitle
\begin{thm}{}{}
    For a sequence of events $\{E_n\}_{n\in\mathbb N}$ with respect to probability measure $p$, 
    if there is $p(E_n^c\cap E_{n+1})<\infty$ and $\lim_{n\to \infty}E_n=0$, then $p(E_n\ i.o.)=1$
\end{thm}
\begin{proof}[Proof]
    Let $G_m=\cup_{n>m} E_n$ and $G_m'=\cup_{n\geq m} (E_n^c\cap E_{n+1})$. Then there is: 
    $$G_m'=\cup_{n\geq m} (E_n^c\cap E_{n+1})=(\cup_{n>m} E_n)/(E_m\cap E_{m+1})=G_m/(E_m\cap E_{m+1})$$
    Then use (1.9,b) and (1.10,a) in textbook there is: 
    $$p(E_n i.o.)=p(\cap_{m\in\mathbb Z^+}G_m)\leq p(G_m)\leq p(G_m')+p(E_m)$$
    As $p(G_m')\leq\sum_{n\geq m}(E_n^c\cap E_{n+1})$, let $m\uparrow\infty$ there is $\lim_{m\to\infty}p(G_m')=0$. 
    By assumption there is $\lim_{m\to\infty}p(E_m)=0$. 

    Combining the two result, let $m\uparrow\infty$ there is 
    $$0\leq p(E_n i.o.)\leq \lim{m\to\infty}p(G_m')+p(E_m)=0$$
\end{proof}

    Now construct a condition that does not fit the condition of normal BC1 but fit this condition: 

    For a point $x$ on $[0,1]$, 
    $E_n$ is that $x$ is smaller than $\frac{1}{n}$, so $p(E_n)=\frac{1}{n}$(thus $p(E_n)\downarrow 0$) and 
    $p(E_{n-1}^c\cap E_n)=0$(thus the sum is $0<\infty$), while $\sum_{n}p(E_n)=\infty$. 

    It is obvious that $p(E_n\ i.o.)=0$(only $x=0$ makes it infinitely often). 
\end{document}
