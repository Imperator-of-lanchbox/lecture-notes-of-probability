\documentclass[UTF8, 12pt]{article}
\fontfamily{DENGL.TTF}
\usepackage{xeCJK}
\usepackage{enumitem}
\usepackage{amsmath}
\usepackage{amssymb}
\usepackage{amsfonts}
\usepackage{mathrsfs}
\usepackage{XCharter}
\usepackage{fancyhdr}
\usepackage{eulervm}
\usepackage{graphicx}
\usepackage{mdframed}
\usepackage{ntheorem}
\usepackage{lipsum}

\topmargin -.5in
\textheight 9in
\oddsidemargin -.25in
\evensidemargin -.25in
\textwidth 7in
\pagestyle{fancy}

\newenvironment{proof}{\noindent\ignorespaces\textbf{Proof:}}{\hfill $\square$\par\noindent}
\newenvironment{solution}{\noindent\ignorespaces\textbf{Solution:}}{\hfill $\square$\par\noindent}

\theoremstyle{break}
\newtheorem{claim}{Claim}
\newtheorem{problem}{Problem}
\newtheorem{lemma}{Lemma}
\newtheorem*{theorem*}{Theorem}

\newenvironment{rcases}{\left.\begin{aligned}}{\end{aligned}\right\rbrace}

\title{Independent Normal Distribution Variable Sequence}
\author{张志成 518030910439}
\date{\today}

\begin{document}
    \maketitle
    
    \begin{problem}[\textbf{Exercise 4.5} of \textit{Chapter E}]
        \textit{Background: } \par
        A random variable $G$ has a normal $N(0,1)$ distribution, then for $x > 0$,
        $$
        P(G > x) = \frac{1}{\sqrt{2\pi}}\int_{x}^{\infty}e^{-\frac{1}{2}y^2}dy.
        $$
        \textbf{Note that this property only is all we need regarding random variable with normal distribution in this problem.}
        \begin{enumerate}
            \item Prove that  $$ P(G > x) \leq \frac{1}{x\sqrt{2\pi}}e^{-\frac{1}{2}x^2}. $$
            \item
                Let $X_1,X_2,...$ be a sequence of independent $N(0,1)$ variables. Prove that with probability $1$, $L \leq 1$, where
                $$
                L := lim\ sup(\frac{X_n}{\sqrt{2logn}}).
                $$
            \item Prove that $$ P(L = 1) = 1.$$
        \end{enumerate}
    \end{problem}

    \begin{proof}
        \begin{enumerate}
            \item This is proven by manipulating the integral.
            \begin{align*}
                P(G > x) &= \frac{1}{\sqrt{2\pi}}\int_{x}^{\infty}e^{-\frac{1}{2}y^2}dy \\
                &\leq \frac{1}{\sqrt{2\pi}} \cdot \frac{1}{x} \cdot \int_{x}^{\infty}e^{-\frac{1}{2}y^2}\cdot y \cdot dy \\
                &= \frac{1}{x\sqrt{2\pi}} \int_{\frac{1}{2}x^2}^{\infty}e^{-y}\cdot dy \\
                &= \frac{1}{x\sqrt{2\pi}} (-e^{-y})\bigm|^{\infty}_{\frac{1}{2}x^2} \\
                &= \frac{1}{x\sqrt{2\pi}}e^{-\frac{1}{2}x^2}.
            \end{align*}

            \item Let $E_n$ denote the event that $\frac{X_n}{\sqrt{2logn}} > \sqrt{1 + \epsilon}$.
            \begin{align*}
                \sum_{i \in \mathbb{N}} P(E_i) &= \sum_{i \in \mathbb{N}} P(\frac{X_n}{\sqrt{2logn}} > \sqrt{1 + \epsilon}) = \sum_{i \in \mathbb{N}} P(X_n > (\sqrt{1 + \epsilon})\sqrt{2logn})\\
                &= \sum_{i \in \mathbb{N}} \frac{1}{\sqrt{2\pi}}\int_{ (\sqrt{1 + \epsilon})\sqrt{2logn} }^{\infty}e^{-\frac{1}{2}y^2}dy \\
                &\leq \sum_{i \in \mathbb{N}} \frac{1}{ (\sqrt{1 + \epsilon})\sqrt{2logn} \sqrt{2\pi}} e^{-\frac{1}{2}((\sqrt{1 + \epsilon})\sqrt{2logn})^2} \\
                &= \sum_{i \in \mathbb{N}} \frac{1}{ (\sqrt{1 + \epsilon})\sqrt{2logn} \sqrt{2\pi}} e^{-(1 + \epsilon)logn} \\
                &< \sum_{i \in \mathbb{N}} \frac{1}{ 2\sqrt{(1 + \epsilon)\pi}} \cdot \frac{1}{n ^ {1 + \epsilon}}. \\
            \end{align*}
            $\sum_{i \in \mathbb{N}} \frac{1}{n ^ {1 + \epsilon}}$ converges, thus $$  \sum_{i \in \mathbb{N}} P(E_i) < \infty. $$

            By \textbf{First Borel-Cantelli Lemma(BC1)}, we have $$ P(E_n, i.o.) = 0, $$
            thus $$ P(E_n^c,ev) = P(\frac{X_n}{\sqrt{2logn}} \leq \sqrt{1 + \epsilon}, ev) = 1. $$
            
            Finally,
            \begin{align*}
                P(L \leq 1) &= P( lim\ sup(\frac{X_n}{\sqrt{2logn}}) \leq 1) = P(\frac{X_n}{\sqrt{2logn}} \leq 1, ev) \\
                % &\geq P(\frac{X_n}{\sqrt{2logn}} \leq 1, ev) \\
                &= \lim_{\epsilon\to 0} P(\frac{X_n}{\sqrt{2logn}} \leq \sqrt{1 + \epsilon}, ev) \\
                &= 1.
            \end{align*}
            
            \item We only need to prove that $ P(L \geq 1) = 1$, then $ P(L \geq 1) = P(L \leq 1) = 1 \Longrightarrow P(L = 1) = 1 $.
            We have
            \begin{align*}
                P(L \geq 1) = P(lim\ sup(\frac{X_n}{\sqrt{2logn}}) \geq 1) = P(\frac{X_n}{\sqrt{2logn}} \geq 1, i.o.). \\ 
            \end{align*}
            Thus we need to prove that $P(\frac{X_n}{\sqrt{2logn}} \geq 1, i.o.) = 1$.

            Let $E_n$ denote the event that $\frac{X_n}{\sqrt{2logn}} \geq 1$. It is easy to verify that $(E_n)$ are independent events since $(X_n)$ are independent random variables.
            \begin{align*}
                \sum_{i \in \mathbb{N}} P(E_i) = \sum_{i \in \mathbb{N}} P(\frac{X_n}{\sqrt{2logn}} \geq 1) = \sum_{i \in \mathbb{N}} P(X_n \geq \sqrt{2logn}).
            \end{align*}
            Similarly to \textit{(1),(2)}, we first show that $$ P(G \geq x) \geq \frac{1}{\sqrt{2\pi}}\cdot \frac{x}{x^2+1} \cdot e^{-\frac{1}{2}x^2}. $$
            Thus,
            \begin{align*}
                \sum_{i \in \mathbb{N}} P(E_i) &\geq \sum_{i \in \mathbb{N}} \frac{1}{\sqrt{2\pi}}\cdot \frac{\sqrt{2logn}}{(\sqrt{2logn})^2+1} \cdot e^{-\frac{1}{2}(\sqrt{2logn})^2} \\
                &= \sum_{i \in \mathbb{N}} \frac{1}{\sqrt{2\pi}}\cdot \frac{\sqrt{2logn}}{2logn+1} \cdot e^{-logn} \\
                &\geq \sum_{i \in \mathbb{N}} \frac{1}{\sqrt{2\pi}} \cdot \frac{1}{n \cdot \sqrt{2logn}} = \infty.
            \end{align*}
            Therefore, by \textbf{Second Borel-Cantelli Lemma(BC2)}, we have $$ P(\frac{X_n}{\sqrt{2logn}} \geq 1, i.o.) = P(E_n, i.o.) = 1, $$ which finishes the proof.
        \end{enumerate}
    \end{proof}
\end{document}