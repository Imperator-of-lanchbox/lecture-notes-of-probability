\documentclass[a4paper, linespread=1.5]{article}
%\usepackage[UTF8]{ctex}
\usepackage{xeCJK, geometry, amsmath, amssymb, amsthm}
\usepackage{graphicx, keyval}
\usepackage[dvipsnames,svgnames,x11names]{xcolor}
\usepackage{float, ifthen, calc, ifplatform, fancyvrb}
\usepackage{minted, hyperref, enumerate, multicol}
\usepackage[all]{xy}
\usepackage{ulem}
%\usepackage{epstopdf}
\usepackage{mathrsfs}
\usepackage{cancel}
\usepackage{algorithm, algorithmic}

%\setlength{\parskip}{0.2\baselineskip}
\setlength{\parindent}{2em}
%\geometry{left=2.7cm,right=2.7cm,top=2.7cm,bottom=2.7cm}


\newtheorem{theorem}{Theorem}
\newtheorem{proposition}[theorem]{Proposition}
\newtheorem{lemma}[theorem]{Lemma}
\newtheorem{corollary}[theorem]{Corollary}
\newtheorem{definition}[theorem]{Definition}
\newtheorem{exercise}[theorem]{Exercise}

\newtheorem{innercustom}{\customname}
\providecommand{\customname}{}
\newcommand{\newcustomtheorem}[2]{
    \newenvironment{#1}[1]
    {
        \renewcommand\customname{#2}
        \renewcommand\theinnercustom{##1}
        \innercustom
    }
    {\endinnercustom}
}
\newcustomtheorem{customthm}{Theorem}
\newcustomtheorem{customprop}{Proposition}
\newcustomtheorem{customlemma}{Lemma}
\newcustomtheorem{customcorollary}{Corollary}
\newcustomtheorem{customdef}{Definition}
\newcustomtheorem{customex}{Exercise}
\newcustomtheorem{customremark}{Remark}

\newcommand{\Natural}{\mathbb{N}}
\newcommand{\Real}{\mathbb{R}}
\newcommand{\BorelSet}{\mathcal{B}}
\newcommand{\addbigcup}{\bigcup{\kern-1.12em{+}}\kern0.3em}
\newcommand{\nth}[1]{#1\textsuperscript{th}}
\newcommand{\IndicatorFunc}[1]{\mathbf{1}_{#1}}

\begin{document}
    \title{$P(A_n) \rightarrow 0$ and $\sum_{n = 1}^{\infty} P(A_n^c \cap A_{n + 1} < \infty)$ \\ $\Rightarrow$ \\ $P(A_n, \textrm{i.o.}) = 0$}
    \author{赖睿航 518030910422}
    \date{\today}
    \maketitle
    
    \textit{This note proves a extension of the First Borel-Cantelli Lemma(BC1).}

    \medskip
    
    Let $(\Omega, \mathcal{F}, P)$ be a probability space and $(A_n \colon n \in \Natural) \subseteq \mathcal{F}$ be a sequence of events. We have the following proposition.
    \begin{proposition}
        If $\lim P(A_n) = 0$ and $\sum\limits_{n = 1}^{\infty} P(A_n^c \cap A_{n + 1}) < \infty$, then $P(A_n, \textrm{i.o.}) = 0$. \cite{Chandra2012}
    \end{proposition}

    \begin{proof}
        For an arbitrary fixed $n \in \Natural$, we have
        \begin{align*}
            P(A_n, \textrm{i.o.}) &= P(\lim \sup A_n) \\
            &= P(\bigcap_n \bigcup_{m \geqslant n} A_m) \\
            &\leqslant P(\bigcup_{m \geqslant n} A_n) \\
            &= P(A_n \sqcup \bigsqcup_{m > n}(A_m \setminus \bigcup_{n \leqslant i < m}A_i)) \\
            &= P(A_n) + \sum_{m > n} P(A_m \setminus \bigcup_{n \leqslant i < m}A_i) \\
            &= P(A_n) + \sum_{m > n} P(A_m \cap \bigcap_{n \leqslant i < m} A_i^c) \\
            &\leqslant P(A_n) + \sum_{m > n} P(A_m \cap A_{m - 1}^c) \\
            &= P(A_n) + \sum_{m \geqslant n} P(A_{m + 1} \cap A_m^c).
        \end{align*}
        Since it is given that $\lim P(A_n) = 0$ and $\sum\limits_{n = 1}^{\infty} P(A_n^c \cap A_{n + 1}) < \infty$, we know that $\lim\limits_{n \rightarrow \infty} \sum\limits_{m \geqslant n} P(A_{m + 1} \cap A_m^c) = 0$. If we let $n \rightarrow \infty$, immediately we get
        \begin{align*}
            P(A_n, \textrm{i.o.}) &\leqslant \lim_{n \rightarrow \infty} P(A_n) + \lim_{n \rightarrow \infty} \sum_{m \geqslant n} P(A_{m + 1} \cap A_m^c) \\
            &= 0 + 0 = 0.
        \end{align*}
        As $P(A_n, \textrm{i.o.}) \geqslant 0$ always holds, it follows that $P(A_n, \textrm{i.o.}) = 0$.
    \end{proof}

    \bibliography{bc1-extension}
    \bibliographystyle{abbrv}
\end{document}
