% This is a template for lecture notes.
\documentclass{article}
\usepackage[UTF8]{ctex}
\usepackage{amssymb}
\usepackage{amsmath}
\usepackage{amsthm}
\usepackage{geometry}
\usepackage{booktabs}
\usepackage{bm}
\usepackage{tcolorbox}
\CTEXoptions[today=old]
%Some commonly used notations
%\geometry{a4paper,bottom = 3cm,left = 3cm, right = 3cm}

%for reference
\usepackage{hyperref}

\usepackage{indentfirst}
\setlength{\parindent}{2em}
\usepackage[capitalise]{cleveref}
\crefname{enumi}{}{}

\newtheorem{theorem}{Theorem}
\newtheorem{lemma}[theorem]{Lemma}
\newtheorem{proposition}[theorem]{Proposition}
\newtheorem{corollary}[theorem]{Corollary}
\newtheorem{fact}[theorem]{Fact}
\newtheorem{definition}[theorem]{Definition}
\newtheorem{remark}[theorem]{Remark}
\newtheorem{question}[theorem]{Question}
\newtheorem{answer}[theorem]{Answer}
\newtheorem{exercise}[theorem]{Exercise}
\newtheorem{example}[theorem]{Example}
%\newenvironment{proof}{\noindent \textbf{Proof:}}{$\Box$}
\newtheorem{observation}[theorem]{Observation}

%to use newcommand for convenience
\newcommand\field{\mathbb{F}}
\newcommand\Real{\mathbb{R}}
\newcommand\Q{\mathbb{Q}}
\newcommand\Z{\mathbb{Z}}
\newcommand\complex{\mathbb{C}}

%this is how we define operators.
\DeclareMathOperator{\rank}{rank} % rank

\title{Notes on Week 6}
\author{Ruihan GUO and Tong CHEN}
\date{\today}
 
\begin{document}

    \maketitle
\section{}

\begin{definition}
Dependency graph for a sequence of events $E_1,\dots,E_n$ is a graph $G = (V,E)$ such that $V = \{1,\dots,N\}$.\par
$E_i$ is independent with $\{E_j:i\sim j\notin E\}$ for all $i\in V$\par
\end{definition}
\paragraph{Lovasz Local Lemma}
Presented in 1975 by Erd{\"o}s and Lovasz.\par
For an event sequence $E_1,\dots,E_n$, and G is a dependency graph of it.\par
\subparagraph{Premise}
\begin{align*}
&(1)\exists p \in (0,1) P(E_i) \leq p \qquad\forall i.\\
&(2)\max deg_G(v) \leq d\\
&(3)4dp \leq 1\\
\end{align*}
\subparagraph{Conclusion}$\qquad P\left(\cap E_{i}^{c}\right) > 0$\par
\paragraph{Proof}Using inductive method\par
for $s = 0,1,\dots,N-1$, $\forall \left|S\right| \leq s$\par
$\left\{
\begin{aligned}
(a)& , & P\left(\bigcap_{j \in S}E_{j}^{c}\right)>0), \\
(b)& , & \forall k \in [n]\setminus S, P(E_k\cap\bigcap_{j \in S} E_{j}^{c}) \leq 2pP\left(\bigcap_{j \in S}E_{j}^{c}\right).
\end{aligned}\right.$\\\par
Then let's start induction!\par
\paragraph{s=0} It's easy to verify it.\par
\paragraph{s$>$0}\par
\subparagraph{For expression a}
\begin{align*}
  P\left(\bigcap_{j \in [n]}E_{j}^{c}\right) &= \frac{P\left(\bigcap_{j \in [n]}E_{j}^{c}\right)}{P\left(\bigcap_{j \in [n-1]}E_{j}^{c}\right)}\times\dots\times\frac{P\left(\bigcap_{j \in [1]}E_{j}^{c}\right)}{P\left(\bigcap_{j \in [0]}E_{j}^{c}\right)}\\
  &\geq (1-2p)^n
\end{align*}
Due to $4dp\leq1$, $2p\leq \frac{1}{2d}$, so $1-2p \geq \frac{1}{2}$, which means the probability is correct.\par
\subparagraph{For expression b}To prove $P\left(E_k\big|\bigcap_{j\in S}E_{j}^{c}\right)\leq 2p$ when $(|S| = s)$\par
Separate the points in S into two parts:\par
$S_1 = \{j\in S: j\sim k$ in $G\}$\par
$S_2 = S \setminus S_1$\par
When $S_1$ is an empty set, $P(E_k|\bigcap_{j\in S}E_j^{c})=P(E_k)\leq p < 2p$\par
Otherwise, $S_1 \neq \emptyset \rightarrow |S_2|<s$\par
Let $F_{S_1} = \bigcap_{j\in S_1} E_{j}^{c}$  $F_{S_2} = \bigcap_{j\in S_2} E_{j}^{c}$\par
\begin{align*}
P\left(E_k\Big|\bigcap_{j\in S}E_{j}^{c}\right) &= P\left(E_k|F_{S_1}\cap F_{S_2}\right)\\
&=\frac{P\left(F_{S_1}\cap E_k|F_{S_2}\right)}{P\left(F_{S_1}|F_{S_2}\right)}\\
P\left(F_{S_1}\cap E_k|F_{S_2}\right) &\leq P\left( E_k|F_{S_2}\right) \\= P\left( E_k\right)\leq p\\
P\left(F_{S_1}|F_{S_2}\right)&=P\left(\bigcap_{i \in S_1}E_{i}^{c}\Big|\bigcap_{j \in S_2}E_{j}^{c}\right)\\
&\geq 1-\sum_{i\in S_1}P\left(E_i\Big|\bigcap_{j \in S_2}E_{j}^{c}\right)\\
&\geq1-2pd \geq \frac{1}{2}
\end{align*}
So that $\frac{P\left(F_{S_1}\cap E_k|F_{S_2}\right)}{P\left(F_{S_1}|F_{S_2}\right)} < 2p$$\hfill\blacksquare$ 
\end{document}
