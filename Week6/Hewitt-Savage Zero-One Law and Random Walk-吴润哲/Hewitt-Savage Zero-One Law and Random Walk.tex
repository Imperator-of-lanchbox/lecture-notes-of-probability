\documentclass[12pt]{article}

%\usepackage[UTF8]{ctex}
\usepackage{geometry}
\usepackage{amsthm}
\usepackage{amsmath}
\usepackage{amssymb}
\usepackage{mathtools}
\usepackage{enumerate}
\usepackage{hyperref} 
\usepackage{tcolorbox}

\geometry{a4paper, left = 2cm, right = 2cm, top = 2cm}

\newcommand\problem[1]{\section*{Problem #1}}

\newcommand\bE{\mathbb{E}}
\newcommand\bF{\mathbb{F}}
\newcommand\bN{\mathbb{N}}
\newcommand\bZ{\mathbb{Z}}
\newcommand\bQ{\mathbb{Q}}
\newcommand\bR{\mathbb{R}}
\newcommand\fC{\mathbf{C}}
\newcommand\fF{\mathbf{F}}
\newcommand\fN{\mathbf{N}}
\newcommand\fQ{\mathbf{Q}}
\newcommand\fR{\mathbf{R}}
\newcommand\fZ{\mathbf{Z}}
\newcommand\cF{\mathcal{F}}
\newcommand\cU{\mathcal{U}}

\newcommand\pro{\mathbf{P}}
\newcommand\ce{\coloneqq}
\newcommand\lproof{\item ``$\Leftarrow$'' :}
\newcommand\rproof{\item ``$\Rightarrow$'' :}

\newcommand{\leb}{\text{Leb}}
\newcommand*{\dif}{\mathop{}\!\mathrm{d}}
\newcommand{\ord}{\text{ord}}
\newcommand{\floor}[1]{\lfloor {#1}\rfloor}
\newcommand{\ind}[1]{\mathbf{1}_{#1}}

\newtheorem{claim}{Claim}
\newtheorem{definition}{Definition}
\newtheorem{lemma}{Lemma}
\newtheorem{theorem}{Theorem}
\newtheorem{corollary}{Corollary}
\newtheorem{remark}{Remark}


\title{Hewitt-Savage Zero-One Law and Random Walk}
\author{WU Runzhe\\
	Student ID : 518030910432\\
	\textsc{Shanghai Jiao Tong University}}
\date{\today}

\begin{document}
	\maketitle
	
	\section{Hewitt-Savage zero-one law}
	
	Let $(\Omega, \cF, \pro)$ be our probability triple.
	
	For a sequence of IID RVs $(X_n)_{n\in\bN}$, let $\cF_n=\sigma(X_1,X_2,\dots,X_n)$, i.e., the $\sigma$-algebra generated by the first $n$ random variables, and $\cF_\infty=\lim\limits_{n\rightarrow\infty}\cF_n$.
	
	By \textit{Doob-Dynkin Lemma} (or \textit{3.13. (d)}), for any $f\in m\cF_\infty$ (namely, $\sigma(f)\subseteq\sigma(X_1,X_2,\dots)$), there exists $Y:\bR^\infty\rightarrow \bR$ such that $f=Y\circ X$ where $X=(X_1,X_2,\dots)$.
	
	By a \textit{finite permutation} of $\bF$ we mean a bijection map $p:\bN\rightarrow\bN$ such that $p=n$ for all but finitely many $n$. We say $f$ is \textit{invariant under finite permutation} or \textit{permutation invariant} or \textit{permutable} if $f=f\circ p$ for every finite permutation $p$ where $f\circ p=Y\circ X\circ p=Y\circ (X_1,X_2,\dots)\circ p\ce Y\circ (X_{p_1},X_{p_2},\dots)$.
	
	We say an event $A$ is permutation invariant if $\ind{A}$ is permutation invariant.
	
	\begin{theorem}
		Suppose that $(X_n)_{n\in\bN}$ is a sequence of IID RVs. Then every permutation invariant event has probability 0 or 1.
	\end{theorem}
	
	\section{Random Walk (trichotomy)}
	
	Let $X_1,X_2,\dots$ be IID RVs, and put $S_n = X_1 + X_2 + \dots$ where $\pro(X_n=0)<1$. Undoubtedly $S_n$ is also a random variable. Furthermore, $\lim \sup S_n$ and $\lim \inf S_n$ are also random variables. By \textit{Hewitt-Savage zero-one law}, the following result is easy to validate.
	
	\begin{lemma}\label{l1}
		$\pro(\lim\sup S_n\in B)=$ 0 or 1 for any $B\in\mathcal{B}$.
	\end{lemma}
	
	\begin{proof}[Proof of lemma \ref{l1}]
		We only need to prove $\lim\sup S_n$ is permutation invariant, or equivalently, $\ind{\lim\sup S_n}$ is permutation invariant. However, it is trivial since $\ind{\lim\sup S_n}$ has nothing to do with the order of first finite random variables.
	\end{proof}
	
	Using the same method in the second part of the proof of Kolmogorov's zero-one law (in 4.11. of our textbook), we obtain the following result.
	
	\begin{corollary}\label{c1}
		$\lim\sup S_n=c$ a.s. for some $c\in[-\infty,+\infty]$.
	\end{corollary}
	
	We call random walk a trichotomy because there are only three possibilities for $\lim S_n$.
	
	\begin{theorem}\label{t1}
		One of the followings happens a.s.:
		\begin{enumerate}[(1)]
			\item $\lim S_n=+\infty$.
			\item $\lim S_n=-\infty$.
			\item $\lim\sup S_n=+\infty, \lim\inf S_n=-\infty$.
		\end{enumerate}
	\end{theorem}
	
	\begin{proof}[Proof of theorem \ref{t1}]
		To finish the proof, we just need to show that the $c$ in corollary \ref{c1} cannot be finite, namely, $c\in\{-\infty,+\infty\}$. 
		
		Suppose $c$ is finite. We know $\lim\sup S_n=c$ a.s., and equivalently, $
		\lim\sup S_{n+1}=c$ a.s.. 
		
		However, obviously we also have $\lim\sup (S_{n+1}-X_1)=c$ a.s. , and thus, $\lim\sup(S_{n+1})=c+X_1$ a.s.. 
		
		Combining these two results, we have $c=c+X_1$ a.s. --- that is, $X_1=0$ a.s., which contradicts our premise $\pro(X_n=0)<1$. 
		
	\end{proof}
	
\end{document}
