% This is a template for lecture notes.
\documentclass[12pt]{article}
\usepackage{amssymb}
%\usepackage[UTF8]{ctex}
\usepackage{amsmath}
\usepackage{amsthm}
\usepackage{geometry}
\usepackage{booktabs}
\usepackage{bm}
\usepackage{cite}
%\usepackage{CJK}
\usepackage[many]{tcolorbox}
%\tcbuselibrary{listingsutf8}
%\tcbuselibrary{skins, breakable, theorems, most}
%\geometry{a4paper,bottom = 3cm,left = 3cm, right = 3cm}

%for reference
\usepackage{hyperref}
\usepackage[capitalise]{cleveref}
\crefname{enumi}{}{}

\newtheorem{theorem}{Theorem}
\newtheorem{lemma}[theorem]{Lemma}
\newtheorem{corollary}[theorem]{Corollary}
\newtheorem{fact}[theorem]{Fact}
\newtheorem{definition}[theorem]{Definition}
\newtheorem*{remark}{Remark}

%\newenvironment{proof}{\noindent \textbf{Proof:}}{$\Box$}

%to use newcommand for convenience
\newcommand\field{\mathbb{F}}
\newcommand\Real{\mathbb{R}}
\newcommand\Q{\mathbb{Q}}
\newcommand\Z{\mathbb{Z}}
\newcommand\complex{\mathbb{C}}
\newcommand\cc{\mathcal{C}}
\newcommand\uu{\mathcal{U}}
\newcommand\pp{\mathcal{P}}
\newcommand\nn{\mathcal{N}}
\newcommand\ff{\mathcal{F}}
\newcommand\one{\bm{1}}
\newcommand\eps{\varepsilon}

\renewcommand\refname{Reference}
\renewcommand{\proofname}{Proof}
\DeclareMathOperator{\range}{range}   
\DeclareMathOperator{\diff}{d}   
\title{Lecture Note 1 : Proof of WLLN and SLLN}
\author{Xinyu Mao}
\date{\today}
\begin{document}
\maketitle

\begin{definition}{Indicator function.}
For $S \subset \Real$,  $\one_S(x) := \begin{cases}
    1, \text{ if $x \in S$}, \\
    0, \text{ otherwise.}
\end{cases}    
$
\end{definition}

\begin{theorem}[WLLN]\label{wlln}
    For all $\eps > 0$, $\lim_{n \to \infty} P[\omega : |\frac{\sum_{i=1}^n d_i(\omega)}{n} - \frac{1}{2}| \geq \eps] = 0$.
\end{theorem}
\begin{proof}
    Let $r_i(\omega) = 2d_i(\omega) - 1$. 
    One can easily check that $\{r_i\}$ are orthogonal functions
    on $\Omega := (0,1]$, that is,
    \begin{equation} \label{ort}
        \int_{\Omega} r_i(\omega)r_j(\omega) \diff \omega = \delta_{ij}.
    \end{equation}
    Let $s_n(\omega) := \sum_{i=1}^n r_i(\omega)$.
    Note that 
    \begin{equation} \label{eq1}
        P\left[\omega : \left|\frac{\sum_{i=1}^n d_i(\omega)}{n} - \frac{1}{2}\right| \geq \eps\right]
        = P[\omega : |s_n(\omega)| \ge n\eps']  
        = \int_{\Omega} \one_{\Omega_2}(\omega) \diff \omega,
    \end{equation}
    where $\eps' = 2\eps, \Omega_2 = \{\omega : |s_n(\omega| \geq n\eps'\}$.
    Since $\eps$ is arbitrarily small,  we can equate $\eps$ and $\eps'$ for simplicity.
    The key observation is that $\one_{\Omega_2}(x) \leq \frac{1}{n^2\eps^2}s_n^2(x)$ and thus
    \begin{equation} \label{eq2}
        \int_{\Omega} \one_{\Omega_2}(\omega) \diff \omega
        \leq \frac{1}{n^2\eps^2} \int_0^1 s_n^2(\omega) \diff \omega
        = \frac{1}{n\eps^2},
    \end{equation}
    where the last equality is obtained by \cref{ort} and the definition of $s_n$.
    Combine \cref{eq1} and \cref{eq2}
    and we get the desired result.
\end{proof}

\begin{definition}
    A set $S \subset R$ is negligible if 
    for all $\eps > 0$, there is a collection of intervals
    $\{I_i\}_{i=1}^{\infty}$ such that $S \subset \bigcup_{i = 1}^{\infty} I_i$,
    and $\sum_{i=1}^\infty |I_i| < \eps$,
    where $|\cdot|$ is the length of the interval.
\end{definition}
\begin{theorem}[SLLN]
   Let $\nn = \{\omega : \lim_{n \to \infty} \frac{s_n(\omega)}{n} = 0\}$,
   $\nn^c$ is uncountable but negligible.
\end{theorem}
\begin{proof}
    For $x \in (0,1]$, say $x = 0.b_1b_2\cdots$, 
    let $f(x) = 0.11b_111b_211b_3\cdots.$ The set 
    $S := \{f(x) : x\in(0,1]\}$ is uncountable, for $f$
    is an injection, and $S \subset \nn^c$ since numbers in $S$
    clearly violate the property of normal number. Hence, 
    $\nn^c$ is uncountable.

    By the same method used in the proof of \cref{wlln}, use $s_n^4$ instead of $s_n^2$,
    we can establish 
    \begin{equation} 
        P[\omega : |s_n(\omega)| \ge n\eps]   \diff \omega
        \leq \frac{1}{n^4\eps^4} \int_0^1 s_n^4(\omega) \diff \omega. 
    \end{equation}
    We rewrite $s_n^4(\omega)$ as 
    \begin{equation}
        s_n^4(\omega) = \sum_{1 \leq i,j,k,l \leq n} 
        r_i(\omega)  r_j(\omega) r_k(\omega)  r_l(\omega). 
    \end{equation}
    We can figure out the integral value of each term 
    by the orthogonality of $\{r_k\}$, which is 
    shown in \cref{table}.
    \begin{table}[h]
        \centering
        \begin{tabular}{c|c}
            \toprule
                Term & Interval value on $\Omega$\\
            \midrule
                $r_i^4$ & 1 \\
                $r_i^2r_j^2$ & 1\\
                $r_i^2r_jr_k (= r_jr_k)$ & 0 \\
                $r_i^3r_j ( = r_ir_j)$ & 0 \\
                $r_ir_jr_kr_l$   & 0  \\
            \bottomrule
        \end{tabular}
        \caption{Interval value of terms in $s_n^4$.  In this table, $i,j,k,l$ are pairwise distinct.}
        \label{table}
    \end{table}
    In $s_n^4$, $n$ terms are of the first kind, 
    $3n(n-1)$ terms are of the second kind, and thus 
    \begin{equation} \label{main}
        P[\omega : |s_n(\omega)| \ge n\eps] \leq  
        \frac{1}{n^4\eps^4} \int_0^1 s_n^4(\omega) \diff \omega 
        =\frac{1}{n^4\eps^4} [n + 3n(n-1)] < \frac{3}{n^2\eps{^4}}.
    \end{equation}

    The key is we can find a decreasing sequence $\{\eps_n\}$ such that 
    \begin{itemize}
        \item $\sum_{n = 1}^\infty \frac{3}{n^2\eps_n^4}$ converges. 
        \item $\lim_{n \to \infty} \eps_n= 0$.
    \end{itemize}
    Let $A_n := \{\omega : |s_n(\omega)| \geq \eps_n\}$.
    Intuitively, numbers in $A_n$ violate the normal property at the $n^{\text{th}}$ digit.
    Note that for all $m > 0$, $\bigcup_{j = m}^\infty A_j$ covers $\nn^c$, 
    because if $\omega \notin A_j$ for all $j \geq m$, $\omega \in \nn$.
    By \cref{main} and the proper choice of $\{\eps_n\}$, we have
    \begin{equation}
        \sum_{n=1}^\infty |A_n| < \sum_{n=1}^\infty \frac{3}{n^2\eps_n^4} < \infty.
    \end{equation}
    Hence, for $\eps > 0$, 
    there is an $m$ such that $\sum_{j = m}^\infty |A_j| < \eps$.
    Meanwhile, $\bigcup_{j = m}^\infty A_j$ covers $\nn^c$, 
    which implies $\nn^c$ is negligible.
\end{proof}

\end{document}