% !TEX program = xelatex
\input{D:/template}
 
\author{Zhiyang Xun}
\title{Reals Allowing Rational Approximation of Too High Order Are Negligible}

\begin{document}     
\maketitle

\begin{definition}
    $A_\phi$ is the set of reals $x$ in $(0, 1]$ such that 
    \[ \abs{x - \frac{p}{q}} < \frac{1}{q^2\phi(q)} \] 
    has infinitely many irreduciple rational solutions $(p, q)$, that is, $(p, q) \in \Z^2$, $q > 0$ and $\gcd(p, q) = 1$. 
\end{definition}

\begin{theorem}
    Suppose that $\phi$ is positive. If $\sum_q \frac{1}{q\phi(q)} < \infty$, then $P (A_\phi) = 0$.
\end{theorem}     

\begin{proof}
    For each $(p, q) \in \Z^+ \times \Z^+$ satisfying $p \leq q$, we define interval 
    $I_{(p, q)} = [\frac{p}{q} - \frac{1}{q^2\phi(q)}, \frac{p}{q} + \frac{1}{q^2\phi(q)}]$.
    Clearly, $A_\phi \subseteq \bigcup I_{(p, q)}$.
    Because each element $x \in A_\phi$ has infinitely many solutions to $\abs{x - \frac{p}{q}} < \frac{1}{q^2\phi(q)}$, and for each $k \geq 1$, only finite $(p, q)$ satisfying $p \leq q < k$, we can see that \[
        A_\phi \subseteq \bigcup_{q=k}^\infty \bigcup_{p=1}^{q} I_{(p, q)} 
    \]

    Now we only need to prove for each $\epsilon > 0$, there is a positive integer $k$ satisfying $\sum_{q=k}^\infty \sum_{p=1}^{q} \abs{I_{(p, q)}} < \epsilon$ to illustrate $P (A_\phi) = 0$.

    We can see
    \begin{align*}
         &\sum_{q=k}^\infty \sum_{p=1}^{q} \abs{I_{(p, q)}} \\
        =&\sum_{q=k}^\infty \sum_{p=1}^{q} \frac{2}{q^2\phi(q)} \\
        =&\sum_{q=k}^\infty \frac{2}{q\phi(q)} \\
        =&2\sum_{q=k}^\infty \frac{1}{q\phi(q)} \\
    \end{align*}

    Since $\sum_q \frac{1}{q\phi(q)} < \infty$, for each $\epsilon > 0$, there is a $k \geq 1$ satisfying $\sum_{q=k}^\infty \frac{1}{q\phi(q)} < \frac{\epsilon}{2}$. It implies $\sum_{q=k}^\infty \sum_{p=1}^{q} \abs{I_{(p, q)}} = 2\sum_{q=k}^\infty \frac{1}{q\phi(q)} < \epsilon$.

    Hence, $P(A_\phi) = 0$.
    
\end{proof}

\end{document}