\documentclass[a4paper, linespread=1.5]{article}
%\usepackage[UTF8]{ctex}
\usepackage{xeCJK}
\usepackage{geometry}
\usepackage{amsmath}
\usepackage{amssymb}
\usepackage{amsthm}
\usepackage{graphicx}
\usepackage{keyval}
\usepackage[dvipsnames,svgnames,x11names]{xcolor}
\usepackage{float}
\usepackage{ifthen}
\usepackage{calc}
\usepackage{ifplatform}
\usepackage{fancyvrb}
\usepackage{minted}
\usepackage{hyperref}
\usepackage{enumerate}
\usepackage{multicol}
\usepackage[all]{xy}
\usepackage{ulem}
\usepackage{epstopdf}
\usepackage{mathrsfs}
\usepackage{cancel}
\usepackage{algorithm}
\usepackage{algorithmic}
\setlength{\parskip}{0.2\baselineskip}
\setlength{\parindent}{2em}
%\geometry{left=2.7cm,right=2.7cm,top=2.7cm,bottom=2.7cm}


\newtheorem{theorem}{Theorem}
\newtheorem{proposition}[theorem]{Proposition}
\newtheorem{lemma}[theorem]{Lemma}
\newtheorem{corollary}[theorem]{Corollary}
\newtheorem{definition}[theorem]{Definition}
\newtheorem{exercise}[theorem]{Exercise}

\newtheorem{innercustom}{\customname}
\providecommand{\customname}{}
\newcommand{\newcustomtheorem}[2]{
    \newenvironment{#1}[1]
    {
        \renewcommand\customname{#2}
        \renewcommand\theinnercustom{##1}
        \innercustom
    }
    {\endinnercustom}
}
\newcustomtheorem{customthm}{Theorem}
\newcustomtheorem{customprop}{Proposition}
\newcustomtheorem{customlemma}{Lemma}
\newcustomtheorem{customcorollary}{Corollary}
\newcustomtheorem{customdef}{Definition}
\newcustomtheorem{customex}{Exercise}
\newcustomtheorem{customremark}{Remark}

\newcommand{\Natural}{\mathbb{N}}
\newcommand{\addbigcup}{\bigcup{\kern-1.12em{+}}\kern0.3em}
\newcommand{\nth}[1]{#1\textsuperscript{th}}

\begin{document}
    \title{Solution to Exercises of March 13}
    \author{赖睿航\ 518030910422}
    \date{\today}
    \maketitle
    
    \begin{customex}{6}[Union of countable infinite intervals]
        1) Let $(I_i)_{i = 0}^\infty$ be a sequence of intervals such that $\bigcup_{i \in \Natural} I_i \supseteq I_0$. Show that $$ \sum_{i \in \Natural} |I_i| \geqslant |I_0| $$.
        
        2) Use 1) to derive again Cantor's Theorem: $[0, 1]$ is uncountable.
    \end{customex}

    \begin{proof}
        1) Let $U_i := \bigcup_{j=1}^i I_j$ for $i = 1, 2, ...$ and let $U_0 = \emptyset$. By definition, we have
        \begin{align*}
            U_i &= \bigcup_{j = 1}^i I_j \\
            &= \addbigcup _{j = 1}^i (I_j \textbackslash (\bigcup_{k = 1}^{j - 1} I_k)) \\
            &= \addbigcup _{j = 1}^i (I_j \textbackslash U_{j - 1})
        \end{align*}
        where operator $\addbigcup$ gets the union of sets which are pairwisely disjoint. Since $\bigcup_{i \in \Natural} I_i \supseteq I_0$, we have
        \begin{align*}
            |I_0| &\leqslant |\bigcup_{i \in \Natural} I_i| \\
            &= | U_\infty | \\ 
            &= | \addbigcup_{i \in \Natural} (I_i \backslash U_{i - 1}) | \\
            &= \sum_{i = 1}^\infty | I_i \backslash U_{i - 1} |
        \end{align*}
        
        Obviously, $|U_i| \geqslant 0$. Therefore, 
        \begin{align*}
            |I_0| &\leqslant \sum_{i = 1}^\infty | I_i \backslash U_{i - 1} | \\
            &\leqslant \sum_{i = 1}^\infty |I_i|
        \end{align*}
        
        2) We prove by contradiction. 
        
        Assume $I_0 := [0, 1]$ is countable. Let $\{a_i\}$ be a sequence of reals in $[0, 1]$ where index $i$ starts from $1$. For each $a_i$, let $I_i := [a_i - \frac{1}{2^{i + 2}}, a_i + \frac{1}{2^{i + 2}}]$. Since for every real number $x \in [0, 1]$ there exists a subscript $k$ with $a_k = x$, we have $x \in I_k$. It follows that $\bigcup_{i = 1}^\infty I_i \supseteq I_0$. By 1) we know that $\sum_{i = 1}^\infty |I_i| \geqslant |I_0|$.
        
        However, by definition $|I_i| = (a_i + \frac{1}{2^{i + 2}}) - (a_i - \frac{1}{2^{i + 2}}) = \frac{1}{2^{i + 1}}$. Sum up all $|I_i|$, and we get $$\sum_{i = 1}^\infty |I_i| = \sum_{i = 1}^\infty \frac{1}{2^{i + 1}} = \frac{1}{2} < 1 = |I_0|$$. This leads to a contradiction.
        
        Therefore, $[0, 1]$ is uncountable.
    \end{proof}

    \begin{customex}{9}[Baby Vitali Lemma]
        Let $A = \{I_1, \ldots, I_n\}$ be a family of finite intervals in
        the real line. Show that there exists a set $B$ of disjoint intervals such that $B \subseteq A$ and $\ell(\bigcup_{i \in B} I) \geqslant \frac{1}{4} \ell(\bigcup_{i \in A} I)$, where $\ell$ is the usual length function.
    \end{customex}

    \begin{proof}
        I write down the proof after referring to some related materials.
        
        Let $N := \{1, 2, \ldots, n\}$ be the set of indeces. We construct the set of pairwisely disjoint intervals $B$ each time by choosing an interval whose length is as large as possible. More precisely, let $B$ be a sequence of intervals which is $\emptyset$ in the beginning. In the $i$-th turn, let $k_i := \arg\max\limits_{j \in N} \{\ell(I_j) | I_j \bigcap I_{k_t} = \emptyset, t = 1, 2, \ldots, i - 1\}$(if there are multiple choices, choose any). And then we add $I_{k_i}$ to the set $B$. Repeat choosing intervals from $A$ until no such interval satisfy the condition. Let $m$ be the number of intervals we chose from $A$ and let $M := \{1, 2, \ldots, m\}$. Finally we get the set $B = \{I_{k_1}, I_{k_2}, \ldots, I_{k_m}\}$.
        
        By construction, it is obvious that the intervals in $B$ are pairwisely disjoint, i.e., for any pair $(i, j)$ with $1 \leqslant i, j \leqslant m, i \neq j$ we have $I_{k_i} \bigcap I_{k_j} = \emptyset$.
        
        Claim that $B$ is what we want, i.e., $\ell(\bigcup_{i \in B} I) \geqslant \frac{1}{4} \ell(\bigcup_{i \in A} I)$.
        
        Now we prove the claim.
        
        Note that for any $i \in N$, there exists a $j \in M$ with $I_i \bigcap I_{k_j} \neq \emptyset$. Otherwise the process of choosing intervals wouldn't stop, which means there are still one or more intervals which can be added to $B$.
        
        Furthermore, for any $i \in N$, there exists a $j \in M$ with $I_i \bigcap I_{k_j} \neq \emptyset$ and $\ell(I_i) \leqslant \ell(I_{k_j})$. If not(i.e., for every $j$ with $I_i \bigcap I_{k_j} \neq \emptyset$, $\ell(I_i) > \ell(I_{k_j})$ holds), $I_i$ should be added to $B$ before all $I_{k_j}$s which have nonempty intersection with $I_i$ were added to $B$ according to the construction.
        
        Now, for each $i \in N$, choose any $j \in M$ such that $I_i \bigcap I_{k_j} \neq \emptyset$ and $\ell(I_i) \leqslant \ell(I_{k_j})$. Suppose $I_{k_j} = [a_{k_j}, b_{k_j}]$. We expand the interval $I_{k_j}$ fourfold in length to $I'_{k_j} := [\frac{1}{2}(5a_{k_j} - 3b_{k_j}), \frac{1}{2}(5b_{k_j} - 3a_{k_j})]$. Since $I_i \bigcap I_{k_j} \neq \emptyset$ and $\ell(I_i) \leqslant \ell(I_{k_j})$, it is obvious that $I_i \subseteq I'_{k_j}$.
        
        Hence we have $(\bigcup_{I \in A} I) \subseteq \bigcup_{i = 1}^m I'_{k_i}$, and thus:
        \begin{align*}
            \ell(\bigcup_{I \in A} I) &\leqslant \ell(\bigcup_{i = 1}^m I'_{k_i}) \\
            &\leqslant \sum_{i = 1}^m \ell(I'_{k_i}) = 4 \sum_{i = 1}^m \ell(I_{k_i}) = 4 \ell(\bigcup_{i = 1}^m I_{k_i}) = 4\ell(\bigcup_{I \in B} I)
        \end{align*}
        
        Equivalently, we have $\ell(\bigcup_{i \in B} I) \geqslant \frac{1}{4} \ell(\bigcup_{i \in A} I)$.
    \end{proof}

    \paragraph*{Remark of Exercise 9.}
        The conclusion of this lemma could be stronger, since in the last but two paragraph of the proof above, if we expand $I_{k_j}$ threefold to $I''_{k_j} := [2a_{k_j} - b_{k_j}, 2b_{k_j} - a_{k_j}]$, $I''_{k_j}$ also covers $I_i$.
        
        So we can get a stronger version: there exists a set $B$ of disjoint intervals such that $B \subseteq A$ and $\ell(\bigcup_{i \in B} I) \geqslant \frac{1}{3} \ell(\bigcup_{i \in A} I)$.
\end{document}
