% This is a template for lecture notes.
\documentclass{article}
\usepackage[UTF8]{ctex}
%\usepackage[left=0.8in,right=0.8in,top=1in,bottom=1in]{geometry}
\usepackage{amssymb}
\usepackage{amsmath}
\usepackage{amsthm}
\usepackage{geometry}
\usepackage{booktabs}
\usepackage{bm}
\usepackage{tcolorbox}
\CTEXoptions[today=old]
%Some commonly used notations
%\geometry{a4paper,bottom = 3cm,left = 3cm, right = 3cm}

%for reference
\usepackage{hyperref}
\usepackage[capitalise]{cleveref}
\crefname{enumi}{}{}

\newtheorem{theorem}{Theorem}
\newtheorem{lemma}[theorem]{Lemma}
\newtheorem{proposition}[theorem]{Proposition}
\newtheorem{corollary}[theorem]{Corollary}
\newtheorem{fact}[theorem]{Fact}
\newtheorem{definition}[theorem]{Definition}
\newtheorem{remark}[theorem]{Remark}
\newtheorem{question}[theorem]{Question}
\newtheorem{answer}[theorem]{Answer}
\newtheorem{exercise}[theorem]{Exercise}
\newtheorem{example}[theorem]{Example}
%\newenvironment{proof}{\noindent \textbf{Proof:}}{$\Box$}
\newtheorem{observation}[theorem]{Observation}
\newtheorem{conclusion}[theorem]{Conclusion}
%to use newcommand for convenience
\newcommand\field{\mathbb{F}}
\newcommand\Real{\mathbb{R}}
\newcommand\Q{\mathbb{Q}}
\newcommand\Z{\mathbb{Z}}
\newcommand\complex{\mathbb{C}}


%this is how we define operators.
\DeclareMathOperator{\rank}{rank} % rank

\title{Proof of $|C|=|R|=|[0,1]|$}
\author{Guo Linsong}
\date{\today}

\begin{document}
    \maketitle

\begin{question}\label{main}
\textbf{Let $C$ be the Cantor set. Show that $|C| = |R| = |[0, 1]|$.}
\end{question}

\begin{definition}\label{cantor}
\textbf{Cantor Set.}
\end{definition}
Let $C_0 = [0, 1]$. For each positive integer $n$, let $C_n$ be obtained from $C_{n-1}$ by dividing each interval of $C_{n-1}$ into three intervals of equal length and then removing the middle open interval from each of the intervals from $C_{n-1}$. The Cantor set is defined to be $\bigcap_{n\geq0}C_n$.

\begin{fact}\label{fact3}
\textbf{The Cantor set can be represented as
$$C=[0,1]\setminus\bigcup_{n=0}^{\infty}\bigcup_{k=0}^{3^n-1}(\frac{3k+1}{3^{n+1}},\frac{3k+2}{3^{n+1}})$$}
\end{fact}

%Considering numbers in ternary notation will make the discussion easier.For example,$\frac{2}{3}$ can be written as $0.2_{3}$, $\frac{8}{9}$ can be written as $0.22_{3}$.

\begin{fact}\label{fact4}
\textbf{The numbers in $C$ have only 0s and 2s in their ternary(base $3$) representation.}
\end{fact}
 Considering some intervals in the form of $(\frac{3k+1}{3^{n+1}},\frac{3k+2}{3^{n+1}})$ are removed from $[0,1]$, \cref{fact4} maybe obvious.But there're some special numbers which don't seem to satisfy the fact, such as $\frac{1}{3}=0.1_{3}$ and $\frac{7}{9} = 0.21_{3}$.These numbers are the left endpoint of intervals  $(\frac{3k+1}{3^{n+1}},\frac{3k+2}{3^{n+1}})$.However, $\frac{1}{3}$ can be written as $0.0222222\cdots_{3}$.Similarly, $\frac{7}{9}$ can be written as $0.020222222\cdots_{3}$ and all the special numbers can be represented in this way.

\begin{lemma}\label{lemma5}
\textbf{There's a surjective mapping from $C$ to $[0,1]$.}
\end{lemma}
The mapping can be defined by taking the ternary numbers that consist of 0s and 2s, replacing all the 2s by 1s, and interpreting the sequence as a binary representation of a real number in $[0,1]$. In a formula,
$$f(\sum_{k\in\mathbb{N}^{+}}a_k3^{-k})=\sum_{k\in\mathbb{N}^{+}}\frac{a_k}{2}2^{-k} (a_k\in\{0,2\})$$

For example, $f(\frac{2}{9})=f(0.02_3)=0.01_2=\frac{1}{4}$.

As the set $\{ \sum_{k\in\mathbb{N}^{+}}\frac{a_k}{2}2^{-k} \}$ is actually $[0,1]$,$f$ is surjective.

\begin{lemma}\label{lemma6}
\textbf{$|C|=|[0,1]|$}
\end{lemma}
Identity mapping is an injective mapping from $C$ to $[0,1]$, so we have $|C|\leq|[0,1]|$.And according to \cref{lemma5}, $|C|\geq|[0,1]|$.Therefore, we can conclude that $|C|=|[0,1]|$.

\begin{lemma}\label{lemma7}
\textbf{$|(0,1)|=|\mathbb{R}|$}
\end{lemma}
We define a mapping $g$ from $(0,1)$ to $\mathbb{R}$:
$$g(x)=tan(\pi x - \frac{\pi}{2})$$

Obviously, $g$ is a bijection, which implies that $|(0,1)|=|\mathbb{R}|$.The conclusion maybe beautiful, but we expect to get $|[0,1]|=|\mathbb{R}|$. So I looked up some papers and found an amazing proposition $|(0,1)| = |[0,1)|$.

\begin{proposition}\label{pro9}
$|(0,1)| = |[0,1)|$
\end{proposition}

Let $b_n=\frac{1}{n+1}$ for $n\in\mathbb{N}^{+}$ and $B=\{b_n|n\in\mathbb{N}^{+}\}$.We define a mapping $h$ from $B$ to $B\bigcup\{0\}$:

\begin{equation}
h(x)=\left\{
\begin{array}{rcl}
0 & & x=b_1\\
b_{n-1} & & x=b_n(n\geq2)
\end{array} \right.
\end{equation}

Assume $h(b_i)=h(b_j)=y$.If $y=0$, then $b_i=b_j=b_1$.Otherwise $y=b_k$, then $b_i=b_j=b_{k+1}$.Hence $h$ is injective.For any $b_n$,$h(b_{n+1})=b_n$ and $h(b_1)=0$, so $h$ is surjective.Therefore,$h$ is bijective.Next we can define identify mapping on $(0,1)-B$(clearly a bijection).Therefore, we can conclude that $|(0,1)|=|[0,1)|$.

Similarly, we can prove that $|[0,1)|=|[0,1]|$. Therefore, we can conclude that
$$R=|(0,1)| = |[0,1)|=|[0,1]|$$

\begin{conclusion}
$|C|=|R|=|[0,1]|$
\end{conclusion}
We have proved that $|C|=|R|$ and $|R|=|[0,1]|$, so we can get the conclusion.

\section*{Reference}
\href{
https://www.math.ubc.ca/~gor/Math220_2016/cardinality_workshop.pdf}{https://www.math.ubc.ca/~gor/Math220\_2016/cardinality\_workshop.pdf}
\end{document}
