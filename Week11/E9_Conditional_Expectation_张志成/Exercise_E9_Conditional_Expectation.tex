\documentclass[UTF8, 12pt]{article}
\fontfamily{DENGL.TTF}
\usepackage[a4paper, total={6in, 8in}]{geometry}
\usepackage{xeCJK}
\usepackage{enumitem}
\usepackage{amsmath}
\usepackage{mathtools}
\usepackage{amssymb}
\usepackage{amsfonts}
\usepackage{mathrsfs}
\usepackage{XCharter}
\usepackage{fancyhdr}
\usepackage{eulervm}
\usepackage{graphicx}
\usepackage{mdframed}
\usepackage{ntheorem}

\pagestyle{fancy}

\newenvironment{proof}{\noindent\ignorespaces\textbf{Proof:}}{\hfill $\square$\par\noindent}
\newenvironment{solution}{\noindent\ignorespaces\textbf{Solution:}}{\hfill $\square$\par\noindent}

\newtheorem{claim}{Claim}
\newtheorem{problem}{Problem}
\newtheorem{lemma}{Lemma}
\newtheorem*{theorem*}{Theorem}
\newtheorem*{remark*}{Remark}

\title{E9. Conditional Expectation}
\author{张志成 518030910439}
\date{\today}

\begin{document}
    \maketitle

    \begin{problem}
        Prove that if $\mathcal{G}$ is a sub-$\sigma$-algebra of $\mathcal{F}$ and if $X \in \mathcal{L}^1(\Omega, \mathcal{F}, \mathbf{P})$ and if $Y \in \mathcal{L}^1(\Omega, \mathcal{G}, \mathbf{P})$ and 
        \begin{equation}
        E(X;G) = E(Y;G)
        \end{equation}
        for every $G$ in a $\pi$-system which contains $\Omega$ and generates $\mathcal{G}$, then $(1)$ holds for every $G$ in $\mathcal{G}$.
    \end{problem}

    \begin{solution}
        Denote the $\pi$-system as $I$. And WLOG, we assume that $X \geq 0$, which is neccesary in the construction of the measures.

        We consider two measures:
        \begin{align*}
            \mu_0: G&\mapsto E(X;G) \\
            \mu_1: G&\mapsto E(Y;G) .
        \end{align*}
        Then $\mu_0$ and $\mu_1$ agree on the $\pi$-system $I$, and therefore must agree on on the $\sigma$-algebra it generates, namely $\sigma(I) = \mathcal{G}$.
        
        Thus, for every $G \in \mathcal{G}$,
        $$
        E(X;G) = E(Y;G) ,
        $$
        which finishes the proof.
    \end{solution}

    \begin{problem}
        Suppose that $X,Y \in \mathcal{L}^1(\Omega, \mathcal{F}, \mathbf{P})$ and that
        \begin{align*}
            &E(X|Y) = Y,\quad a.s., \\
            &E(Y|X) = X,\quad a.s.
        \end{align*}
        Prove that $P(X = Y) = 1$.
    \end{problem}

    \begin{solution}
        We first consider $E(X-Y;Y\leq c) = E(X;Y\leq c) - E(Y;Y\leq c)$.

        Since $E(X|Y) = Y, a.s.$, for every $G \in \sigma(Y)$, 
        \begin{align*}
            E(X;G) = E(Y;G)
        \end{align*}
        and therefore
        \begin{align*} 
            &E(X;Y\leq c) = E(Y;Y\leq c) \\
            \Longrightarrow &E(X-Y; Y\leq c) = 0.
        \end{align*}

        Since $$E(X-Y; Y\leq c) = E(X-Y; X > c, Y \leq c) + E(X-Y; X \leq c, Y \leq c), $$
        and because $$E(X-Y; X > c, Y \leq c) \geq 0 ,$$
        we have $$ E(X-Y; X \leq c, Y \leq c) \leq 0 .$$
        Similarly, we have $$ E(Y-X; X \leq c, Y \leq c) \leq 0 .$$
        Thus, 
        \begin{align*} 
            &E(X-Y; X \leq c, Y \leq c) = E(Y-X; X \leq c, Y \leq c) = 0 \\
            \Longrightarrow \quad &E(X-Y; X > c, Y \leq c) = 0 = E((X-Y)I_{\{X > c, Y \leq c\}}) \\
            \Longrightarrow \quad &E(I_{\{X > c, Y \leq c\}}) = 0
        \end{align*}
        Since $X > Y \Longleftrightarrow \exists c\in\mathbb{Q}\ X > c, Y \leq c$, and by countably additivity of $P$, 
        $$ P(X > Y) = P(\bigcup_{c\in\mathbb{Q}} \{X > c\} \cap \{Y \leq c\})\leq \sum_{c\in\mathbb{Q}} P(X > c, Y \leq c) = \sum E(I_{\{X > c, Y \leq c\}}) = 0. $$
        Similarly $ P(X < Y) = 0$ and thus, $$ P(X = Y) = 1. $$
    \end{solution}
\end{document}