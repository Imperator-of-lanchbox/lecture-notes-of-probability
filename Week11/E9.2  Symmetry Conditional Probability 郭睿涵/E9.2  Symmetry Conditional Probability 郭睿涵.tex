\documentclass[UTF8]{ctexart}
\usepackage{amsmath}
\usepackage{amssymb}
\usepackage{amsthm}
\usepackage{graphicx}
\usepackage{bm}
\usepackage{CJK}
\usepackage{float}
\usepackage{mdframed}

\usepackage{indentfirst}
\setlength{\parindent}{2em}

\providecommand{\abs}[1]{\lvert#1\rvert}
\providecommand{\norm}[1]{\lVert#1\rVert}
\providecommand{\ud}[1]{\underline{#1}}

\newmdtheoremenv{thm}{Theorem}
\newmdtheoremenv{lemma}[thm]{Lemma}
\newmdtheoremenv{fact}[thm]{Fact}
\newmdtheoremenv{cor}[thm]{Corollary}
\newtheorem{eg}{Example}
\newtheorem{ex}{Exercise}
\newmdtheoremenv{defi}{Definition}
\newenvironment{sol}
  {\par\vspace{3mm}\noindent{\it Solution}.}
  {\qed \\ \medskip}

\newcommand{\ov}{\overline}
\newcommand{\ca}{{\cal A}}
\newcommand{\cb}{{\cal B}}
\newcommand{\cc}{{\cal C}}
\newcommand{\cd}{{\cal D}}
\newcommand{\ce}{{\cal E}}
\newcommand{\cf}{{\cal F}}
\newcommand{\ch}{{\cal H}}
\newcommand{\cl}{{\cal L}}
\newcommand{\cm}{{\cal M}}
\newcommand{\cp}{{\cal P}}
\newcommand{\cs}{{\cal S}}
\newcommand{\cz}{{\cal Z}}
\newcommand{\eps}{\varepsilon}
\newcommand{\ra}{\rightarrow}
\newcommand{\la}{\leftarrow}
\newcommand{\Ra}{\Rightarrow}
\newcommand{\dist}{\mbox{\rm dist}}
\newcommand{\bn}{{\mathbb N}}
\newcommand{\bz}{{\mathbb Z}}

\newcommand{\expe}{{\mathsf E}}
\newcommand{\pr}{{\mathsf{Pr}}}
\usepackage{amsthm,amsmath,amssymb}

\usepackage{mathrsfs}

\setlength{\parindent}{0pt}
%\setlength{\parskip}{2ex}
\newenvironment{proofof}[1]{\bigskip\noindent{\itshape #1. }}{\hfill$\Box$\medskip}
\usepackage{amsthm,amsmath,amssymb}

\theoremstyle{definition}
\newtheorem{problem}{Problem}
\newtheorem*{problem*}{Problem}

\pagenumbering{gobble}

\begin{document}

\title{E9.2 Symmetry Conditional Probability}
\date{May. 22, 2020}

\maketitle
\paragraph{E9.2 } Suppose $X,Y \in \mathcal{L}^1(\Omega,\mathcal{F},P)$, and $E(X|Y) = Y,a.s.,$  $E(Y|X) = X,a.x..$ To proof $P(X=Y)=1$
\paragraph{Solution}
As the hint said,
\begin{align*}
	E((X-Y)I_{Y\leq c}) = E((X-Y)I_{Y\leq c}I_{X\leq c}) + E((X-Y)I_{Y\leq c}I_{X>c})\\
	E((X-Y)I_{X\leq c}) = E((X-Y)I_{Y\leq c}I_{X\leq c}) + E((X-Y)I_{X\leq c}I_{Y>c})
\end{align*}
At the same time,
\begin{align*}
	E((X-Y)I_{X\leq c}) &= E((X-E(X|Y))I_{X\leq c})\\&= E(XI_{X\leq c}) - E((X|Y)I_{X\leq c})\\&= E(XI_{X\leq c}) - E(XI_{X\leq c}) = 0
\end{align*}
So that

\begin{align*}
	 E((X-Y)I_{Y\leq c}I_{X>c}) = E((X-Y)I_{X\leq c}I_{Y>c})
\end{align*}
Because the left side is non-negative and the right side is non-positive, both of them equal to 0. Then we can present the following events:
\begin{align*}
	(X<Y) = \bigcup_{c\in \mathbb{Q}}(X\leq c)\cap (Y > c)\\
	(X>Y) = \bigcup_{c\in \mathbb{Q}}(X > c)\cap (Y \leq c)
\end{align*}
They are all null events, which leads to the event$(X\neq Y)$ is a null event. Then we got $P(X = Y) = 1$.
\subparagraph{Question} Is the union of infinite null-set is also a null-set?
\paragraph{Another way to proof} 
\begin{align*}
	E(X^2) = E(X E(Y|X)) = E(E(XY|X)) = E(XY)\\
	E(Y^2) = E(Y E(X|Y)) = E(E(XY|Y)) = E(XY)\\
\end{align*}
So that
\begin{align*}
	E((X-Y)^2) = E(X^2 + Y^2 - 2XY) = 0
\end{align*}
Which means X and Y is different on a null-set.
\end{document}

