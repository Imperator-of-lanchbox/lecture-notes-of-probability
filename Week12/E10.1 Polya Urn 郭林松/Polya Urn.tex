\documentclass{article}
\usepackage[utf8]{inputenc}
\usepackage{amssymb}
\usepackage{amsmath}
\usepackage{amsthm}
\usepackage{geometry}
\usepackage{booktabs}
\usepackage{bm}
\usepackage{tcolorbox}
\usepackage{setspace}

\renewcommand{\baselinestretch}{1.3}

\newtheorem{theorem}{Theorem}
\newtheorem{lemma}[theorem]{Lemma}
\newtheorem{claim}[theorem]{Claim}
\newtheorem{proposition}[theorem]{Proposition}
\newtheorem{corollary}[theorem]{Corollary}
\newtheorem{fact}[theorem]{Fact}
\newtheorem{definition}[theorem]{Definition}
\newtheorem{remark}[theorem]{Remark}
\newtheorem{question}[theorem]{Question}
\newtheorem{answer}[theorem]{Answer}
\newtheorem{exercise}[theorem]{Exercise}
\newtheorem{example}[theorem]{Example}
\newtheorem{observation}[theorem]{Observation}
%\newtheorem*{solution}{Solution}

\title{A Discussion About Pólya Urn}
\author{Guo Linsong 518030910419}
\date{\today}

\begin{document}

\maketitle


\begin{tcolorbox}
    \begin{problem}
    The following exercise is E10.1 of the textbook.

      At time $0,$ an urn contains 1 black ball and 1 white ball. At each time $1,2,3, \ldots,$ a ball is chosen at random from the urn and is replaced together with a new ball of the same colour. Just after time $n,$ there are therefore $n+2$ balls in the urn, of which $B_{n}+1$ are black, where $B_{n}$ is the number of black balls chosen by time $n$.  Let $M_{n}=\left(B_{n}+1\right) /(n+2),$ the proportion of black balls in the urn just after time $n .$
   \end{problem}
\end{tcolorbox}

\begin{question}
Prove that (relative to a natural filtration which you should specify $) M$ is a martingale.
\end{question}

\begin{solution}
Let $\mathcal{F}_{n}:=\sigma(B_i,0\le i \le n)$ be the natural filtration, then $M=(M_n:n\ge 0)$ is adapted and integrable. Next we prove that $\mathbb{E}[M_n|\mathcal{F}_{n-1}]=M_{n-1},\text{a.s.}(n\ge 1)$.
\begin{equation*}
    \begin{array}{rl}
        & \mathbb{E}\left[M_n|\mathcal{F}_{n-1}\right] \\
     =  & \mathbb{E}\left[\frac{B_n+1}{n+2}\big|\mathcal{F}_{n-1}\right] \\
     =  & \frac{1}{n+2}\mathbb{E}\left[B_{n}+1\big|\mathcal{F}_{n-1}\right] \\
     =  & \frac{1}{n+2}\mathbb{E}\left[B_{n-1}+\frac{B_{n-1}+1}{n+1}+1\big|\mathcal{F}_{n-1}\right] \\
     =  & \mathbb{E}\left[\frac{B_{n-1}+1}{(n-1)+2}\big|\mathcal{F}_{n-1}\right] \\
     = & M_{n-1}
    \end{array}
\end{equation*}
Therefore, $M$ is a martingale relative to $\{\{\mathcal{F}_n\},\mathbf{P}\}$.
\end{solution}


\begin{question}
Prove that $\mathbf{P}\left(B_{n}=k\right)=(n+1)^{-1}$  for $0 \leq k \leq n$.
\end{question}

\begin{solution}

\begin{equation*}
    \begin{array}{cc}
        & \mathbf{P}(B_n=k)\\
        = &{n \choose k} \frac{(1\times\cdots\times k)\times(1\times\cdots\times(n-k))}{2\times3\times\cdots\times(n+1)}\\
        = &\frac{n!}{k!(n-k)!} \cdot\frac{k!(n-k)!}{(n+1)!}\\
        = & \frac{1}{n+1}\\
    \end{array}
\end{equation*}
\end{solution}

\begin{question}
What is the distribution of $\Theta$, where
$\Theta:=\lim M_{n} ?$
\end{question}

\begin{solution}
We have known that $P(B_n=k)=\frac{1}{n+1}$, so $M_n$ is uniform in $\{\frac{1}{n+2},\frac{2}{n+2},\cdots,\frac{n+2}{n+2}\}$. Thus we have
\begin{equation*}
    \begin{array}{rl}
         & P(\Theta\le t) (t\in[0,1])\\
       = & P(\lim_{n\rightarrow\infty}M_n \le t) \\
       = & \lim_{n\rightarrow\infty}P(M_n \le t) \\
     %  = & \lim_{n\rightarrow\infty}P(\frac{B_n+1}{n+2} \le t) \\
      % = & \lim_{n\rightarrow\infty}P(B_n \le t(n+2)-1) \\
       = & t
    \end{array}
\end{equation*}
Therefore, $\Theta$ satisfies the uniform distribution.
\end{solution}

\end{document}
