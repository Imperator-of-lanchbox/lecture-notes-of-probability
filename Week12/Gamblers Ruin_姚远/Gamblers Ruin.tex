% This is a template for lecture notes.
\documentclass{article}
\usepackage[UTF8]{ctex}
\usepackage{amssymb}
\usepackage{amsmath}
\usepackage{amsthm}
\usepackage{geometry}
\usepackage{booktabs}
\usepackage{bm}
\usepackage{tcolorbox}
\CTEXoptions[today=old]
%Some commonly used notations
%\geometry{a4paper,bottom = 3cm,left = 3cm, right = 3cm}

%for reference
\usepackage{hyperref}
\usepackage[capitalise]{cleveref}
\crefname{enumi}{}{}

\newtheorem{theorem}{Theorem}
\newtheorem{lemma}[theorem]{Lemma}
\newtheorem{proposition}[theorem]{Proposition}
\newtheorem{corollary}[theorem]{Corollary}
\newtheorem{fact}[theorem]{Fact}
\newtheorem{definition}[theorem]{Definition}
\newtheorem{remark}[theorem]{Remark}
\newtheorem{question}[theorem]{Question}
\newtheorem{answer}[theorem]{Answer}
\newtheorem{exercise}[theorem]{Exercise}
\newtheorem{example}[theorem]{Example}
%\newenvironment{proof}{\noindent \textbf{Proof:}}{$\Box$}
\newtheorem{observation}[theorem]{Observation}

%to use newcommand for convenience
\newcommand\field{\mathbb{F}}
\newcommand\Real{\mathbb{R}}
\newcommand\Q{\mathbb{Q}}
\newcommand\Z{\mathbb{Z}}
\newcommand\complex{\mathbb{C}}

%this is how we define operators.
\DeclareMathOperator{\rank}{rank} % rank

\title{Gamblers Ruin}
\author{Yao Yuan}
\date{\today}

\begin{document}
\maketitle

\begin{tcolorbox}
    \begin{exercise} 
        Gambler's Ruin.
        Suppose that $X_1, X_2, ...$ are IID RVs with $P[X = +1] = p$, $P[X = -1] = q$, where $0 < p = 1 - q < 1$, and $p \neq q$.
        Suppose that $a$ and $b$ are integers with $0 < a < b$.
        Define $S_n = a + X_1 + X_2 + ... + X_n$, $T = \text{inf}~\{n \mid S_n = 0 ~\text{or}~ S_n = b\}$.

        Deduce the values of $P(S_T = 0)$ and $E(S_T)$.
    \end{exercise} 
\end{tcolorbox}

This is an interesting problem about a gambler who starts with an initial fortune of $\$ a$ and then on each successive gamble either wins $\$ 1$ or loses $\$ 1$ independent of the past. The gambler's goal is to get $\$ b$. 

We denote $P_i = P(S_T = b)$ to be the probability that the gambler wins when $a = i$. Clearly $P_{0}=0$ and $P_{b}=1$ by definition, and we next proceed to compute $P_{i}, 1 \leq i \leq b-1$

We now focus on a certain gamble our gambler makes when the gambler has money $i$. If $X = 1$, then the gambler will own one more doller, so by Markov property, the gambler now can win with probability $P_{i + 1}$. Similarly, if $X = -1$, the gambler can win with probability $P_{i - 1}$. Thus we can get the following recursion.
\[
    P_{i}=p P_{i+1}+q P_{i-1}
\]
The recursion can be rewritten as $p P_{i}+q P_{i}=p P_{i+1}+q P_{i-1},$ thus we can get
\[
    P_{i+1}-P_{i}=\frac{q}{p}\left(P_{i}-P_{i-1}\right)
\]
And then we can easily derive
\[
    P_{i+1}-P_{i} = (\frac{q}{p})^i (P_1 - P_0), 0 < i < b
\]
Thus
\[
    \begin{aligned}
    P_{i+1} & = P_{i+1} - P_0 \\
            & = \sum_{k = 0}^{i}\left(P_{k+1}-P_{k}\right) \\
            & = \sum_{k = 0}^{i}\left(\frac{q}{p}\right)^{k} (P_1 - P_0) \\
            & = \sum_{k = 0}^{i}\left(\frac{q}{p}\right)^{k} P_1 \\
            & = P_1 \frac{1 - (\frac{q}{p}) ^ {i + 1}} {1 - (\frac{q}{p})} 
    \end{aligned}
\]
Now we let $i = b - 1$, then we have
\[
    1 = P_b = P_1 \frac{1 - (\frac{q}{p}) ^ {b}} {1 - (\frac{q}{p})} 
\]
Thus
\[
    P_1 = \frac{1 - \frac{q}{p}} {1 - (\frac{q}{p}) ^ b}
\]
And that leads us to
\[
    P_i = \frac{1 - (\frac{q}{p}) ^ i} {1 - (\frac{q}{p}) ^ b}
\]
Thus
\[
    P(S_T = 0) = 1 - P_a = \frac{(\frac{q}{p}) ^ a - (\frac{q}{p}) ^ b} {1 - (\frac{q}{p}) ^ b}
\]

\[
    E(S_T) = b~\frac{1 - (\frac{q}{p}) ^ i} {1 - (\frac{q}{p}) ^ b}
\]
\\
Reference: http://www.columbia.edu/~ks20/FE-Notes/4700-07-Notes-GR.pdf
\end{document}