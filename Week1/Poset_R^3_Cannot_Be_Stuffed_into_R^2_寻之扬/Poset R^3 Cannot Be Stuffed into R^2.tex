\documentclass[UTF8]{ctexart}
\usepackage{amsmath}
\usepackage{amssymb}
\usepackage{amsthm}
\usepackage{graphicx}
\usepackage{CJK}
\usepackage{float}
\usepackage{mdframed}
\usepackage[a4paper, total={5.5in, 9in}]{geometry}

\renewcommand*{\figurename}{Figure} 

\providecommand{\abs}[1]{\lvert#1\rvert}
\providecommand{\norm}[1]{\lVert#1\rVert}
\providecommand{\ud}[1]{\underline{#1}}

\newmdtheoremenv{thm}{Theorem}
\newmdtheoremenv{lemma}[thm]{Lemma}
\newmdtheoremenv{fact}[thm]{Fact}
\newmdtheoremenv{cor}[thm]{Corollary}
\newmdtheoremenv{pro}[thm]{Problem}
\newtheorem{eg}{Example}
\newtheorem{ex}{Exercise}
\newmdtheoremenv{defi}{Definition}
\newenvironment{sol}
  {\par\vspace{3mm}\noindent{\it Solution}.}
  {\qed \\ \medskip}

\newcommand{\ov}{\overline}
\newcommand{\ca}{{\cal A}}
\newcommand{\cb}{{\cal B}}
\newcommand{\cc}{{\cal C}}
\newcommand{\cd}{{\cal D}}
\newcommand{\ce}{{\cal E}}
\newcommand{\cf}{{\cal F}}
\newcommand{\ch}{{\cal H}}
\newcommand{\cl}{{\cal L}}
\newcommand{\cm}{{\cal M}}
\newcommand{\cp}{{\cal P}}
\newcommand{\cs}{{\cal S}}
\newcommand{\cz}{{\cal Z}}
\newcommand{\eps}{\varepsilon}
\newcommand{\ra}{\rightarrow}
\newcommand{\la}{\leftarrow}
\newcommand{\Ra}{\Rightarrow}
\newcommand{\dist}{\mbox{\rm dist}}
\newcommand{\bn}{{\mathbb N}}
\newcommand{\bz}{{\mathbb Z}}

\newcommand{\expe}{{\mathsf E}}
\newcommand{\pr}{{\mathsf{Pr}}}

\newcommand{\floor}[1]{\lfloor#1\rfloor}


\setlength{\parindent}{0pt}
\setlength{\parskip}{1ex}
\newenvironment{proofof}[1]{\bigskip\noindent{\itshape #1. }}{\hfill$\Box$\medskip}

\theoremstyle{definition}
\newtheorem{problem}{Problem}
\newtheorem*{problem*}{Problem}

\pagenumbering{gobble}

\CTEXoptions[today=old]
\title{Poset $\mathbb{R}^3$ Cannot Be Stuffed into $\mathbb{R}^2$}
\author{Xun Zhiyang}

\begin{document}

\maketitle

\begin{pro}
    View $\mathbb{R}^{k}$ as a poset which $\left( x_1,x_2,\ldots x_{k}\right) \le 
    \left( y_1,y_2,\ldots,y_{k} \right) $ if and only if
    $y-x\ge 0$ namely $\forall i,y_i \ge  x_i$. Is there
    a subposet of $\mathbb{R}^{2}$ poset which is isomorphic with $\mathbb{R}^{3}$?
\end{pro}

The answer is no.

\section{A Naïve Discomfort}

When we look at a cube drawn on a plane, we always feel there is something wrong. In Figure 1, we can see $A$ and $B$ are not comparable in $\mathbb{R}^3$ poset. However, as we draw them on a plane, it \textbf{seems} as if they were perfectly comparable. We'll prove this discomfort actually makes sense.

\begin{figure}[ht]
        \centering
        \includegraphics[width=0.5\textwidth]{newcube.png}
        \caption{Something goes wrong}
\end{figure}

\section{Proof}

The proof is actually a formalization of the discomfort above.

Assume there is an isomorphism between poset $\mathbb{R}^3$ and a subposet of $\mathbb{R}^2$. We denote this bijection as $\varphi$. Now we introduce the following notations:
\begin{align*}
    (1, 0, 0) = a &\stackrel{\varphi}{\longrightarrow} a' = (x_a, y_a) \\
    (0, 1, 1) = b &\stackrel{\varphi}{\longrightarrow} b' = (x_b, y_b) \\
    (0, 0, 1) = c &\stackrel{\varphi}{\longrightarrow} c' = (x_c, y_c) \\
    (1, 1, 0) = d &\stackrel{\varphi}{\longrightarrow} d' = (x_d, y_d) \\
    (1, 0, 1) = e &\stackrel{\varphi}{\longrightarrow} e' = (x_e, y_e) \\
    (0, 1, 0) = f &\stackrel{\varphi}{\longrightarrow} f' = (x_f, y_f)
\end{align*}

Without loss of generality, we assume $x_a \geq x_b$ (the proof of the $x_a \leq x_b$ case is almost the same). Because $a$ and $b$ are not comparable, $a'$ and $b'$ are not as well. 
Thus we know $y_a \leq y_b$.

From $c \leq b$, we can see $x_c \leq x_b$. Similarly, $x_a \leq x_d$. Thus, $x_c \leq x_b \leq x_a \leq x_d$. Since $c'$ and $d'$ are not comparable, $y_c \geq y_d$.


From $e \geq a$, we know $e' \geq a'$. Thus $x_e \geq x_a \geq x_b$.
Also, $e \geq c$ and we can conclude $y_e \geq y_c \geq y_d$.
In addition, $f \leq b$ indicates $x_f \leq x_b$ and $f \leq d$ leads to $y_f \leq y_d$.

Here is the discomfort. We've proved $x_f \leq x_b \leq x_e$ and $y_f \leq y_d \leq y_e$. It immediately leads to $f' \leq e'$! However, we know $e$ and $f$ are not comparable. Our assumption that such bijection exists leads to a contradiction.

Hence, a subposet of $\mathbb{R}^2$ cannot be isomorphic with $\mathbb{R}^3$.



\begin{figure}[ht]
    \centering
    \includegraphics[width=0.5\textwidth]{labeled.png}
    \caption{Some points on a cube}
\end{figure}

% The $x_a \leq x_b$ case is very similar,



\end{document}

 