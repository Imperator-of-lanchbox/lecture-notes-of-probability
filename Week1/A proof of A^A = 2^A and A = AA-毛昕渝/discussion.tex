% This is a template for lecture notes.
\documentclass[12pt]{article}
\usepackage{amssymb}
\usepackage[UTF8]{ctex}
\usepackage{amsmath}
\usepackage{amsthm}
\usepackage{geometry}
\usepackage{booktabs}
\usepackage{bm}
\usepackage{cite}
%\usepackage{CJK}
\usepackage[many]{tcolorbox}
%\tcbuselibrary{listingsutf8}
%\tcbuselibrary{skins, breakable, theorems, most}
%\geometry{a4paper,bottom = 3cm,left = 3cm, right = 3cm}
\CTEXoptions[today=old]
%for reference
\usepackage{hyperref}
\usepackage[capitalise]{cleveref}
\crefname{enumi}{}{}


\newtheoremstyle{mythm}{1.5ex plus 1ex minus .2ex}{1.5ex plus 1ex minus .2ex} 
    {}{\parindent}{\bfseries}{}{1em}{} 
\theoremstyle{mythm}
\newtheorem{theorem}{Theorem}
\newtheorem{lemma}[theorem]{Lemma}
\newtheorem{proposition}[theorem]{Proposition}
\newtheorem{fact}[theorem]{Fact}
\newtheorem{definition}[theorem]{Definition}
\newtheorem*{remark}{Remark}

%\newenvironment{proof}{\noindent \textbf{Proof:}}{$\Box$}

%to use newcommand for convenience
\newcommand\field{\mathbb{F}}
\newcommand\Real{\mathbb{R}}
\newcommand\Q{\mathbb{Q}}
\newcommand\Z{\mathbb{Z}}
\newcommand\complex{\mathbb{C}}
\newcommand\cc{\mathcal{C}}
\newcommand\uu{\mathcal{U}}
\newcommand\pp{\mathcal{P}}
\newcommand\ff{\mathcal{F}}
\renewcommand\refname{Reference}

\DeclareMathOperator{\range}{range}   

\title{A proof of $|A^A| = |2^A| $ using Zorn's Lemma}
\author{Xinyu Mao}
\date{\today}
\begin{document}
\maketitle

\begin{tcolorbox}
\begin{theorem} \label{main}
    Let $A$ be an infinite set, then $|A^A| = |2^A|$.
\end{theorem}
\end{tcolorbox}
To validate \cref{main}, the following theorem is of great importance.

\begin{tcolorbox}
\begin{theorem} \label{helper}
    Let $A$ be an infinite set, then $|A \times A| = A$. 
\end{theorem}  
\end{tcolorbox}
\begin{proof}[Proof]   
    Let 
    $$
    \ff = \{f \in (X \times X)^X : X \subseteq A, f \text{ is a bijection}\}.
    $$
    Note that a function from $A$ to $B$ can be viewed as 
    a subset of $A \times B$, and thus $\ff \subset \pp(A \times A \times A)$,
    where $\pp(\cdot)$ is the power set of some set. 
    In the following argument, for function $f$, $\range(f) := \{y:\exists x f(x) = y\}$
    
    Now we check that the poset $(\ff, \subset)$ satisfies
    the condition of \textit{Zorn's lemma}. 
    First, $\ff \neq \emptyset$ since $\emptyset \in \ff$,
    and it remains to show that \textit{all chains in $\ff$ are closed}.
    Let $\cc$ be a chain in $\ff$. 
    Clearly, $\phi := \bigcup_{f \in \cc} f$ is also a
    function. Since every $f \in \cc$ is bijection by definition, 
    say $Y := \range(\phi) \subseteq A$, $\phi$ is a bijection from $Y \times Y$ to $Y$,
    and thus $\phi \in \ff$.
    In another word, $\phi$ is a upper bound of $\cc$ in $\ff$. 

    By \textit{Zorn's lemma} ,there is a maximal element in $\ff$,
    which is denoted by $\psi$. Let $U = \range(\psi)$, we shall
    show that $A\setminus U$ is finite. Assume that $A\setminus U$
    is infinite, then there is a countable subset of $A\setminus U$,
    say $V$. We know that there is a bijection $\sigma : V \times V \to V$.
    By definition, $\sigma \in \ff$. 
    Note that the domains of $\psi$ and $\sigma$ have no intersection, 
    and hence $\psi \subset \psi \cup \sigma \in \ff$, 
    which is in contradiction with the maximality of $\psi$.
    Since $\psi$ is a bijection from $U \times U \to U$,
    we conclude that $|A \times A| = |U \times U| = |U| = |A|$.
\end{proof}     

\begin{proof}[Proof of \cref{main}]
By \cref{helper} we establish $|A \times A| = |A|$,
and hence 
$$
|A^A| \leq |\pp(A \times A)| = |\pp(A)| = |2^A|.
$$
One the other hand, choose $a,b \in A$ arbitrarily, then a injection 
from $2^A$ to $A^A$ is given by 
$$
\phi : 2^A \to A^A, f \mapsto f' \text{ where }
f'(x) = \begin{cases}
    a, \text{ if $f(x) = 0$,} \\
    b, \text{ if $f(x) = 1$.}
\end{cases}
$$
Hence, $|2^A| \leq |A^A|$, completing the proof.
\end{proof}

\begin{remark}
    The main idea of the first proof comes from \cite{Dong88}.
    I love this proof for it only uses \textit{Zorn's lemma}
    and the basic conception of set. 
    Other proofs of \cref{helper} is based on the rigorous definition of 
    \textit{ordinal} and \textit{cardinality}, such 
    as the one in \cite{Li19}.
     
\end{remark}
\bibliographystyle{alphaurl}
\bibliography{ref}
\end{document}