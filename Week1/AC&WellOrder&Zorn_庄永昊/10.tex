% This is a template for lecture notes.
\documentclass{article}
\usepackage[UTF8]{ctex}
\usepackage{amssymb}
\usepackage{amsmath}
\usepackage{amsthm}
\usepackage{geometry}
\usepackage{booktabs}
\usepackage{bm}
\usepackage{tcolorbox}
\usepackage{xunicode, mathrsfs, xltxtra, amsfonts, caption, latexsym}
\CTEXoptions[today=old]
%Some commonly used notations
%\geometry{a4paper,bottom = 3cm,left = 3cm, right = 3cm}

%for reference
\usepackage{hyperref}
\usepackage[capitalise]{cleveref}
\crefname{enumi}{}{}

\newtheorem{theorem}{Theorem}
\newtheorem{lemma}[theorem]{Lemma}
\newtheorem{proposition}[theorem]{Proposition}
\newtheorem{corollary}[theorem]{Corollary}
\newtheorem{fact}[theorem]{Fact}
\newtheorem{definition}[theorem]{Definition}
\newtheorem{remark}[theorem]{Remark}
\newtheorem{question}[theorem]{Question}
\newtheorem{answer}[theorem]{Answer}
\newtheorem{exercise}[theorem]{Exercise}
\newtheorem{example}[theorem]{Example}
%\newenvironment{proof}{\noindent \textbf{Proof:}}{$\Box$}
\newtheorem{observation}[theorem]{Observation}

%this is how we define operators.
\DeclareMathOperator{\rank}{rank} % rank

\newenvironment{myproof}{\ignorespaces\paragraph{Proof:}}{\hfill $\square$\par\noindent}

\title{Probability, Week 1, exerciese 10}
\author{庄永昊}
\date{\today}
\def\mfa{\mathfrak A}

\begin{document}
\maketitle

1. The proof of AC $\leftrightarrow$ Well-order is easy: 
\begin{myproof}
The well-ordering principle can be constructed by choosing an element $x_S$ from the set $S$, 
then choolse $x_{S'}$ from $S'=S/\{x_S\}$...In this way, $x_S\leq x_{S'}\leq x_{S''}...$ is a well-order.
(the check is too trivial to write, maybe someone can help me do this?) 

And if the well-ordering principle is correct, in every set, simply choosing the minimum element can make AC proved. 
\end{myproof}

2. Here is the proof of Well-Ordering principle $\rightarrow$ Zorn's Lemma. It uses the General Principle of Recursive Definition, 
which is also proved below. 

\begin{myproof}
    Denote $S_x$ be the set of all elements smaller than $x$ with respect to a well order of the set. 

    Claim that the General Principle of Recursive Definition is correct.(which will be proved at the end)

    Since in every set there is a well order, construct a well order $\leq$ in $\mathscr U$. 
    Denote $S_\alpha=\{\beta|\beta\leq\alpha\}$, for every $\prec$
    define a function $h:\mathscr U\rightarrow Pow(\mathscr U)$ that(By the General Principle of Recursive Definition): 
    \begin{equation}
        h(\alpha)=\left\{
        \begin{aligned}
            &\alpha&\text{if $S_\alpha=\emptyset$}\\
            &h(S_\alpha)&\text{if for some $\beta$ in $S_\alpha$, there is neither $\alpha\prec\beta$ nor $\beta\prec\alpha$}\\
            &h(S_\alpha)\cup\alpha&\text{otherwise}
        \end{aligned}
        \right.
    \end{equation}
    Now denote $$C=\bigcup_{\alpha\in\mathscr U}h(\alpha)$$, 
    it is obvious that $C$ is a chain with respect to $\mathscr U$ and for every element in $\mathscr U$, 
    it is either a member(named $\gamma$) of $C$ or an element that fits: 
    $$\exists \alpha\in C, (\neg\alpha\prec\gamma)\cap(\neg\gamma\prec\alpha)$$
    Now denote $\prec$ with $\subset$, it is obvious that 
    $$\forall \alpha\in C, \alpha\prec\bigcup C$$
    If there is an element $\gamma\in\mathscr U-C$ that $\bigcup C\prec\gamma$, 
    then $\forall\alpha\in C, \alpha\prec\gamma$, hence $\gamma\in C$, which is contradicted with $\gamma\in\mathscr U-C$.

    So there is no element $\gamma$ that $\bigcup C\prec\gamma$, that is, $\bigcup C$ is a maximal element in $\mathscr U$.
\end{myproof}
    Now prove the General Principle of Recursive Definition:

    First prove the principle of transfinite induction that: if $J$ is a well-order set and $J_0$ is an inductive subset of $J$, there is $J_0=J$.

    This principle is proved by contradiction. Assume the set $J'=J-J_0$ is not empty, since $J$ is well-ordered, 
    there is a minimum element in $J'$(denote the element as $x$), as $x$ is the minimum one, $S_x\subset J_0$. 
    By the definition of the inductive set(if $S_x\subset J_0$, then $x\in J_0$), $x$ should be in $J_0$, 
    which is contradicted with $x\in J'$.

    Now give the detailed definition of the General Principle of Recursive Definition:

    Assume $J$ is a well-ordered set and $C$ is a set, 
    $\mathscr F$ is the set of all functions from $J$ or $S_\alpha(\alpha\in J)$ to $C$, 
    then for an unique function $\rho:\mathscr F\rightarrow C$, there is an unique $h:J\rightarrow C$, 
    which satisfies that $\forall \alpha\in J, h(\alpha)=\rho(h_{S\alpha})$. 

    Now prove the principle:

    1)if $k$ and $g$ are in $dom(\rho)$, then for all $x\in dom(k)\cap dom(g)$, there is $k(x)=g(x)$. 
    Otherwise, we can find a smallest $x$ by which $k(x)\neg g(x)$, and we can prove such $x$ actually satisfy $k(x)=g(x)$;

    2)if there is $k:S_\alpha\rightarrow C$ which is in $dom(\rho)$, then $k':S_\alpha\cup\{\alpha\}\rightarrow C$ is in $dom(\rho)$, 
    which can be defined by extending the $k$.(adding $\alpha$ by definition);

    3)if for all $\alpha\in K\subset J$ there is $h_\alpha:S_\alpha\rightarrow C$ in $dom(\rho)$, then there is: 
    $$k:\bigcup_{\alpha\in K}S_\alpha\rightarrow C, k\in dom(\rho)$$
    This can be proved by using 1) to prove that when the domains of two function overlap, choosing any one is acceptable; 

    4)by the principle of the transfinite induction, there is:
    $$\forall\beta\in J\text{,there is }h_\beta:S_\beta\rightarrow C, h_\beta\in dom(\rho)$$
    This can be proved by selecting a minimum element not satisfying it, then using 2) and 3) to prove that it actually satisfies it. 

    5)then the General Principle of Recursive Definition is proved by the set of all elements do not satisfy it is empty. (still choose a minimum to prove it)
\end{document}