% This is a template for lecture notes.
\documentclass{article}
% \usepackage[UTF8]{ctex}
\usepackage{amssymb}
\usepackage{amsmath}
\usepackage{amsthm}
\usepackage{geometry}
\usepackage{booktabs}
\usepackage{bm}
\usepackage{tcolorbox}
\usepackage{graphicx}
% \CTEXoptions[today=old]
%Some commonly used notations
%\geometry{a4paper,bottom = 3cm,left = 3cm, right = 3cm}

%for reference
\usepackage{hyperref}
\usepackage[capitalise]{cleveref}
\crefname{enumi}{}{}

\newtheorem{theorem}{Theorem}
\newtheorem{lemma}[theorem]{Lemma}
\newtheorem{proposition}[theorem]{Proposition}
\newtheorem{corollary}[theorem]{Corollary}
\newtheorem{fact}[theorem]{Fact}
\newtheorem{definition}[theorem]{Definition}
\newtheorem{remark}[theorem]{Remark}
\newtheorem{question}[theorem]{Question}
\newtheorem{answer}[theorem]{Answer}
\newtheorem{exercise}[theorem]{Exercise}
\newtheorem{example}[theorem]{Example}
%\newenvironment{proof}{\noindent \textbf{Proof:}}{$\Box$}
\newtheorem{observation}[theorem]{Observation}

%to use newcommand for convenience
\newcommand\field{\mathbb{F}}
\newcommand\Real{\mathbb{R}}
\newcommand\Q{\mathbb{Q}}
\newcommand\Z{\mathbb{Z}}
\newcommand\complex{\mathbb{C}}

%this is how we define operators.
\DeclareMathOperator{\rank}{rank} % rank

\title{Further Discussion of Exercise 4}
\author{Fu Lingyue}
\date{\today}

\begin{document}
    \maketitle
    The exercise 4 put forward that 
    $\mid\mathbb{R} \times \mathbb{R}\mid = \mid \mathbb R\mid$.
    After reading some reference books, I'm trying to prove that 
    $\mid A\times A\mid = \mid A\mid$ for any infinite set $A$.


    \begin{proof}
    First of all, 
    it is clear that the proposition is true when $A$ is countable, 
    for the exercise 4 has been proved and every countable $A$ is isomorphism to $\mathbb R$.
    For $A$ is infinite, we can find a countable $B\subseteq A$ 
    (maybe because $\aleph_0$ is the smallest, I'm not sure).
    And for $\mid B\times B\mid = \mid B\mid$, 
    we can find a bijection $f: B\mapsto B\times B.$

    Then we consider the set of all countable $B \subseteq A$ and
    its corresponding function $f_B$. We denote it as
    $$Z = \{ \langle B,f_B\rangle  \mid B\text{ is the countable subset of }A.\}$$
    Next, we define a partial order $ < $ on set $Z$. 
    $\langle B_1,f_1\rangle  \leq \langle B_2,f_2\rangle $ when the following two conditions are met:
    \begin{enumerate}
        \item[(1)] $B_1 \subset B_2$;
        \item[(2)]  For every $x \in f_1(B_1)$, $x \in f_2(B_2)$ as well.
        \item[(3)]  $f_1$ concides $f_2$ in the domain $B_1$, i.e., 
        $$f_1(b) = f_2(b) \text{ for every }b\in B_1$$  
        
    \end{enumerate}
    
    Here we get a parital order.
    In order to apply Zorn's lemma, we have to find upperbounds of those chains.
    Assume here we have a chain $C_0 \leq C_1 \leq \dots C_n\leq \dots$.
    Denote their union $U$ as 
    $$ U = \bigcup_{i =0}^\infty C_i$$
    And we can also get the combination $g$ of $f_i$, 
    which is a function from $U$ to $U \times U$.

    Claim that $g$ is a bijection on $U$. 
    We begin our prove here.
        
    If $c_i,c_j$ belong to two different set of the chain, 
        then these two set must be $C_i \leq C_j$ or $C_j \leq C_i$.
        In this way, $g(c_i) \not= g(c_j)$ because of the definition(3) of $\leq$.
        Thus $g$ is injective.

    Consider arbitraty $(u,v)\in U \times U.$ 
    If $u \in C_i$ and $v\in C_j$, then $u,v \in C_{max(i,j)}$,
     i.e., $(u,v)\in C_{max(i,j)} \times C_{max(i,j)}$. 
    For every function $f_i$ is bijective, $(u,v)$ must have a preimage in $C_{max(i,j)}$.
    Thus $g$ is surjective.

    Hence, union set $U$ is the upper bound of the chain.
    According to \textbf{Zorn's Lemma}, $Z$ has a maximal element $\langle M,f_M\rangle $.
    $f_M$ is the bijection function from $M$ to $M\times M.$
    Now we have to prove $M$ has the same cardinality with $A$.
    We prove by contradiction as follows.
    \begin{lemma}
        If $A,B$ are two infinite sets, then 
        $$\mid A\mid +\mid B\mid = max\{\mid A\mid,\mid B\mid\}.$$
    
        \begin{proof}
           It can be proved by Cantor-Schröder-Bernstein Theorem, 
           maybe I will finish it in the future.
        \end{proof}
    \end{lemma}

    Assume that $\mid M\mid < \mid A\mid$ ($M\subset A$, then $\mid M\mid \leq \mid A\mid$).
    Denote $R = A\backslash M.$ We can conclude from lemma 1 that 
    $\mid A \mid=\mid M\mid+\mid R\mid= max\{\mid M\mid,\mid R\mid\}.$ 
    For our assumption stipulate $\mid M\mid \not= \mid A\mid$,
     then $\mid R\mid = \mid A\mid$ and $\mid R\mid > \mid M\mid.$
    Find a subset $R'$ of $R$, and $R'$ has the same cardinality as $M$.
    Denote $N = M\cup R'$($M$ and $R'$ are disjoint), then we obtain
    $$\mid N\mid = \mid M\mid + \mid R'\mid = \mid M\mid$$
    Then we extend the function $f_M$ to $f_N: N\mapsto N\times N$ as follows
    \[
        f_N(n) = 
        \begin{cases}
            f_M(n) &  n \in M,\\
            (x,y) \text{ where }(x,y)\in (N\times N)\backslash (M\times M)& n\notin M
        \end{cases}
    \]
        Then we obtain a bijection function $f_N$ of set $N$. 
        (The existence of $f_N$ can be proved by the same cardinary of $M$ and $R'$)
    We can visualize the extension as Figure \ref{extension} shows.
    \begin{figure}[h]
        \centering
        \includegraphics[width = 0.5\textwidth]{extension.jpg}
        \caption{The extension of set $M$}
        \label{extension}
    \end{figure}
    Thus we get a bigger set $N$ and its corresponding bijection function $f_N$.
    This contradicts to the suppose that  $M$ is the maximal set.

    Therefore, the assumption fails and $\mid M\mid = \mid A\mid$. In this case, $M$, $A$, $M\times M$
     and $A\times A$ has the same cardinality.
\end{proof}

\end{document}
