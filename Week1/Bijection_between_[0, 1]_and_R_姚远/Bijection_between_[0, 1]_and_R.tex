% This is a template for lecture notes.
\documentclass{article}
\usepackage[UTF8]{ctex}
\usepackage{amssymb}
\usepackage{amsmath}
\usepackage{amsthm}
\usepackage{geometry}
\usepackage{booktabs}
\usepackage{bm}
\usepackage{tcolorbox}
\CTEXoptions[today=old]
%Some commonly used notations
%\geometry{a4paper,bottom = 3cm,left = 3cm, right = 3cm}

%for reference
\usepackage{hyperref}
\usepackage[capitalise]{cleveref}
\crefname{enumi}{}{}

\newtheorem{theorem}{Theorem}
\newtheorem{lemma}[theorem]{Lemma}
\newtheorem{proposition}[theorem]{Proposition}
\newtheorem{corollary}[theorem]{Corollary}
\newtheorem{fact}[theorem]{Fact}
\newtheorem{definition}[theorem]{Definition}
\newtheorem{remark}[theorem]{Remark}
\newtheorem{question}[theorem]{Question}
\newtheorem{answer}[theorem]{Answer}
\newtheorem{exercise}[theorem]{Exercise}
\newtheorem{example}[theorem]{Example}
%\newenvironment{proof}{\noindent \textbf{Proof:}}{$\Box$}
\newtheorem{observation}[theorem]{Observation}

%to use newcommand for convenience
\newcommand\field{\mathbb{F}}
\newcommand\Real{\mathbb{R}}
\newcommand\Q{\mathbb{Q}}
\newcommand\Z{\mathbb{Z}}
\newcommand\complex{\mathbb{C}}

%this is how we define operators.
\DeclareMathOperator{\rank}{rank} % rank

\title{Bijection between $[0, 1]$ and $\Real$}
\author{Yao Yuan}
\date{\today}

\begin{document}
\maketitle

\begin{tcolorbox}
    \begin{theorem} 
        Find an explicit bijection from $[0, 1]$ to $\Real$.
    \end{theorem} 
\end{tcolorbox}

\begin{fact}\label{fact:real}
    There exists a bijection between $(0, 1)$ and $\Real$.
\end{fact}

\begin{proof}
    Function $f: y = \tan(\pi x - \frac\pi2)$ is obviously a bijection between $(0, 1)$ and $\Real$.
\end{proof}

\begin{fact}\label{fact:extension}
    There exists a bijection between $(0, 1)$ and $[0, 1]$.
\end{fact}

\begin{proof}
    Choose an infinite sequence $(x_n)_{n \geq 1}$ of distinct elements of $(0,1)$. 
    Let $X=\{x_{n} \mid n \geq 1\}$, hence $X \subset (0,1) .$ Let $x_0=1 .$ Define $f(x_n) = x_{n+1} $ for every $n \geq 0$ and $ f(x) = x $ for every $ x $ in $(0,1) \backslash X$. Then $f$ is defined on $(0,1]$ and the map $f:(0,1] \rightarrow(0,1)$ is bijective.
    
    Similarly, we can find a bijection between $(0, 1]$ and $[0, 1]$. Thus there exists a bijection between $(0, 1)$ and $[0, 1]$.
\end{proof}

%use \cref instead of \ref
Combine \cref{fact:real} and \cref{fact:extension}
and we get a bijection between $[0, 1]$ and $\Real$.

\end{document}