\documentclass{ximera}
\usepackage[UTF8]{ctex}

\title{Basic Probability Theory}
\author{刘成锴}

\begin{document}
	
\begin{abstract}
	Week 1
	
	刘成锴
	
	学号:518030910425
\end{abstract}
\maketitle

\section{Exercises}

\subsection{Exercise 3}
	
\begin{solution}
	$$
	f(x) = \tan({x - \frac{1}{2}})\pi
	$$
	
	$f:[0, 1] \rightarrow \mathbb{R}$ is a simple injection.
	
	$$
	g(x)=\left\{
	\begin{array}{lc}
	0 & x = -\infty \\ 
	\frac{\arctan(x)}{\pi} + \frac{1}{2} & -\infty < x < \infty \\ 
	1 & x = \infty
	\end{array}\right.
	$$
	
	$g:\mathbb{R} \rightarrow [0, 1]$ is a simple injection.
	
	Hence, we find an explicit bijection from [0, 1] to $\mathbb{R}$
\end{solution}

\subsection{Exercise 4}
\begin{solution}
	I would like to prove $ |\mathbb{R}| = |\mathbb{R}^2| $.
	$$
	|\mathbb{R}| \rightarrow |\mathbb{R}^2|
	$$
	For any $a \in \mathbb{R}$, there is $(a, a) \in \mathbb{R}^2$. It's an injection.
	
	$$
	|\mathbb{R}^2| \rightarrow |\mathbb{R}|
	$$
	利用Exercise 3 中的$g$,我们可以将一个属于$\mathbb{R}$中的数映射到区间[0, 1]上。
	
	Then, for any $(a, b) \in \mathbb{R}^2$, 我们可以通过映射,得到$(g(a), g(b))$。
	
	将$g(a), g(b)$分别表示为:$0.a_1a_2a_3...$ 和 $0.b_1b_2b_3...$。
	
	然后我们再找一个映射,得到一个实数$c$,且$c \in [0, 1]$。$c = 0.a_1b_1a_2b_2a_3b_3...$。
	
	综合以上映射,we get an injection from $\mathbb{R}^2$ to $\mathbb{R}$。
	
	Hence, there is an bijection from $\mathbb{R}^2$ to $\mathbb{R}$.
	
	That is, $ |\mathbb{R}| = |\mathbb{R}^2|$.
	
\end{solution}
	

\end{document}