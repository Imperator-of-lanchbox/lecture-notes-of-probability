\documentclass[UTF8]{article}

\usepackage[T1]{fontenc}
\usepackage{textcomp}
\usepackage{theorem}
% \usepackage[dutch]{babel}
\usepackage{amsmath, amssymb}
\usepackage{import}
\usepackage{pdfpages}
\usepackage{transparent}
\usepackage{xcolor}
\usepackage{enumerate}
\usepackage{setspace} 
%\usepackage{ebgaramond}
%\fontfamily{ebgaramond}
% ------------- coding style setting --------------%
%\usepackage{libertine}
\usepackage{listings}
\usepackage{enumitem}
\setlist{nosep}
\definecolor{codegreen}{rgb}{0,0.6,0}
\definecolor{codegray}{rgb}{0.5,0.5,0.5}
\definecolor{codepurple}{rgb}{0.58,0,0.82}
\definecolor{backcolour}{rgb}{0.95,0.95,0.92}

\lstdefinestyle{mystyle}{
    backgroundcolor=\color{backcolour},
    commentstyle=\color{codegreen},
    keywordstyle=\color{magenta},
    numberstyle=\tiny\color{codegray},
    stringstyle=\color{codepurple},
    basicstyle=\ttfamily\footnotesize,
    breakatwhitespace=false,
    breaklines=true,
    captionpos=b,
    keepspaces=true,
    numbers=left,
    numbersep=5pt,
    showspaces=false,
    showstringspaces=false,
    showtabs=false,
    tabsize=2
}

\lstset{style=mystyle}

% ----------------- geometry and fancy head -----------

\usepackage{geometry}
\geometry{left=2.5cm,right=2.5cm,top=3cm,bottom=3cm}
\usepackage[many]{tcolorbox}
\tcbuselibrary{skins, breakable, theorems}

\usepackage{fancyhdr}
\usepackage{syntonly} % dubugging
% \syntaxonly
\fancypagestyle{mainFancy}{
    \fancyhf{}
    %\renewcommand\headrulewidth{0pt}       % 页眉横线
    %\renewcommand\footrulewidth{0pt}
    
    \fancyhead[L]{Probability Theory}       % 页眉章标题
    \fancyhead[R]{Assignment}         % 页眉文章题目
    \fancyfoot[C]{\thepage}                 % 页眉编号
}
\pagestyle{mainFancy}


% --------------- environment setting ------------------

\newtheorem{thm}{Theorem}
\newtheorem{pro}{Problem}
\newtheorem{lemma}{Lemma}
\newtheorem{defi}{Definition}
\newtheorem{li}{Example}
\newenvironment{proof}{\paragraph{Proof:}}{\hfill$\square$}
\newenvironment{jie}{\paragraph{Show:}}{\hfill$\square$}

\numberwithin{pro}{section}
\numberwithin{thm}{section}
\numberwithin{defi}{section}
\numberwithin{lemma}{section}


\tcolorboxenvironment{pro}{
  enhanced,
  borderline={0.4pt}{0.4pt}{black},
  boxrule=0.4pt,
  colback=white,
  coltitle=black,
  sharp corners,
}
\tcolorboxenvironment{thm}{
  enhanced,
  borderline={0.4pt}{0.4pt}{black},
  boxrule=0.4pt,
  colback=white,
  coltitle=black,
  sharp corners,
}
\tcolorboxenvironment{lemma}{
  enhanced,
  borderline={0.4pt}{0.4pt}{black},
  boxrule=0.4pt,
  colback=white,
  coltitle=black,
  sharp corners,
}
\tcolorboxenvironment{defi}{
  enhanced,
  borderline={0.4pt}{0.4pt}{black},
  boxrule=0.4pt,
  colback=white,
  coltitle=black,
  sharp corners,
}

% ----------------- macros and command -----------------
\usepackage{stmaryrd} 
\newcommand\contra{\scalebox{1.5}{$\lightning$}}
\definecolor{correct}{HTML}{009900}
\newcommand\correct[2]{\ensuremath{\:}{\color{red}{#1}}\ensuremath{\to }{\color{correct}{#2}}\ensuremath{\:}}
\newcommand\green[1]{{\color{correct}{#1}}}

% horizontal rule
\newcommand\hr{
		    \noindent\rule[0.5ex]{\linewidth}{0.5pt}
	}
\def\mf(#1){\mathfrak{#1}} 
\def\setn(#1,#2){\left\{#1_1,#1_2,\cdots, #1_#2 \right\}  }


\let\implies\Rightarrow
\let\impliedby\Leftarrow
\let\iff\Leftrightarrow
\let\ldots\cdots


\newcommand\dif{\,\mathrm{d}}
\newcommand\e{\,\mathrm{e}}
\newcommand\R{\,\mathbb{R}}
\newcommand\Q{\,\mathbb{Q}}
\newcommand\C{\,\mathbb{C}}
\newcommand\N{\,\mathbb{N}}
\newcommand\A{\,\mathbb{A}}
\newcommand\Z{\,\mathbb{Z}}
\newcommand\ep{\,\varepsilon}
\newcommand\F{\,\varphi}
\newcommand\T{\,\mathbb{T}}
\newcommand\HH{\,\mathbb{H}}
\author{Yujie Lu \quad Haichen Dong \\ \textsc{ACM Class 18} }

\title{\huge\textsc{Discussions for Exercise 8}}
\begin{document}
\maketitle

\section{A natural thought}
 \begin{pro}
		 View $\R^{k}$ as a poset which $\left( x_1,x_2,\ldots x_{k}\right) \le 
		 \left( y_1,y_2,\ldots,y_{k} \right) $ if and only if
		 $y-x\ge 0$ namely $\forall i,y_i \ge  x_i$. Is there
		 any injective poset homomorphism from $\R^{3}$ to $\R^{2}$?
 \end{pro}



\begin{proof}
		First notice that every real number in $(0,1)$ could be represented as 
	\[
	\sum_{i=1}^{\infty} a_i 10^{-i}
	.\] 
	Which would be helpful to construct a mapping. Consider $\forall \left( a,b,c \right) \in (0,1)^{3}  $, their representations
	are 
	\[
			\sum_{i=1}^{\infty} a_i 10^{-i}, \sum_{i=1}^{\infty} b_i 10^{-i},
	\sum_{i=1}^{\infty} c_i 10^{-i}
	.\] 
	We consider a mapping 
	\begin{align*}
			f: (0,1)^{3} &\to (0,1)^{2}  \\
			\left( a,b,c \right) &\to \left( d,e \right) 
	.\end{align*}
	which satisfies that
	\begin{align*}
			d &= \sum_{i=1}^{\infty} a_i 10^{-2i+1} + \sum_{i=1}^{\infty} b_i 10^{-2i} \\
			e &= c
	.\end{align*}
	$f$ is obviously a bijection. Now we prove it is poset homomorphism. 
	We only need to prove that if $\left( a,b,c \right) \le \left( a',b',c' \right) $ then 
	\[ f\left( a,b,c \right)  = \left( d,e \right) \le  \left( d',e' \right) = f\left( a',b',c' \right) \]
	It's clear that $e=c\le c' = e'$. And now consider the relationship between $d,d'$.
	Knowing that  $a\le a'$, thus $a=a'$ or $\exists k$ s.t. 
	\[
	a_j = a'_j, \forall j < k \quad a_k <  a'_k
	.\] 
	Similarly for $b\le b'$ there will be a $l$. 
	If $a\neq a',b\neq b'$, let $s = \mathrm{min}\left( k,l \right) $, then 
	\[
	d_j = d'_j, \forall j < s, \quad d_s < d'_s
	.\] 
	Hence $d\le d'$. 
	Otherwise if $a=a'$ or  $b=b'$, we can discuss these cases similarly.
	That is $\left( d,e \right) \le \left( d',e' \right) $.

	Now we extend the mapping to $\R^{3}\to \R^{2}$. Just notice that there are 2 bijections
	\begin{align*}
			g_1: \left( -\infty,\infty \right) &\mapsto  (0,\infty)\\
			x\mapsto &\mathrm{e} ^{x} 
	.\end{align*}
	and 
	\begin{align*}
			g_2: \left( 0,\infty \right) &\mapsto (0,1)   \\
			x &\mapsto \frac{2}{\pi} \tan^{-1} x
	.\end{align*}
	And most importantly, $g_1,g_2$ are both monotonic functions, also poset homomorphisms.

	To sum up, the homomorphism I constructed is 
	\[
	g_1 \circ g_2 \circ f \circ g_2^{-1} \circ g_1 ^{-1}
	.\] 
\end{proof}

This construction, however, is a typical wrong solution. Since $\frac{1}{2}$ could 
be represented either as $0.5000\ldots$ or $0.49999\ldots$. Thus it's almost impossible
to construct a bijection in this way. 

Cantor originally tried this method himself, but Dedekind pointed out the problem of nonunique decimal representations.

Nevertheless, if we always choose the latter representation, i.e. the one that ends
with $999\ldots$, we could construct 
an injection.  

In order to constrcut a bijection, \textbf{Haichen Dong}  has come up with a revision to
this method.
\section{Bijection between $\R$ and  $\R^2$}

\begin{thm}
    There is a bijection $f$ between $\R $ and $\R ^2$, which implies $|\R | = |\R ^2|$.
\end{thm}

\begin{proof}
    First we construct a bijective map from $\R $ to $(0, 1)$:
    \begin{align*}
        g:& \R\to (0,1) \\
          &= x \mapsto \frac{2 \tan^{-1} x}{\pi} 
    .\end{align*}

    The next step is to find a bijection from $(0, 1)$ to $(0, 1]$, which can be obtained by
    \begin{align*}
        h:& (0,1) \to (0,1] \\
          & x \mapsto \begin{cases}
              2^{-n-1}, x = 2^{-n} \\
              x, \text{ otherwise }
          \end{cases} 
    .\end{align*}

    And also the 2-dimensional version $g', h'$.

    This construction will work with any sequence $\{a_1, a_2, \cdots \} $ with $a_1 = 1$ and $\lim_{n\to \infty} a_n = 0$.

    The final step is to fix the problem brought by the construction that maps $(0.a_1a_2a_3\cdots, 0.b_1b_2b_3\cdots ) $ to $(0.a_1b_1a_2b_2\cdots )$

    That problem exists because finite numbers, for example, $\frac{1}{2}$, can be written as both $0.500\cdots $ and $0.499\cdots $. So we make the choice that always written finite numbers with infinite $9$s trailing.

    But without further construction, there will be no inverse image for $0.5303030\cdots $. Noticing that there will never be infinite zero string in the resulting number. So we can involve the zero string together with the non-zero digit after it.

    For example, $0.53030303\cdots = 0.[5][3][03][03][03]\cdots$, is mapped by $(0.50303\cdots ,0.30303\cdots )$. Call this mapping $\phi $, and finally we get the bijection:
    \begin{align*}
        f: &\R ^2 \to \R \\
           &x \mapsto (g^{-1} \circ h^{-1} \circ \phi \circ h' \circ g' )(x)
    .\end{align*}
\end{proof}


\section{Future work}
By our trivial thought, we could find a injective poset homomorphism. 
But how about surjections?
\begin{pro}
		 View $\R^{k}$ as a poset which $\left( x_1,x_2,\ldots x_{k}\right) \le 
		 \left( y_1,y_2,\ldots,y_{k} \right) $ if and only if
		 $y-x\ge 0$ namely $\forall i,y_i \ge  x_i$. Is there
		 a subposet of $\R^{3}$ poset which is isomorphic with $\R^{2}$?
\end{pro}

\end{document}
