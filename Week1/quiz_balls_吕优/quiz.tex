\documentclass[UTF8]{ctexart}
\usepackage{amsmath}
\usepackage{amssymb}
\usepackage{amsthm}
\usepackage{graphicx}
\usepackage{CJK}
\usepackage{float}
\usepackage{mdframed}
\providecommand{\abs}[1]{\lvert#1\rvert}
\providecommand{\norm}[1]{\lVert#1\rVert}
\providecommand{\ud}[1]{\underline{#1}}

\newmdtheoremenv{thm}{Theorem}
\newmdtheoremenv{lemma}[thm]{Lemma}
\newmdtheoremenv{fact}[thm]{Fact}
\newmdtheoremenv{cor}[thm]{Corollary}
\newtheorem{eg}{Example}
\newtheorem{ex}{Exercise}
\newmdtheoremenv{defi}{Definition}
\newenvironment{sol}
  {\par\vspace{3mm}\noindent{\it Solution}.}
  {\qed \\ \medskip}

\newcommand{\ov}{\overline}
\newcommand{\ca}{{\cal A}}
\newcommand{\cb}{{\cal B}}
\newcommand{\cc}{{\cal C}}
\newcommand{\cd}{{\cal D}}
\newcommand{\ce}{{\cal E}}
\newcommand{\cf}{{\cal F}}
\newcommand{\ch}{{\cal H}}
\newcommand{\cl}{{\cal L}}
\newcommand{\cm}{{\cal M}}
\newcommand{\cp}{{\cal P}}
\newcommand{\cs}{{\cal S}}
\newcommand{\cz}{{\cal Z}}
\newcommand{\eps}{\varepsilon}
\newcommand{\ra}{\rightarrow}
\newcommand{\la}{\leftarrow}
\newcommand{\Ra}{\Rightarrow}
\newcommand{\dist}{\mbox{\rm dist}}
\newcommand{\bn}{{\mathbb N}}
\newcommand{\bz}{{\mathbb Z}}

\newcommand{\expe}{{\mathsf E}}
\newcommand{\pr}{{\mathsf{Pr}}}


\setlength{\parindent}{0pt}
%\setlength{\parskip}{2ex}
\newenvironment{proofof}[1]{\bigskip\noindent{\itshape #1. }}{\hfill$\Box$\medskip}

\theoremstyle{definition}
\newtheorem{problem}{Problem}
\newtheorem*{problem*}{Problem}
\newtheorem{solution}{Solution}
\newtheorem*{solution*}{Solution}

\begin{document}

\title{Discussions on some quizzes}
\author{吕优 518030910427}
\date{2020 3 9}

\maketitle

\begin{problem}
    罐子里有 70 个黑球和 30 个白球。每次从中取一个球直到罐子中只含单色球为止。最后罐子中剩的都是白球的概率为多少?
\end{problem}

\begin{solution}
    我们假设罐子里有$n$个黑球和$m$个白球。
    我们可以先考虑这样一个操作,每次从罐子中取一个球直到罐子取空为止,那这样操作得到的序列一共有$\frac{n+m!}{n!m!}$种
    然后为了满足题目的条件,我们删除序列最后一段连续的同色球,假如最后一个球是黑色,我们就从后一直删到球变成白色为止
    我们会发现得到的新序列满足题目的条件,并且两两序列互不相同。这样,每一个序列都代表着一种合法的取球方案,我们考虑
    最后的一段序列是白球的情况个数,这个数目就等于之前序列中最后一个球是白球的数目,一共有$\frac{(n+m-1)!}{n!(m-1)!}$种
    所以最后罐子中剩的都是白球的概率为$\frac{\frac{(n+m-1)!}{n!(m-1)!}}{\frac{n+m!}{n!m!}}=\frac{m}{n+m}$,即为
    白色球占所有球的比例,而且通过上述讨论不难发现,如果罐子中有若干种颜色的球,按题目的要求操作,最后罐子种所剩的球为
    某一特定颜色球的概率就是这种颜色的球所占的比例。
\end{solution}

\end{document}

